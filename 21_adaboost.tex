\def\err{{\mathop{\overline{\mathsf{err}}}}}

\section{AdaBoost}

%\subsection{Boosting}

\begin{frame}
  \frametitle{Boosting Methods}

  \begin{itemize}
    \item Boosting is one of the most powerful learning techniques introduced in the last twenty years.
    \item Combines output of many \structure{\textit{weak} classifiers} to produce a powerful committee.
    \item The most popular boosting algorithm is called \structure{\textit{AdaBoost}} \\
      (Freund and Schapire, 1997).
  \end{itemize}
\end{frame}


\begin{frame}
  \frametitle{Boosting Methods \cont}

  \begin{citeblock}{Definition}

    A \structure{weak classifier} is one whose error rate is only slightly better than random guessing.
  \end{citeblock}
  \pspread

  \structure{Idea of boosting:}

  \begin{itemize}
    \item Sequentially apply the \emph{weak} classifier to repeatedly modified versions of the data.
    \item This produces a sequence of classifiers.
    \item Weighted majority vote yields the final prediction.
  \end{itemize}
\end{frame}


\begin{frame}
  \frametitle{Boosting Methods \cont}

  \begin{itemize}
    \item Consider a two-class problem with $y \in \{ -1, +1\}$. \\[.25cm] %\pause
    \item Given a set of observations $\mathcal{D} = \{(\vec{x}_i, y_i); \quad i=1,\ldots,N\}$ and a classifier $G(\vec x)$, the \structure{error rate} on the training sample is:
      \begin{displaymath}
        \err = \frac{1}{N} \sum_{i=1}^N I(y_i \neq G(\vec{x}_i))
      \end{displaymath}
      where $I$ is the \structure{indicator function}.
  \end{itemize}
\end{frame}


\begin{frame}
  \frametitle{Boosting Methods \cont}

  \begin{itemize}
    \item Sequentially applied weak classifiers produce a sequence $G_m(\vec x)$ with $m=1,2,\ldots,M$ %\pause
    \item The \structure{combined prediction} is then:
      \begin{displaymath}
        G(\vec x) = \sign \left( \sum_{m=1}^M \alpha_m G_m(\vec x) \right)
      \end{displaymath}
      %\pause
    \item The weighting factors $\alpha_1, \ldots, \alpha_M$ are computed by the boosting algorithm. %\pause
    \item Each $\alpha_m$ weights the output of the corresponding classifier $G_m(\vec x)$.
  \end{itemize}
\end{frame}


\begin{frame}
  \frametitle{Boosting Methods}

  \structure{Schematic of boosting procedure:}

  \begin{center}
    \resizebox{.35\linewidth}{!}{
      \input{\texfigdir/boosting_schematic.pstex_t}
    }
  \end{center}
\end{frame}


\begin{frame}
  \frametitle{Boosting Methods \cont}

  \structure{Modifications on the data:}

  \begin{itemize}
    \item Each boosting step consists of applying weights $w_1, w_2, \ldots, w_N$ to the training samples. %\pause
    \item Initially, the weights are set to $w_i = 1/N$. %\pause
    \item Thus the first classifier in the sequence is trained the usual way. %\pause
    \item For $m \geq 2$ the weights are individually modified. %\pause
    \item The classifiers $G_m$, with $m \geq 2$ are trained on differently-weighted samples.
  \end{itemize}
\end{frame}


\begin{frame}
  \frametitle{Boosting Methods}

  \structure{Weighting scheme:}

  \begin{itemize}
    \item At step $m$ the misclassified observations from $G_{m-1}$ have increased weights.
    \item The weights for correctly classified samples from $G_{m-1}$ have decreased weights.
  \end{itemize}
  \pspread

  \structure{Effects of the weighting scheme:}

  \begin{itemize}
    \item Observations that are difficult to classify correctly get ever-increasing influence.
    \item Each successive classifier is forced to concentrate more on those observations that were misclassified by the previous one.
  \end{itemize}
\end{frame}


\subsection{AdaBoost}

\begin{frame}
  \frametitle{AdaBoost}

  \structure{AdaBoost algorithm:}
%
  \begin{centernss}
    \small
    \resizebox{.85\linewidth}{!}{
      \begin{struktogramm}(100,50)
        \assign{Initialize weights: $w_i =\gets 1/N ~,~i=1, \ldots, N$}
        \assign{$m \gets 1$}
        \until{$ m = M$}
          \assign{\structure{Fit classifier $G_m(\vec x)$ to training data using $\vec w$}}
          \assign{\structure{Compute classification error:}\\
            $ \mathsf{err}_m = \frac{\sum_{i=1}^N w_i I(y_i \neq G_m(x_i))}{\sum_{i=1}^N w_i}$
          }
          \assign{\structure{Compute classifier weights:}\\
            $ \alpha_m = \log \left( \frac{1- \mathsf{err}_m}{\mathsf{err}_m} \right) $
          }
          \assign{\structure{Compute new sample weights:}\\
            $ w_i \leftarrow w_i \exp \left[ \alpha_m I(y_i \neq G_m(x_i)) \right] ~,~ i = 1, \ldots, N $
          }
          \assign{$m \gets m + 1$}
        \untilend
        \assign{\structure{Output:} $G(\vec x) = \sign \left( \sum_{m=1}^M \alpha_m G_m(\vec x) \right) $}
      \end{struktogramm}
    }
  \end{centernss}

%   \begin{center}
%     \small
%     \begin{struktogramm}{10cm}{0.7cm}
%       \BLOCK {Initialize weights: $w_i = 1/N ~,~i=1, \ldots, N$}
%       \BLOCK {Set $m:=1$}
%       \REPEAT {
%         \BLOCK {\structure{Fit classifier $G_m(\vec x)$ to training data using $\vec w$}}
%         \BLOCK {\structure{Compute classification error:}\\
%           $ \mathsf{err}_m = \frac{\sum_{i=1}^N w_i I(y_i \neq G_m(x_i))}{\sum_{i=1}^N w_i}$
%         }
%         \BLOCK {\structure{Compute classifier weights:}\\
%           $ \alpha_m = \log \left( \frac{1- \mathsf{err}_m}{\mathsf{err}_m} \right) $
%         }
%         \BLOCK {\structure{Compute new sample weights:}\\
%           $ w_i \leftarrow w_i \exp \left[ \alpha_m I(y_i \neq G_m(x_i)) \right] ~,~ i = 1, \ldots, N $
%         }
%         \BLOCK {Set $m:= m + 1$}
%       }
%       UNTIL {$ m = M$}
%       \BLOCK {\structure{Output:} $G(\vec x) = \sign \left( \sum_{m=1}^M \alpha_m G_m(\vec x) \right) $}
%     \end{struktogramm}
%   \end{center}
\end{frame}


\begin{frame}
  \frametitle{AdaBoost \cont}

  \structure{Notes:}

  \begin{itemize}
    \item This version of AdaBoost is called \structure{discrete}, because each $G_m(\vec x)$ returns a discrete class label. %\pause
    \item AdaBoost can be modified to return also a real-valued prediction in the interval $[-1,+1]$.  %\pause
    \item Instead of taking just any classifier for $G_m(\vec x)$, that classifier may be used that results in the smallest error at step $m$.
  \end{itemize}
  \pspread

  AdaBoost dramatically increases the performance of even a very weak classifier.
\end{frame}


\begin{frame}
  \frametitle{AdaBoost \cont}

  \structure{Example from J.\ Matas and J.\ \v{S}ochman \\(Centre for Machine Perception, Technical University, Prague):}
%
  \begin{center}
    \resizebox{!}{6.5cm}{
      \input{\texfigdir/adaboost1.pstex_t}
    }
  \end{center}
\end{frame}


\begin{frame}
  \frametitle{AdaBoost \cont}

  \structure{Example from J.\ Matas and J.\ \v{S}ochman \\(Centre for Machine Perception, Technical University, Prague):}

  \begin{center}
    \resizebox{!}{6.5cm}{
      \alt<2->{
        \input{\texfigdir/adaboost3.pstex_t}
      }{
        \input{\texfigdir/adaboost2.pstex_t}
      }
    }
  \end{center}
\end{frame}


\begin{frame}
  \frametitle{AdaBoost \cont}

  \structure{Example from J.\ Matas and J.\ \v{S}ochman \\(Centre for Machine Perception, Technical University, Prague):}
%
  \begin{center}
    \resizebox{!}{6.5cm}{
      \input{\texfigdir/adaboost4.pstex_t}
    }
  \end{center}
\end{frame}


\begin{frame}
  \frametitle{AdaBoost \cont}

  \structure{Example from J.\ Matas and J.\ \v{S}ochman \\(Centre for Machine Perception, Technical University, Prague):}

  \begin{columns}[c, onlytextwidth]
    \begin{column}{0.5\textwidth}
      \begin{figure}
        \centering
        \resizebox{.7\linewidth}{!}{
          \alt<8->{
            \input{\texfigdir/adaboost-classifier.pstex_t}
          }{\alt<7>{
            \input{\texfigdir/adaboost-classifier7.pstex_t}
          }{\alt<6>{
            \input{\texfigdir/adaboost-classifier6.pstex_t}
          }{\alt<5>{
            \input{\texfigdir/adaboost-classifier5.pstex_t}
          }{\alt<4>{
            \input{\texfigdir/adaboost-classifier4.pstex_t}
          }{\alt<3>{
            \input{\texfigdir/adaboost-classifier3.pstex_t}
          }{\alt<2>{
            \input{\texfigdir/adaboost-classifier2.pstex_t}
          }{
            \input{\texfigdir/adaboost-classifier1.pstex_t}
          }}}}}}}
        }
        \caption{\footnotesize Final classifier (based on a perceptron).}
      \end{figure}
    \end{column}\begin{column}{0.5\textwidth}
      \begin{figure}
        \centering
        \resizebox{\linewidth}{!}{
          \alt<8->{
            \input{\texfigdir/adaboost-error.pstex_t}
          }{\alt<7>{
            \input{\texfigdir/adaboost-error7.pstex_t}
          }{\alt<6>{
            \input{\texfigdir/adaboost-error6.pstex_t}
          }{\alt<5>{
            \input{\texfigdir/adaboost-error5.pstex_t}
          }{\alt<4>{
            \input{\texfigdir/adaboost-error4.pstex_t}
          }{\alt<3>{
            \input{\texfigdir/adaboost-error3.pstex_t}
          }{\alt<2>{
            \input{\texfigdir/adaboost-error2.pstex_t}
          }{
            \input{\texfigdir/adaboost-error1.pstex_t}
          }}}}}}}
        }
        \caption{\footnotesize Reduction of the classification error w.\,r.\,t.\ the sequence length.}
      \end{figure}
  \end{column}
  \end{columns}
\end{frame}

{
\usebackgroundtemplate{\includegraphics[width=\paperwidth]{png/nextTime.png}}
\begin{frame}[plain]
\end{frame}
}


\subsection{Exponential Loss}

\begin{frame}
  \frametitle{Exponential Loss}

  \begin{itemize}
    \item Boosting fits an \structure{additive model} in a \structure{set of elementary basis functions}: \\[.1cm]
      \begin{center}
        \tikz[baseline]{
          \node[fill=bl1!100, anchor=base, rounded corners=3pt, inner sep=1mm] (d1) {
            \color{bl3}
            $
              \displaystyle
              f_M(\vec x) = 
              \sum_{m=1}^M \beta_m \, b(\vec x ; \vec \gamma_m)
            $
          };
        }
      \end{center}
      \vspace{.1cm}
      where $\beta_m$ are expansion coefficients and $b(\vec x ; \vec \gamma_m)$ is a basis function given a set of parameters $\vec \gamma_m$. \\[.3cm] \pause
    \item Additive expansions are very popular in learning techniques:
      \begin{itemize}
        \item single-hidden-layer neural networks (perceptron)
        \item wavelets
        \item classification trees
        \item etc.
      \end{itemize}
  \end{itemize}
\end{frame}


\begin{frame}
  \frametitle{Exponential Loss \cont}

  \begin{itemize}
    \item Expansion models are typically fit by \structure{minimizing a loss function $L$} averaged over the training data:
      \begin{equation} 
        \label{eq:loss}
        \min_{\{ \beta_m, \vec \gamma_m \}_1^M } ~ 
        \left\{
          \sum_{i=1}^N L \big( y_i, f_M(\vec{x}_i) \big) =
          \sum_{i=1}^N L \left( y_i, \sum_{m=1}^M \beta_m \, b(\vec{x}_i ; \vec \gamma_m) \right)
        \right\}
      \end{equation}
      \\[.3cm] \pause
    \item \structure{Forward stagewise modeling} approximates the solution to \eqref{eq:loss}: \\[.1cm]
      \begin{itemize}
        \item New basis functions are sequentially added. \\[.1cm] %\pause
        \item Parameters and coefficients of already added functions are not changed. \\[.1cm] %\pause
        \item At each iteration, the subproblem of fitting just a single basis function is solved.
      \end{itemize}
%       \begin{displaymath}
%         \min_{\beta, \vec \gamma} ~\sum_{i=1}^N L(y_i, \beta b(\vec{x}_i ; \vec{\gamma}))
%        \end{displaymath}
  \end{itemize}
\end{frame}


\begin{frame}
  \frametitle{Exponential Loss \cont}

  \begin{itemize}
    \item The \structure{$m$-th subproblem} may be rewritten as: \\[.3cm]
      \begin{center}
        \tikz[baseline]{
          \node[fill=bl1!100, anchor=base, rounded corners=3pt, inner sep=1mm] (d1) {
            \color{bl3}
            $
              \displaystyle
              (\beta_m, \vec \gamma_m) = \argmin_{\beta, \vec \gamma} ~\sum_{i=1}^N L\big(y_i, f_{m-1}(\vec{x}_i) + \beta \, b(\vec{x}_i ; \vec{\gamma})\big)
            $
          };
        }
      \end{center}
      \vspace{.3cm} \pause
    \item AdaBoost is equivalent to a forward stagewise additive modeling using an \structure{exponential loss function}:
      \begin{displaymath}
        L(y, f(\vec x)) = \exp (-y \, f(\vec x))
      \end{displaymath}
  \end{itemize}
\end{frame}


\begin{frame}
  \frametitle{Exponential Loss \cont}

  \structure{Proof:}

  \begin{itemize}
    \item For AdaBoost, the basis functions are the classifiers $G_m(\vec x) \in \{-1,+1\}$. %\pause
    \item Using the \structure{exponential loss function}, one must solve at each step:
      \begin{eqnarray*}
        (\beta_m, G_m) = \argmin_{\beta, G} ~\sum_{i=1}^N \exp \left[ -y_i \big(f_{m-1}(\vec{x}_i) + \beta G(\vec{x}_i)\big) \right]
      \end{eqnarray*}
      %\pause
    \item Using the weight $w_i^{(m)} = \exp(-y_i \, f_{m-1}(\vec{x}_i))$ this can be rewritten as:
      \begin{eqnarray*}
        (\beta_m, G_m) = \argmin_{\beta, G} ~\sum_{i=1}^N w_i^{(m)} \exp (-\beta y_i G(\vec{x}_i))
      \end{eqnarray*}
  \end{itemize}
\end{frame}


\begin{frame}
  \frametitle{Exponential Loss \cont}

  \structure{Observations:}

  \begin{itemize}
    \item Since $w_i^{(m)}$ is independent of $\beta$ and $G(\vec x)$, it can be seen as a weight applied to each observation. %\pause
    \item However, this weight depends on $f_{m-1}$, so the weights change with \\
      each iteration $m$.
  \end{itemize}
\end{frame}


\begin{frame}
  \frametitle{Exponential Loss \cont}

  \vspace{-.5cm}
  \begin{eqnarray*}
    (\beta_m, G_m) 
      &=& \argmin_{\beta, G} ~\sum_{i=1}^N w_i^{(m)} \exp (-\beta y_i G(\vec{x}_i)) \\ %\pause
      &=& \argmin_{\beta, G} ~ e^{-\beta} \sum_{y_i = G(\vec{x}_i)} w_i^{(m)} + e^{\beta} \sum_{y_i \neq G(\vec{x}_i)} w_i^{(m)} \\ %\pause
      &=& \argmin_{\beta, G} ~ \left( e^{\beta} - e^{-\beta} \right) \sum_{i=1}^N w_i^{(m)} I(y_i \neq G(\vec{x}_i)) + e^{-\beta} \sum_{i=1}^N w_i^{(m)}
  \end{eqnarray*}
  \pause

  \begin{itemize}
    \item Thus, for any value $\beta > 0$ the solution for $G_m(\vec x)$ is: \\[.25cm]
      \begin{center}
        \tikz[baseline]{
          \node[fill=bl1!100, anchor=base, rounded corners=3pt, inner sep=1mm] (d1) {
            \color{bl3}
            $
              \displaystyle
              G_m = \argmin_G \sum_{i=1}^N w_i^{(m)} \, I(y_i \neq G(\vec{x}_i))
            $
          };
        }
      \end{center}
  \end{itemize}
\end{frame}


\begin{frame}
  \frametitle{Exponential Loss \cont}

  \begin{itemize}
    \item Plugging the reformulated $G_m(\vec x)$ into the objective function and solving for $\beta_m$ yields: \\[.15cm]
      \begin{center}
        \tikz[baseline]{
          \node[fill=bl1!100, anchor=base, rounded corners=3pt, inner sep=1mm] (d1) {
            \color{bl3}
            $
              \displaystyle
              \beta_m = \frac{1}{2} \log \frac{1 - \mathsf{err}_m}{\mathsf{err}_m}
            $
          };
        }
      \end{center}
      \vspace{.25cm}
      
      with $\mathsf{err}_m$ being the \structure{minimized weighted error rate}: \\[.15cm]
      
      \begin{displaymath}
        \mathsf{err}_m = \frac{\sum_{i=1}^N w_i^{(m)} I(y_i \neq G_m(\vec{x}_i))}{\sum_{i=1}^N w_i^{(m)}}
      \end{displaymath}
  \end{itemize}
\end{frame}


\begin{frame}
  \frametitle{Exponential Loss \cont}

  \begin{itemize}
    \item From the update formula of the approximation:
      \begin{eqnarray*}
        f_m(\vec x) = f_{m-1}(\vec x) + \beta_m \, G_m(\vec x)
      \end{eqnarray*}
      we can calculate the \structure{weights for the next iteration}:
      \begin{eqnarray*}
        w_i^{(m+1)} 
          &=& w_i^{(m)} ~ e^{-\beta_m y_i G_m(\vec{x}_i)} \\ \vspace{1cm} %\pause
          & & \mbox{\small(using: $-y_i G_m(\vec{x}_i) = 2 I (y_i \neq G_m(\vec{x}_i)) - 1$)} \\[.5cm] %\pause
          &=& w_i^{(m)} e^{\alpha_m I(y_i \neq G_m(\vec{x}_i))} e^{-\beta_m}
      \end{eqnarray*}
      with $\alpha_m = 2 \beta_m$.
  \end{itemize}

  \hfill \qed
\end{frame}


\begin{frame}
  \frametitle{Exponential Loss \cont}

  \structure{Compare result to initial AdaBoost algorithm:}

  \begin{itemize}
    \item \structure{Exponential loss:}
      \begin{eqnarray*}
        \beta_m     &=& \frac{1}{2} \log \frac{1 - \mathsf{err}_m}{\mathsf{err}_m} \\
        \alpha_m    &=& 2 \beta_m \\
        w_i^{(m+1)} &=& w_i^{(m)} ~ e^{\alpha_m I(y_i \neq G_m(\vec{x}_i))} ~ e^{-\beta_m}
      \end{eqnarray*}
      \pspread
    \item \structure{AdaBoost:}
      \begin{eqnarray*}
        \alpha_m &=&          \log \frac{1- \mathsf{err}_m}{\mathsf{err}_m} \\
        w_i      &\leftarrow& w_i \, e^{\alpha_m I(y_i \neq G_m(\vec{x}_i))}
      \end{eqnarray*}
  \end{itemize}
\end{frame}


\begin{frame}
  \frametitle{Exponential Loss \cont}

  \begin{center}
    \resizebox{.85\linewidth}{!}{
      \alt<4->{
        \input{\texfigdir/exploss4.pstex_t}
      }{\alt<3>{
        \input{\texfigdir/exploss3.pstex_t}
      }{\alt<2>{
        \input{\texfigdir/exploss2.pstex_t}
      }{
        \input{\texfigdir/exploss1.pstex_t}
      }}}
    }
  \end{center}
\end{frame}


\begin{frame}
  \frametitle{Exponential Loss \cont}

  \structure{Discussion:}

  \begin{itemize}
    \item The equivalence of AdaBoost to forward stagewise additive modeling was discovered five years after its invention. %\pause
    \item The AdaBoost criterion yields a monotone decreasing function of the \structure{margin}: $y\,f(\vec x)$. \\[.5cm] %\pause
    \item In $\{-1, 1\}$ classification, the margin plays a role similar to the \\
      residuals $y - f(\vec x)$ in regression:
      \begin{itemize}
        \item Observations with $y_i \, f(\vec{x}_i) > 0$ are classified correctly
        \item Observations with $y_i \, f(\vec{x}_i) < 0$ are misclassified 
        \item The decision boundary is at $f(\vec x) = 0$
     \end{itemize}
  \end{itemize}
\end{frame}


\begin{frame}
  \frametitle{Exponential Loss \cont}

  \structure{Discussion (cont.):}

  \begin{itemize}
    \item The goal of the classification algorithm is to produce positive margins as frequently as possible. %\pause
    \item Thus, any loss criterion should penalize negative margins more heavily than positive ones! %\pause
    \item The exponential criterion concentrates much more influence on observations with large negative margins.
  \end{itemize}
  \pspread

  Due to the exponential loss, AdaBoost performance is known to degrade rapidly:
  \begin{itemize}
    \item in situations of noisy data
    \item when there are wrong class labels in the training data
  \end{itemize}
\end{frame}

{
\usebackgroundtemplate{\includegraphics[width=\paperwidth]{png/nextTime.png}}
\begin{frame}[plain]
\end{frame}
}


%\subsection{Face Detection}

\begin{frame}
  \frametitle{Face Detection}

  Famous algorithm developed by \structure{Viola and Jones} in 2001.
  \spread

  \structure{Contributions:}

  \begin{itemize}
    \item Integral images for feature computation.
    \item Usage of AdaBoost for boosting.
    \item Classifier cascade for fast rejection of non-face regions.
  \end{itemize}
  \spread

  Various implementations available: for example look into \textit{OpenCV FaceDetector} sample code.
\end{frame}


\begin{frame}
  \frametitle{Face Detection \cont}

  \structure{Features:}

  \begin{itemize}
    \item Features are adapted from Haar basis functions. %\pause
    \item They are calculated by subtracting the sum of the pixels in the white from the sum of the pixels in the gray rectangles: \\[.3cm] %\pause
  \end{itemize}

  \structure{4 types of features used:}

  \begin{center}
    \resizebox{.4\linewidth}{!}{
      \input{\texfigdir/haar_features.pstex_t}
    }
  \end{center}
\end{frame}


\begin{frame}
  \frametitle{Face Detection \cont}

  \begin{itemize}
    \item The exhaustive set of features is very large: $> 45\,000$ for a 380x280 image. %\pause
    \item Rectangle features can be efficiently computed using an \\
     \structure{Integral Image $II$}:
      \begin{displaymath}
        II(x,y) = \sum_{x' \leq x, y' \leq y} I(x', y')
      \end{displaymath}
  \end{itemize}
  \spread

  \begin{center}
    \resizebox{.4\linewidth}{!}{
      \input{\texfigdir/integral_image.pstex_t}
    }
  \end{center}
\end{frame}


\begin{frame}
  \frametitle{Face Detection \cont}

  \structure{Computation of the sum of intensities for any rectangle D  with just \\ 3 basis operations based on the integral image:} \\[.5cm]

  \begin{center}
    \resizebox{.4\linewidth}{!}{
      \input{\texfigdir/integral_image2.pstex_t}
    }
  \end{center}

  \begin{itemize}
    \item Value at 1: A %\pause
    \item Value at 2: A + B %\pause
    \item Value at 3: A + C %\pause
    \item Value at 4: A + B + C + D %\pause
    \item Sum within D: $4 + 1 - (2 + 3)$
  \end{itemize}
\end{frame}


\begin{frame}
  \frametitle{Face Detection \cont}

  \structure{Boosting:}

  \begin{itemize}
    \item The classification functions are restricted to each depend on a single feature only. %\pause
    \item This way, AdaBoost can be interpreted as an effective feature selecting algorithm:
      \begin{itemize}
        \item The single rectangle feature is selected which best separates the observations. %\pause
        \item For each feature, the weak learner determines the optimal threshold classification function.
      \end{itemize}
  \end{itemize}
\end{frame}


\begin{frame}
  \frametitle{Face Detection \cont}

  \begin{itemize}
    \item A weak classifier $G_j(\vec x)$ consists of a \structure{feature $x_j$}, an \structure{optimal threshold $\theta_j$}, and a \structure{parity $s_j$} to indicate the direction of the inequality:
      \begin{displaymath}
        G_j(\vec x) = \begin{cases}
                        1 \quad \mbox{if} ~s_j f_j(\vec x) < s_j \theta_j \\
                        0 \quad \mbox{otherwise}
                      \end{cases}
      \end{displaymath}
      with $\vec x$ being a sub-window of an image.
      \pspread
    \item Very aggressive process to discard the vast majority of features.
  \end{itemize}
\end{frame}


\begin{frame}
  \frametitle{Face Detection \cont}

  \begin{itemize}
    \item Best feature selected by AdaBoost:
  \end{itemize}

  \begin{center}
    \begin{figure}
            \copyrightbox[b]{
      \makebox[.6\linewidth]{
          \includegraphics[width=0.4\linewidth]{\pngdir/bestfeature.\png}
      }
    }{Christian Hacker, Anton Batliner, and Elmar Noeth. \href{https://www5.informatik.uni-erlangen.de/Forschung/Publikationen/2006/Hacker06-AYL.pdf}{Are You Looking at Me, Are You Talking with Me: Multimodal Classification of the Focus of Attention.}}
  \end{figure}
  \end{center}
\end{frame}


\begin{frame}
  \frametitle{Face Detection \cont}

  \structure{Classifier Cascade:}

  \begin{itemize}
    \item Despite the efficiency of the feature computation, the complexity of the problem is still huge. %\pause
    \item Evaluating the full AdaBoost sequence on all sub-windows of the image takes too much time.
    \pspread
    \item Idea of cascaded classifiers:
    \begin{itemize}
      \item Simpler classifiers used to reject the majority of sub-windows. %\pause
      \item Each stage is again created by AdaBoost. %\pause
      \item Stage 1 uses the two features shown before; detects 100\,\% faces with \\ false positive rate (FPR) of around 40\,\%.
    \end{itemize}
  \end{itemize}
\end{frame}


\begin{frame}
  \frametitle{Face Detection \cont}

  \begin{itemize}
    \item The cascade has the overall shape of a degenerated decision tree: \\[.5cm]
      \begin{center}
        \resizebox{.7\linewidth}{!}{
          \input{\texfigdir/cascade.pstex_t}
        }
      \end{center}
      \pspread
    \item A negative classification at any stage yields an immediate rejection of the sub-window.
  \end{itemize}
\end{frame}


\begin{frame}
  \frametitle{Face Detection \cont}

  \begin{itemize}
    \item Structure of cascade reflects the (usually) overwhelming majority of negative sub-windows in an image. %\pause
    \item Goal is to reject as many negatives as possible at the earliest stage of the processing.
  \end{itemize}
  \spread

  \structure{Training of the cascade:}

  \begin{itemize}
    \item Subsequent classifiers are trained using only those examples which pass through all the previous stages. %\pause
    \item The next classifier faces a more difficult task than the previous one. %\pause
    \item Trade-off between more features achieving higher detection rates and lower false positive rates while requiring more computation time.
  \end{itemize}
\end{frame}


\begin{frame}
  \frametitle{Face Detection \cont}

  \structure{Final classifier:}

  \begin{itemize}
    \setlength\itemsep{0.5cm}
    \item 32 layer cascade of increasingly strong boosted classifiers
    
    \begin{center}
      \begin{tabular}{r@{}l r c c}
        \multicolumn{2}{c}{\structure{Layer}} & 
        \structure{\# Features} & 
        \structure{TPR} &
        \structure{FPR}\\
        \hline
           &1  &   2~~~~~ & 100\,\% & 40\,\% \\
           &2  &   5~~~~~ & 100\,\% & 20\,\% \\
         3-&5  &  20~~~~~ & \\
         6-&7  &  50~~~~~ & \\
         8-&12 & 100~~~~~ & \\
        13-&32 & 200~~~~~ & \\ %\pause
      \end{tabular}
    \end{center}

    \item Number of stages and features adapted until false positive rate on validation was nearly zero while maintaining high correct rate.
  \end{itemize}
\end{frame}


\begin{frame}
  \frametitle{Face Detection \cont}

  \structure{Example images from the training set:}

  \begin{center}
  \begin{figure}
            \copyrightbox[b]{
      \makebox[.6\linewidth]{
          \includegraphics[width=0.45\linewidth]{\pngdir/FaceTraining.\png}
      }
    }{Paul Viola and Michael Jones. \href{https://www.researchgate.net/publication/3940582_Rapid_Object_Detection_using_a_Boosted_Cascade_of_Simple_Features}{Rapid Object Detection using a Boosted Cascade of Simple Features. IEEE Conf Comput Vis Pattern Recognit.}}
  \end{figure}
  \end{center}
\end{frame}


\begin{frame}
  \frametitle{Face Detection \cont}

  \structure{Results on test set:}
  \begin{center}
  \begin{figure}
            \copyrightbox[b]{
      \makebox[.6\linewidth]{
          \includegraphics[width=0.5\linewidth]{\pngdir/facedetection.\png}
      }
    }{Christian Hacker, Anton Batliner, and Elmar Noeth. \href{https://www5.informatik.uni-erlangen.de/Forschung/Publikationen/2006/Hacker06-AYL.pdf}{Are You Looking at Me, Are You Talking with Me: Multimodal Classification of the Focus of Attention.}}
  \end{figure}
  \end{center}
\end{frame}


%\subsection{Lessons Learned}

\begin{frame}
  \frametitle{Lessons Learned}

  \begin{itemize}
    \item Boosting of weak classifiers can yield a powerful combined classifier \\[.3cm]
    \item AdaBoost algorithm \\[.3cm]
    \item Weights of the classifiers and on the data are adjusted \\[.3cm]
    \item AdaBoost minimizes an exponential loss function \\[.3cm]
    \item Viola \& Jones face detection algorithm
  \end{itemize}
\end{frame}


\subsection{Further Reading}

\begin{frame}
  \frametitle{Further Reading}

  Most of the content of this lecture has been taken from: \\[.25cm]

  \small
  \begin{itemize}
    \item T.\ Hastie, R.\ Tibshirani, J.\ Friedman: \\
      \structure{The Elements of Statistical Learning}, \\
      2nd Edition, Springer, 2009.\\%[.15cm]
    \item Y.\ Freund, R.\,E.\ Schapire: \\
      \structure{A decision-theoretic generalization of on-line learning and \\
	    an application to boosting},\\
      Journal of Computer and System Sciences, 55(1):119-139, 1997.\\%[.15cm]
    \item P.\,A.\ Viola, M.\,J.\ Jones: 
      \structure{Robust Real-Time Face Detection}, \\
      International Journal of Computer Vision 57(2): 137-154, 2004.\\%[.15cm]
    \item J.\ Matas and J.\ \v{S}ochman:
      \structure{AdaBoost}, \\
      Centre for Machine Perception, Technical University, Prague. \\%[.15cm]
      \structure{\point\url{http://cmp.felk.cvut.cz/~sochmj1/adaboost_talk.pdf}}
  \end{itemize}
\end{frame}
