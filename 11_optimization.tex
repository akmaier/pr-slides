\section{Optimization}

\subsection{Motivation}

\begin{frame}
  \frametitle{Motivation}

  \begin{itemize}
    \item Optimization is crucial for many solutions in pattern recognition, \\
      pattern analysis, machine learning, artificial intelligence, etc. \\[.25cm]
    \item Optimization has many faces: 
      \begin{itemize}
        \item discrete optimization,
        \item combinatorial optimization,
        \item genetic algorithms,
        \item gradient descent,
        \item unconstrained and constrained optimization,
        \item linear programming,
        \item convex optimization, etc. \\[.25cm]
      \end{itemize}
    \item There is no lecture on pattern recognition without a \\
      refresher course on optimization techniques. \\[.25cm]
    \item Each researcher has his own favorite optimization algorithm.
  \end{itemize}
\end{frame}


\subsection{Unconstrained Optimization}

\begin{frame}
  \frametitle{Convexity}

  \begin{citeblock}{Definition}

    A function $f: \real^d \rightarrow \real$ is \structure{\emph{convex}} if the domain $\mathsf{dom}(f)$ of $f$ is a convex set \\
    and if $\forall \vec x, \vec y \in \mathsf{dom}(f)$, and $\theta$ with $0 \leq \theta \leq 1$, we have

    \begin{displaymath}
      f(\theta \vec x + (1-\theta) \vec y) \leq \theta f(\vec x) + (1-\theta) f(\vec y)
    \end{displaymath}
    \pause

    A function $f: \real^d \rightarrow \real$ is \structure{\emph{concave}} if $-f$ is convex.

  \end{citeblock}
  \pause
  \vspace{0.5cm}

  \structure{Geometric interpretation:} \\[.25cm]

  The line segment between $(\vec x, f(\vec x))$ and $(\vec y, f(\vec y))$ lies above the graph of $f$.
\end{frame}


\begin{frame}
  \frametitle{Unconstrained Optimization}
 
  Let us assume in the following that we have to compute the minimum \\
  of a convex function

  \begin{displaymath}
    f: \real^d \rightarrow \real
  \end{displaymath}

  that is twice differentiable. \\[.25cm] \pause

  The unconstrained optimization problem is just the solution of the\\
  minimization problem

  \begin{displaymath}
    \vec x^* = \argmin_{\vec x} f(\vec x)
  \end{displaymath}

  where $\vec x^*$ denotes the optimal point.
\end{frame}


\begin{frame}
  \frametitle{Unconstrained Optimization \cont}
  
  For this particular family of functions, a necessary and sufficient condition \\
  for the  minumum are the zero-crossings of the function's gradient:
  
  \begin{displaymath}
    \nabla f(\vec x^*) = \vec 0.
  \end{displaymath}
\end{frame}


\begin{frame}
  \frametitle{Unconstrained Optimization \cont}

  Most methods follow an \structure{iterative scheme}:

  \begin{eqnarray*}
    \mbox{initialization} & & \vec x ^{(0)}\\
    \mbox{iteration step} & & \vec x^{(k+1)}= g(\vec x^{(k)}) 
  \end{eqnarray*}

  where $g: \real^d\rightarrow \real^d$ is the update function. \\[0.5cm] \pause

  The iterations \structure{terminate}, if 
  
  \begin{displaymath}
    \|\vec x^{(k+1)} - \vec x^{(k)}\| < \epsilon,
  \end{displaymath}
  
  i.\,e.\ no further significant change.
\end{frame}


\subsection{Descent Methods}

\begin{frame}

  \frametitle{Descent Methods}

  We now consider iteration schemes that produce a sequence of estimates \\
  according to the update function
%  
  \begin{displaymath}
    \vec x^{(k+1)} = g(\vec x^{(k)}) = \vec x^{(k)} + t^{(k)}\Delta \vec x^{(k)} .
  \end{displaymath}
%  
  where 
%  
  \begin{eqnarray*}
    \Delta \vec x^{(k)}\in \real ^d: & & \quad \mbox{is the \structure{search direction} in the $k$-th iteration}\\
    t^{(k)}\in \real:                & & \quad \mbox{denotes the \structure{step length} in the $k$-th iteration}
  \end{eqnarray*}
  \pause
%  
  and where we expect 
%  
  \begin{displaymath}
    f( \vec x^{(k+1)} ) < f(\vec x^{(k)}) ~, \quad \mbox{i.\,e.}~~ \nabla f(\vec x^{(k)})^T \Delta \vec x^{(k)} < 0
  \end{displaymath}
%  
  except ~~$ \vec x^{(k+1)}=\vec x^{(k)}=\vec x^*$.
\end{frame}


\begin{frame}
  \frametitle{Taylor Approximation}
 
  For many problems it is always good to know the \structure{second order Taylor approximation}:

  \begin{displaymath}
    f(\vec x + t\cdot \Delta\vec x) 
    \approx f(\vec x) + t\cdot \nabla f(\vec x)^T \Delta \vec x +
      \frac{1}{2} t^2\cdot\Delta \vec x^T \nabla^2 f(\vec x)\Delta \vec x
  \end{displaymath}
\end{frame}


\begin{frame}
  \frametitle{Descent Methods \cont}
  
  \begin{algorithmic}
    \STATE \structure{Input:} function $f$, initial estimate $\vec x^{(0)}$
    \STATE Initialize: $k:=0$
    \REPEAT
      \STATE Select (or compute) descent direction 
      \STATE Line search (1-D optimization): $$t^{(k)}= \argmin_{t \geq 0}f(\vec x^{(k)}+t\cdot \Delta\vec x^{(k)})$$
      \STATE Update:
        \begin{displaymath}
          \vec x^{(k+1)} = \vec x^{(k)} + t^{(k)}\Delta \vec x^{(k)} .
        \end{displaymath}
      \STATE $k:=k+1$
    \UNTIL{$\|\vec x^{(k)} - \vec x^{(k-1)}\| < \epsilon$}
    \STATE \structure{Output:} $\vec x^{(k)}$
  \end{algorithmic}
\end{frame}


\subsection{Backtracking Line Search}

\begin{frame}
  \frametitle{Line Search Methods}

  \begin{itemize}  
    \item Multivariate optimization in its described form requires a \\
      proper line search method.\\[.25cm]
    \item Exact line search along the straight line $\{\vec x + t\Delta \vec x ~|~ t \geq 0\}$ has to solve
      \begin{displaymath}
        t^* = \argmin_{t\geq 0} f(\vec  x + t\Delta \vec x)
      \end{displaymath}
      and is rarely used. \\[.25cm]
    \item An overview of methods can be found in numerical recipes.
  \end{itemize}
\end{frame}


\begin{frame}
  \frametitle{Line Search Methods \cont}
  
  \structure{Setting $t = 0.25$:}
  
  \begin{center}
    \resizebox{.7\linewidth}{!}{
        \input{\texfigdir/stepsize_example3.pstex_t}
    }
  \end{center}
\end{frame}


\begin{frame}
  \frametitle{Line Search Methods \cont}
  
  \structure{Setting $t=1.9$:}
  
  \begin{center}
    \resizebox{.7\linewidth}{!}{
      \alt<24->{
        \input{\texfigdir/stepsize_example1_24.pstex_t}
      }{\alt<23>{
        \input{\texfigdir/stepsize_example1_23.pstex_t}
      }{\alt<22>{
        \input{\texfigdir/stepsize_example1_22.pstex_t}
      }{\alt<21>{
        \input{\texfigdir/stepsize_example1_21.pstex_t}
      }{\alt<20>{
        \input{\texfigdir/stepsize_example1_20.pstex_t}
      }{\alt<19>{
        \input{\texfigdir/stepsize_example1_19.pstex_t}
      }{\alt<18>{
        \input{\texfigdir/stepsize_example1_18.pstex_t}
      }{\alt<17>{
        \input{\texfigdir/stepsize_example1_17.pstex_t}
      }{\alt<16>{
        \input{\texfigdir/stepsize_example1_16.pstex_t}
      }{\alt<15>{
        \input{\texfigdir/stepsize_example1_15.pstex_t}
      }{\alt<14>{
        \input{\texfigdir/stepsize_example1_14.pstex_t}
      }{\alt<13>{
        \input{\texfigdir/stepsize_example1_13.pstex_t}
      }{\alt<12>{
        \input{\texfigdir/stepsize_example1_12.pstex_t}
      }{\alt<11>{
        \input{\texfigdir/stepsize_example1_11.pstex_t}
      }{\alt<10>{
        \input{\texfigdir/stepsize_example1_10.pstex_t}
      }{\alt<9>{
        \input{\texfigdir/stepsize_example1_09.pstex_t}
      }{\alt<8>{
        \input{\texfigdir/stepsize_example1_08.pstex_t}
      }{\alt<7>{
        \input{\texfigdir/stepsize_example1_07.pstex_t}
      }{\alt<6>{
        \input{\texfigdir/stepsize_example1_06.pstex_t}
      }{\alt<5>{
        \input{\texfigdir/stepsize_example1_05.pstex_t}
      }{\alt<4>{
        \input{\texfigdir/stepsize_example1_04.pstex_t}
      }{\alt<3>{
        \input{\texfigdir/stepsize_example1_03.pstex_t}
      }{\alt<2>{
        \input{\texfigdir/stepsize_example1_02.pstex_t}
      }{
        \input{\texfigdir/stepsize_example1_01.pstex_t}
      }}}}}}}}}}}}}}}}}}}}}}}
    }
  \end{center}
\end{frame}


\begin{frame}
  \frametitle{Line Search Methods \cont}
  
  \structure{Setting $t^{(k+1)}= \frac{1}{2} t^{(k)}$ and starting with $t^{(0)} = 0.5$:}
  
  \begin{center}
    \resizebox{.7\linewidth}{!}{
        \input{\texfigdir/stepsize_example2.pstex_t}
    }
  \end{center}
\end{frame}


\begin{frame}
  \frametitle{Backtracking Line Search}

  \begin{center}
    \resizebox{.7\linewidth}{!}{
      \alt<6->{
        \input{\texfigdir/armijo6.pstex_t}
      }{\alt<5>{
        \input{\texfigdir/armijo5.pstex_t}
      }{\alt<4>{
        \input{\texfigdir/armijo4.pstex_t}
      }{\alt<3>{
        \input{\texfigdir/armijo3.pstex_t}
      }{\alt<2>{
        \input{\texfigdir/armijo2.pstex_t}
      }{
        \input{\texfigdir/armijo1.pstex_t}
      }}}}}
    }
  \end{center}
\end{frame}


\begin{frame}
  \frametitle{Backtracking Line Search \cont}

  The Armijo-Goldstein line search algorithm: \\[0.5cm]

  \begin{algorithmic}
    \STATE \structure{Input:} function $f$, search direction $\Delta\vec x$
    \STATE Initialize: $t:=1$
    \STATE Select: $\alpha\in [0,0.5]$ and $\beta\in [0,1]$.
    \WHILE{$f(\vec x + t\Delta \vec x) > f(\vec x) +  \alpha t \cdot \nabla f(\vec x)^T\Delta\vec x$}
      \STATE $t := \beta t$
    \ENDWHILE
    \STATE \structure{Output:} $t$
  \end{algorithmic}
\end{frame}


\subsection{Gradient Descent Methods}

\begin{frame}
  \frametitle{Gradient Descent Methods}

  A natural choice of the search direction is the \structure{negative gradient}:
  
  \begin{displaymath}
    \Delta \vec x^{(k)} = -\nabla f(\vec x^{(k)})
  \end{displaymath}
  \vspace{1cm}

  \structure{Rule of thumb:}\\[.15cm]
  
  The negative gradient is the steepest descent direction.
\end{frame}


\begin{frame}
  \frametitle{Gradient Descent Methods \cont}

  \begin{algorithmic}
    \STATE \structure{Input:} function $f$, initial estimate $\vec x^{(0)}$
    \STATE intialize: $k:=0$
    \REPEAT
      \STATE Set descent direction: $\Delta\vec x^{(k)}=- \nabla f(\vec x^{(k)}) $
      \STATE Line search (1-D optimization): $$t^{(k)}= \argmin_{t \geq 0}f(\vec x^{(k)}+t\cdot \Delta\vec x^{(k)})$$
      \STATE Update:
        \begin{displaymath}
          \vec x^{(k+1)} = \vec x^{(k)} + t^{(k)}\Delta \vec x^{(k)} .
        \end{displaymath}
      \STATE $k:= k+1$
    \UNTIL{$\|\vec x^{(k)} - \vec x^{(k-1)}\|_2 < \epsilon$}
    \STATE \structure{Output:} $\vec x^{(k)}$
  \end{algorithmic}
\end{frame}

{
\usebackgroundtemplate{\includegraphics[width=\paperwidth]{png/nextTime.png}}
\begin{frame}[plain]
\end{frame}
}


\subsection{Steepest Descent Methods}

\subsubsection{Algorithm}

\begin{frame}
  \frametitle{Steepest Descent Methods}

  \structure{(Normalized) steepest descent, what does it mean?} \\[.25cm]

  We search for the unit vector that shows the largest decrease \\
  in the linear approximation of $f$:
  
  \begin{displaymath}
    \Delta \vec x = \argmin_{\vec u} \{\nabla f(\vec x)^T \vec u ; \ \| \vec u\|_p = 1\}
  \end{displaymath}   
  \pause         
   
  \structure{Conclusions:} 

  \begin{itemize}
    \item The steepest descent direction depends on the chosen norm.
    \item The negative gradient is not necessarily the best choice for the \\
      search direction.
  \end{itemize}
\end{frame}


\begin{frame}
  \frametitle{Steepest Descent Methods \cont}

  We consider now the first order Taylor approximation of $f(\vec x +\vec u)$ \\
  around the selected position $\vec x$:
  
  \begin{displaymath}
    f(\vec x+\vec u) \approx f(\vec x) + \nabla f(\vec x)^T \vec u.
  \end{displaymath}
  \pause
  
  \begin{itemize}
    \item Here $ \nabla f(\vec x)^T \vec u$ is the directional derivative at $\vec x$ in direction $\vec u$.
    \item The vector $\vec u$ denotes a descent direction if the inner product \\
      with the gradient vetor is negative, i.\,e.
      \begin{displaymath}
        \nabla f(\vec x)^T \vec u < 0~.
      \end{displaymath}
  \end{itemize}
\end{frame}


\begin{frame}
  \frametitle{Steepest Descent Methods \cont}

  \vspace*{-.25cm}
  \begin{algorithmic}
    \STATE \structure{Input:} function $f$, initial estimate $\vec x^{(0)}$, norm $\|.\|$
    \STATE intialize: $k:=0$
    \REPEAT
      \STATE Compute highest descent direction: $$ \Delta \vec x ^{(k)}= \argmin_{\vec u} \{\nabla f(\vec x^{(k)})^T \vec u ; \ \| \vec u\| = 1\}$$    
      \STATE Line search (1-D optimization): $$t^{(k)}= \argmin_{t \geq 0}f(\vec x^{(k)}+t\cdot \Delta\vec x^{(k)})$$
      \STATE Update:
        \begin{displaymath}
          \vec x^{(k+1)} = \vec x^{(k)} + t^{(k)}\Delta \vec x^{(k)} .
        \end{displaymath}
      \STATE $k:=k+1$
    \UNTIL{$\|\vec x^{(k)} - \vec x^{(k-1)}\| < \epsilon$}
    \STATE \structure{Output:} $\vec x^{(k)}$
  \end{algorithmic}
\end{frame}


\subsubsection{$L_2$-norm}

\begin{frame}
  \frametitle{$L_2$-Norm}

  The unit ball for the $L_2$-norm:

  \begin{figure}
    \resizebox{.3\linewidth}{!}{
      \alt<4->{
        \input{\texfigdir/unit_ball_gradient_L2_3.pstex_t}
      }{\alt<3>{
        \input{\texfigdir/unit_ball_gradient_L2_2.pstex_t}
      }{\alt<2>{
        \input{\texfigdir/unit_ball_gradient_L2_1.pstex_t}
      }{
        \input{\texfigdir/unit_ball_gradient_L2_0.pstex_t}
      }}}
    }
  \end{figure}
 
  For the $L_2$-norm the steepest descent direction is the negative gradient:
  
  \begin{displaymath}
    \Delta \vec x 
    = \argmin_{\vec u} \{\nabla f(\vec x)^T \vec u ; \ \| \vec u\|_2 = 1\}
    = -\nabla f(\vec x)
  \end{displaymath}
\end{frame}


\subsubsection{$L_1$-norm}

\begin{frame}
 \frametitle{$L_1$-Norm}
 
  The unit ball for the $L_1$-norm: \\[.25cm]
  
  \begin{columns}
  
    \column{.5\linewidth}
      \centering
      \resizebox{.9\linewidth}{!}{
        \alt<4->{
          \input{\texfigdir/unit_ball_gradient_L1_3.pstex_t}
        }{\alt<3>{
          \input{\texfigdir/unit_ball_gradient_L1_2.pstex_t}
        }{\alt<2>{
          \input{\texfigdir/unit_ball_gradient_L1_1.pstex_t}
        }{
          \input{\texfigdir/unit_ball_gradient_L1_0.pstex_t}
        }}}
      }
    
      \column{.5\linewidth}
        \centering
        \onslide<5->
        \resizebox{.9\linewidth}{!}{
          \input{\texfigdir/unit_ball_gradient_L1_4.pstex_t}
        }
  \end{columns}
\end{frame}


\begin{frame}
  \frametitle{$L_1$-Norm \cont}

  \begin{itemize}
    \item The steepest descent for the $L_1$-norm selects in each iteration \\
      the component of $\nabla f(\vec x)$ with maximum absolute value and then \\
      decreases or increases dependent on the sign of the selected component. \pause
    \item Let $i$ be the index of the gradient component with maximum absolute \\
      value, and let $\vec e_i\in\real^d$ denote the corresponding base vector.\\
      The steepest descent direction is given by:
      \begin{eqnarray*}
        \Delta \vec x &=& \argmin_{\vec u} \{\nabla f(\vec x)^T \vec u ; \ \| \vec u\|_1 = 1\} \\
                      &=& -\mbox{sgn}\left(\frac{\partial}{\partial x_i} f(\vec x)\right)\vec e_i
      \end{eqnarray*}
      \pause
    \item \structure{Note:} Steepest descent using the $L_1$-norm results in the {\em coordinate descent algorithm}.
  \end{itemize}
\end{frame}


\subsubsection{$L_\infty$-norm}

\begin{frame}
  \frametitle{$L_\infty$-Norm}

  The unit ball for the $L_\infty$-norm: \\[.25cm]

  \begin{columns}
  
    \column{.5\linewidth}
      \centering
      \resizebox{.9\linewidth}{!}{
        \alt<2->{
          \input{\texfigdir/unit_ball_gradient_Linfinity_2.pstex_t}
        }{
          \input{\texfigdir/unit_ball_gradient_Linfinity_1.pstex_t}
        }
      }
    
      \column{.5\linewidth}
        \centering
        \onslide<3->
        \resizebox{.9\linewidth}{!}{
          \input{\texfigdir/unit_ball_gradient_Linfinity_3.pstex_t}
        }
  \end{columns}
\end{frame}


\subsubsection{Quadratic $L_{\mat P}$-norm}

\begin{frame}
  \frametitle{$L_{\mat P}$-Norm}
  
  The unit ball for the $L_{\mat P}$-norm:

  \begin{figure}
    \resizebox{.45\linewidth}{!}{
      \alt<3>{
        \input{\texfigdir/unit_ball_gradient_LP_3.pstex_t}
      }{\alt<2>{
        \input{\texfigdir/unit_ball_gradient_LP_2.pstex_t}
      }{
        \input{\texfigdir/unit_ball_gradient_LP_1.pstex_t}
      }}
    }
  \end{figure}
\end{frame}


\begin{frame}
  \frametitle{$L_{\mat P}$-Norm \cont}
  
  The \structure{steepest descent} for the $L_{\mat P}$-norm is given by:
%  
  \begin{eqnarray*}
    \Delta \vec x 
      &=& \argmin_{\vec u} \{\nabla f(\vec x)^T \vec u ; \ \| \vec u\|_{\mat P} = 1\} \\ \pause
      &=& \argmin_{\vec u} \{\nabla f(\vec x)^T \vec u ;\  (\vec u^T{\mat P}\vec u)^{\frac{1}{2}} = 1\} \\ \pause
      &=& \argmin_{\vec u} \{\nabla f(\vec x)^T \vec u ;\ \| {\mat P}^{\frac{1}{2}} \vec u\|_2 = 1\}
  \end{eqnarray*}
  \pause
%  
  As we did in the LDA-transform, we introduce a transform to get \\
  \structure{spherical data}:
%
  \begin{displaymath}
    \vec u' = {\mat P}^{\frac{1}{2}} \vec u
  \end{displaymath}
  \pause
%
  and thus
%  
  \begin{displaymath}
    f(\vec u) = f({\mat P}^{-\frac{1}{2}}\vec u') = f'(\vec u')
  \end{displaymath}
\end{frame}


\begin{frame}
  \frametitle{$L_{\mat P}$-Norm \cont}
  
  Instead of $f(\vec x)$ we now minimize $f'(\vec x')$ using the $L_2$-norm and \\
  back-transform the result:
  
  \begin{eqnarray*}
    \Delta \vec x' 
      &=& \argmin_{\vec u} \{\nabla f'(\vec x')^T \vec u' ; \ \| \vec u'\|_{2} = 1\} \\[.15cm] \pause
      &=& -\nabla f'(\vec x') \\[.15cm] \pause
      &=& -\mat P^{-\frac{1}{2}}\nabla f({\mat P}^{-\frac{1}{2}}\vec x') \\[.15cm] \pause
      &=& - {\mat P}^{-\frac{1}{2}}\nabla f(\vec x)
  \end{eqnarray*}
\end{frame}


\begin{frame}
  \frametitle{$L_{\mat P}$-Norm \cont}
 
  Now we get for $\Delta \vec x$:

  \begin{eqnarray*}
    \Delta \vec x 
      &=& {\mat P}^{-\frac{1}{2}} \Delta \vec x' \\[.15cm] \pause
      &=& {\mat P}^{-\frac{1}{2}} \left( - {\mat P}^{-\frac{1}{2}}\nabla f(\vec x) \right) \\[.15cm] \pause
      &=& - \mat{P}^{-1} \nabla f(\vec x).
  \end{eqnarray*}
  \pause
   
  \structure{Conclusion:} The steepest descent for the $L_{\mat P}$-norm is given by
  
  \begin{displaymath}
     \Delta \vec x = -\mat{P}^{-1} \nabla f(\vec x)~.
  \end{displaymath}
\end{frame}


\subsection{Newton's Method}

\begin{frame}
  \frametitle{Newton's Method}

  \structure{The idea:}
  
  \begin{itemize}
    \item Select a point.
    \item Compute the minimum of the second order Taylor approximation.
  \end{itemize}
  
  \begin{center}
    \resizebox{.7\linewidth}{!}{
      \alt<8->{
        \input{\texfigdir/newton-raphson8.pstex_t}
      }{\alt<7>{
        \input{\texfigdir/newton-raphson7.pstex_t}
      }{\alt<6>{
        \input{\texfigdir/newton-raphson6.pstex_t}
      }{\alt<5>{
        \input{\texfigdir/newton-raphson5.pstex_t}
      }{\alt<4>{
        \input{\texfigdir/newton-raphson4.pstex_t}
      }{\alt<3>{
        \input{\texfigdir/newton-raphson3.pstex_t}
      }{\alt<2>{
        \input{\texfigdir/newton-raphson2.pstex_t}
      }{
        \input{\texfigdir/newton-raphson1.pstex_t}
      }}}}}}}
    }
  \end{center}  
\end{frame}


\begin{frame}
  \frametitle{Newton's Method \cont}
 
  Second order Taylor approximation:
  \begin{displaymath}
    f(\vec x+\Delta \vec x) 
    \approx f(\vec x) + \nabla f(\vec x)^T \Delta \vec x + \frac{1}{2} \Delta \vec x^T (\nabla^2 f(\vec x))\Delta \vec x 
  \end{displaymath}
  \pause

  Now we select $\Delta \vec x$ such that 
  \begin{displaymath}
    \nabla\{  f(\vec x) + \nabla f(\vec x)^T \Delta \vec x + \frac{1}{2}\Delta \vec x^T (\nabla^2 f(\vec x))\Delta \vec x\} 
    = 0
  \end{displaymath}
  \pause

  Obviously the gradient is
  \begin{displaymath}
    \nabla f(\vec x)+ \nabla^2 f(\vec x)\Delta \vec x 
    = 0
  \end{displaymath}
  \pause

  and thus
  \begin{displaymath}
    \Delta \vec x 
    = -(\nabla^2 f(\vec x))^{-1} \nabla f(\vec x)
  \end{displaymath} 
\end{frame}


\begin{frame}
  \frametitle{Newton's Method \cont}

  \structure{Conclusion:} \\[.25cm]
 
  Newton's method is an $\vec x$-dependent steepest descent method regarding \\
  the $L_{\mat P}$-norm, where $\mat P = \nabla^2 f(\vec x)$ is the Hessian.
\end{frame}


\begin{frame}
  \frametitle{Damped Newton's Method}
  
  \begin{algorithmic}
    \STATE \structure{Input:} function $f$, initial estimate $\vec x^{(0)}$
    \STATE intialize: $k:=0$
    \REPEAT
      \STATE Compute Newton step:     $$ \Delta \vec x^{(k)}= -\nabla^2 f(\vec x^{(k)})^{-1} \nabla f(\vec x^{(k)}) $$
      \STATE Line search (1-D optimization): $$t^{(k)}= \argmin_{t \geq 0}f(\vec x^{(k)}+t\cdot \Delta\vec x^{(k)})$$
      \STATE Update:
        \begin{displaymath}
          \vec x^{(k+1)} = \vec x^{(k)} + t^{(k)}\Delta \vec x^{(k)} .
        \end{displaymath}
    \STATE $k:=k+1$
   \UNTIL{$\|\vec x^{(k)} - \vec x^{(k-1)}\| < \epsilon$}
   \STATE \structure{Output:} $\vec x^{(k)}$
 \end{algorithmic}
\end{frame}


\subsection{Lessons Learned}

\begin{frame}
  \frametitle{Lessons Learned}

  \begin{itemize}
    \item Gradient descent is widely applied. \\[.25cm]
    \item Gradient descent and coordinate descent are special cases of steepest descent methods. \\[.25cm]
    \item Steepest descent method depends on the chosen norm.
  \end{itemize}
\end{frame}

{
\usebackgroundtemplate{\includegraphics[width=\paperwidth]{png/nextTime.png}}
\begin{frame}[plain]
\end{frame}
}


\subsection{Further Readings}

\begin{frame}
  \frametitle{Further Readings}

  This chapter is basically copied from: \\[.25cm]
 
  \begin{itemize}
    \item S.~Boyd, L.~Vandenberghe: \\
      \structure{Convex Optimization}, \\
      Cambridge University Press, 2004. \\
      \point{\small \url{http://www.stanford.edu/~boyd/cvxbook/}} \\[.25cm]
    \item Jorge Nocedal, Stephen Wright: \\
      \structure{Numerical Optimization}, \\
      Springer, New York, 1999.
  \end{itemize}
\end{frame}


\subsection{Comprehensive Questions}

\begin{frame}
  \frametitle{Comprehensive Questions}

  \begin{itemize}
    \item What is the general formulation for an unconstrained optimization problem? \\[.5cm]
    \item Why do we need a line search in gradient descent approaches? \\[.5cm]
    \item What is the Armijo-Goldstein line search algorithm? \\[.5cm]
    \item What are the steepest descent directions if we apply the $L_\infty$, $L_1$, $L_2$, and $L_{\mat P}$ norm?
  \end{itemize}
\end{frame}
