\section{Discriminant Analysis II}

\subsection{Rank Reduced Linear Discriminant Analysis}

\begin{frame}
	\frametitle{Rank Reduced Linear Discriminant Analysis}

	\begin{center}
		\begin{minipage}{0.8\textwidth}
			\begin{itemize}
				\item[{\bf Problem}:] How to choose an $L$-dimensional subspace with $L=K-1$ that is good for  LDA? \\[.5cm]
				\item[{\bf Idea}:] Maximize the spread of the $L$-dimensional projection of centroids. \\[.5cm]
				\item[{\bf Solution}:] Principal component analysis, i.\,e.\ \\
				      we compute the principal components of the \\
				      covariance matrix of the mean vectors
				      $$\vec{\mu}'_y = \phi(\vec{\mu}_y) \in \real^{K-1},$$
				      where $y=1,2,\dots, K$.
			\end{itemize}
		\end{minipage}
	\end{center}
\end{frame}


\begin{frame}
	\frametitle{Rank Reduced Linear Discriminant Analysis \cont}

	In \structure{Principle Component Analysis (PCA)} we compute a linear mapping $\mat\Phi \in \real^{L \times (K-1)}$ that
	results in the highest spread of projected features:
	\pause

	\begin{displaymath}
		\mat{\Phi}^* = \argmax_{\mat{\Phi}} \left(
		\frac{1}{K}
		\sum_{y=1}^K (\mat{\Phi}\vec {\mu}'_y - \mat {\Phi}\bar{\mat \mu}' )^T
		(\mat{\Phi}\vec {\mu}'_y - \mat {\Phi}\bar{\mat \mu}' ) +
		\sum_{i=1}^{L} \lambda_i (\| \mat{\Phi}_i \|_2^2 - 1)
		\right)
	\end{displaymath}
	% 
	where we applied the \structure{Lagrange multiplier} method to allow for the maximization of the spread subject to
	%
	\begin{displaymath}
		\| \mat {\Phi}_i \|_2^2 = 1 ~, \quad i = 1, \ldots, K-1 ~.
	\end{displaymath}
	%
	Here $\| \mat{\Phi}_i \|_2^2$ denotes the $L_2$ norm of the $i$-th row vector of $\mat{\Phi}$.
\end{frame}


\begin{frame}
	\frametitle{Excursus: Optimization with Constraints}

	\structure{Lagrange Multipliers:} simple example

	\begin{columns}
		\column{.5\linewidth}

		\begin{itemize}
			\item Find the maximum of \\
			      $f(x,y) = x +y$ \\[.25cm]
			      \onslide<2->{
			\item constraint: $x^2 + y^2 = 1$ \\[.25cm]
			      }
			      \onslide<4->{
			\item Lagrange function: \\
			      $L(x,y,\lambda) = x + y + \lambda(x^2 + y^2 - 1)$ \\[0.25cm]
			      }
			      \onslide<5->{
			\item Set the partial derivatives to zero: \\
			      $\frac{\partial L}{\partial x} =
				      \frac{\partial L}{\partial y} =
				      \frac{\partial L}{\partial \lambda}
				      \stackrel{!}{=} 0$
			      }
		\end{itemize}

		\column{.5\linewidth}

		\begin{center}
			\resizebox{\linewidth}{!}{
				\alt<3->{
					\input{\texfigdir/lagrange_multiplier3.pstex_t}
				}{\alt<2>{
						\input{\texfigdir/lagrange_multiplier2.pstex_t}
					}{
						\input{\texfigdir/lagrange_multiplier1.pstex_t}
					}}
			}
		\end{center}
	\end{columns}
\end{frame}


\begin{frame}
	\frametitle{Rank Reduced Linear Discriminant Analysis \cont}

	We need some facts from matrix calculus: \\
	\point \structure{\url{www.matrixcookbook.com}}%
	\footnote{website currently off-line -- you can still download it \point\href{http://www2.imm.dtu.dk/pubdb/views/publication_details.php?id=3274}{\structure{here}}} \\
	%  
	\spread

	\begin{enumerate}
		\item Let $\vec\mu$ denote the mean and $\mat \Sigma$ the covariance matrix of a \\
		      random vector $\vec x$, then we get:
		      \begin{displaymath}
			      E[(\mat A \vec x)^T(\mat A \vec x)] =
			      {\mbox{tr}}(\mat A \mat \Sigma \mat A^T) + (\mat A \vec \mu)^T(\mat A\vec \mu)
		      \end{displaymath}
		      \pause
		\item The matrix derivative is:
		      \begin{displaymath}
			      \frac{\partial \mbox{tr}( \mat X \mat B \mat X^T)}{\partial \mat X} = \mat X\mat B^T +\mat X\mat B
		      \end{displaymath}
	\end{enumerate}
\end{frame}


\begin{frame}
	\frametitle{Rank Reduced Linear Discriminant Analysis \cont}

	For our \structure{optimization problem} this implies:

	\begin{eqnarray*}
		& & \frac{\partial}{\partial{\mat{\Phi}}}
		\left\{
		\frac{1}{K} \sum_{y=1}^K (\mat{\Phi} \vec{\mu}'_y - \mat{\Phi} \bar{\vec{\mu}}' )^T
		(\mat{\Phi} \vec{\mu}'_y - \mat{\Phi} \bar{\vec{\mu}}' )  +
		\sum_{i=1}^{L} \lambda_i (\| \mat{\Phi}_i \|^2_2 - 1)
		\right\} \\[.5cm] \pause
		& & = \quad \frac{\partial}{\partial{\mat{\Phi}}}
		\left\{
		\frac{1}{K} \sum_{y=1}^K \left( \mat\Phi (\vec{\mu}'_y - \bar{\vec{\mu}}') \right)^T
		\left( \mat\Phi (\vec{\mu}'_y - \bar{\vec{\mu}}') \right) +
		\sum_{i=1}^{L} \lambda_i (\| \mat {\Phi}_i \|^2_2 - 1)
		\right\} \\[.5cm] \pause
		& & = \quad \frac{\partial}{\partial{\mat{\Phi}}}
		\left\{
		\mbox{tr} (\mat\Phi \ \mat \Sigma_{\mbox{inter}} \ \mat\Phi^T) +
		\sum_{i=1}^{L} \lambda_i (\| \mat {\Phi}_i \|_2^2 - 1)
		\right\}
		\stackrel{!}{=} \mat 0.
	\end{eqnarray*}
\end{frame}


\begin{frame}
	\frametitle{Rank Reduced Linear Discriminant Analysis \cont}

	Now we compute the \structure{partial derivatives}:

	\begin{displaymath}
		\frac{\partial}{\partial{\mat{\Phi}}}
		\left\{
		\mbox{tr} (\mat \Phi \ \mat \Sigma_{\mbox{inter}}  \ \mat \Phi^T) + \sum_{i=1}^{L} \lambda_i (\| \mat {\Phi}_i \|^2_2 - 1)  \right\} =
		2 \mat \Phi\mat  \Sigma_{\mbox{inter}} +2 \mat{\lambda} \mat \Phi = \mat 0
	\end{displaymath}
	\pspread

	This results in the eigenvalue and eigenvector problem:
	\begin{displaymath}
		\mat{\Sigma}_{\mbox{inter}} \mat \Phi^T = \mat{\lambda}' \mat \Phi^T
	\end{displaymath}
	\pspread

	\structure{Note:} \\
	In original PCA, the transform $\mat \Phi$ maximizes the overall spread using the covariance matrix of all features:

	\begin{displaymath}
		\mat{\Sigma} \mat{\Phi}^T = \mat{\lambda}' \mat \Phi^T
	\end{displaymath}
\end{frame}


\begin{frame}
	\frametitle{Rank Reduced Linear Discriminant Analysis \cont}

	\begin{algorithmic}
		\STATE  \structure{Input:} training data:  $S = \{ (\vec x_1, y_1), (\vec x_2, y_2), (\vec x_3, y_3), \dots, (\vec x_m, y_m) \}$ \\[.3cm]
		\STATE 1. Compute the covariance matrix of transformed mean vectors
		$$\widehat{\mat{\Sigma}}_{\mbox{inter}} =
			\frac{1}{K} \sum_{y=1}^K (\vec \mu'_y - \bar{\vec \mu}')(\vec \mu'_y - \bar{\vec \mu}')^T,$$
		\quad\,where $\bar{\vec \mu}' = \frac{1}{K} \cdot \sum_{y=1}^{K}\vec\mu'_y$. \\[.3cm]
		\pause
		\STATE 2. Compute the $L$ eigenvectors of the covariance matrix belonging \\
		\quad\,to the largest eigenvalues. \\[.3cm]
		\pause
		\STATE 3. The eigenvectors are the rows of the mapping $\mat\Phi$ from the \\
		\quad\,$(K-1)$- to the $L$-dimensional feature space. \\[.3cm]
		\STATE  \structure{Output:} matrix $\mat\Phi$
	\end{algorithmic}
\end{frame}

{
\usebackgroundtemplate{\includegraphics[width=\paperwidth]{png/nextTime.png}}
\begin{frame}[plain]
\end{frame}
}


\subsection{Fisher Transform}

\begin{frame}
	\frametitle{Fisher Transform}

	The described method to compute the LDA mapping is not the original derivation.\\[.5cm]

	\structure{Original method}

	\begin{center}
		\resizebox{0.6\linewidth}{!}{
			\input{\texfigdir/lda.pstex_t}
		}
	\end{center}
\end{frame}

\begin{frame}
	\frametitle{Fisher Transform}

	The described method to compute the LDA mapping is not the original derivation.\\[.5cm]

	\structure{Original method}

	\begin{columns}
		\column{.65\linewidth}

		\small \vspace{-.5cm}
		\begin{itemize}
			\item Project samples $\vec{x}_i$ onto a straight line with direction $\vec{r}$, $\lVert \vec{r} \rVert_2 = 1$:
			      \begin{displaymath}
				      \tilde{x}_i = \vec{x}_i^T \vec{r}
			      \end{displaymath}
			      \onslide<2->{
			      \vspace{-.5cm}
			\item Maximize the ratio of the between-class scatter and the within-class scatter:
			      \begin{displaymath}
				      \vec{r}^* =
				      \argmax_{\vec{r}} J(\vec{r}) =
				      \argmax_{\vec{r}} \frac{|\tilde{\mu}_1 - \tilde{\mu}_2|^2}{\tilde{s}_1^2 + \tilde{s}_2^2}
			      \end{displaymath}
			      }
			      \onslide<3->{
			      \vspace{-.5cm}
			\item Classify by applying a threshold to $\tilde{x}_i$
			      }
		\end{itemize}

		\column{.35\linewidth}

		\onslide<1->{
			\resizebox{\linewidth}{!}{
				\input{\texfigdir/lda.pstex_t}
			}
		}
	\end{columns}
\end{frame}


\begin{frame}
	\frametitle{Fisher Transform \cont}

	\structure{Finding $\vec{r}^*$}

	\begin{enumerate}
		\item Mean and scatter matrix for each class:
		      {\small
		      \begin{eqnarray*}
			      \vec{\mu}_k &=& \frac{1}{m_k} \sum_{\substack{~i = 1\\ y_i = k}}^{m_k} \vec{x}_i \\
			      \mat{S}_k   &=& \sum_{\substack{~i = 1\\ y_i = k}}^{m_k} (\vec{x}_i - \vec{\mu}_k) (\vec{x}_i - \vec{\mu}_k)^T
		      \end{eqnarray*}
		      }
		\item \structure{Within-class scatter} matrix:
		      {\small
		      \begin{displaymath}
			      \structure{\mat{S}_\mathsf{W}} = \mat{S}_1 + \mat{S}_2
		      \end{displaymath}
		      }
		\item \structure{Between-class scatter} matrix:
		      {\small
		      \begin{displaymath}
			      \structure{\mat{S}_\mathsf{B}} = (\vec{\mu}_1 - \vec{\mu}_2) (\vec{\mu}_1 - \vec{\mu}_2)^T
		      \end{displaymath}
		      }
	\end{enumerate}
\end{frame}


\begin{frame}
	\frametitle{Fisher Transform \cont}

	\structure{Finding $\vec{r}^*$}

	\begin{enumerate}
		\setcounter{enumi}{3}
		\item Expressing $\tilde{\mu}_k$ and $\tilde{s}_k^2$ of the projected samples in terms of $\vec{\mu}_k$ and $\mat{S}_k$:
		      {\small
		      \begin{eqnarray*}
			      \onslide<1->{
				      \tilde{\mu}_k &=&
				      \onslide<1->{ \frac{1}{m_k} \sum_{\substack{~i = 1\\ y_i = k}}^{m_k} \tilde{x}_i }
				      \onslide<2->{ = \frac{1}{m_k} \sum_{\substack{~i = 1\\ y_i = k}}^{m_k} \vec{r}^T  \vec{x}_i }
				      \onslide<3->{ = \vec{r}^T \vec{\mu}_k } \\
			      }
			      \onslide<1->{
				      \tilde{s}_k^2 &=&
				      \onslide<1->{ \sum_{\substack{~i = 1\\ y_i = k}}^{m_k} (\tilde{x}_i - \tilde{\mu}_k)^2 }
				      \onslide<4->{ = \sum_{\substack{~i = 1\\ y_i = k}}^{m_k} (\vec{r}^T \vec{x}_i - \vec{r}^T \vec{\mu}_k)^2 }
				      \onslide<5->{ = \vec{r}^T \mat{S}_k \vec{r} }
			      }
		      \end{eqnarray*}
		      }
		      \onslide<6->{
		\item Plug it into $J(\vec{r})$:
		      {\small
		      \begin{displaymath}
			      J(\vec{r})
			      = \frac{|\tilde{\mu}_1 - \tilde{\mu}_2|^2}{\tilde{s}_1^2 + \tilde{s}_2^2}
			      \onslide<7->{ = \frac{\vec{r}^T \mat{S}_B \vec{r}}{\vec{r}^T \mat{S}_W \vec{r}}}
		      \end{displaymath}
		      }
		      }
		      \onslide<7->{
		\item[] This is known as the \structure{Generalized Rayleigh Quotient}.
		      }
	\end{enumerate}
\end{frame}


\begin{frame}
	\frametitle{Fisher Transform \cont}

	\structure{Finding $\vec{r}^*$}

	\begin{enumerate}
		\setcounter{enumi}{5}
		\item Maximizing the Generalized Rayleigh Quotient is equivalent to solving the following \structure{generalized eigenvalue problem}:
		      {\small
		      \begin{eqnarray*}
			      \mat{S}_\mathsf{B} \vec{r}^* &=& \lambda \mat{S}_\mathsf{W} \vec{r}^* \\
			      \mat{S}_\mathsf{W}^{-1} \mat{S}_\mathsf{B} \vec{r}^* &=& \lambda \vec{r}^* \\
		      \end{eqnarray*}
		      }
		\item \structure{Note}: $\mat{S}_\mathsf{B} \vec{r}^*$ is always in the direction of $\vec{\mu}_1 - \vec{\mu}_2$; \\
		      no need to compute the eigenvalues and eigenvectors of $\mat{S}_\mathsf{W}^{-1} \mat{S}_\mathsf{B}$! \\[.25cm]
		      The direction of $\vec{r}^*$ is:
		      {\small
		      \begin{displaymath}
			      \vec{r}^* = \mat{S}_\mathsf{W}^{-1} (\vec{\mu}_1 - \vec{\mu}_2)
		      \end{displaymath}
		      }
	\end{enumerate}
\end{frame}


\begin{frame}
	\frametitle{Fisher Transform \cont}

	\begin{itemize}
		\item Usually the total linear mapping for LDA is computed dimension by dimension through the maximization of the
		      Rayleigh ratio for each projection axis $\vec a^*$:
		      \begin{eqnarray*}
			      \vec a^* &=& \argmax_{\vec a}\ \frac{\vec a^T \mat \Sigma_{\mbox{inter}} \vec a}{\vec a^T \mat \Sigma_{\mbox{intra}} \vec a}
		      \end{eqnarray*}
		      \pause
		\item The solution is a generalized eigenvalue problem: $\vec a$ is the eigenvector of
		      $$\mat \Sigma_{\mbox{intra}} ^{-1} \mat \Sigma_{\mbox{inter}}$$
		      that belongs to the largest eigenvalue.
	\end{itemize}
\end{frame}


\begin{frame}
	\frametitle{Fisher Transform \cont}

	In literature the optimization problem is mostly rewritten:
	\spread

	\begin{itemize}
		\item Equivalent constrained optimization problem
		      \begin{eqnarray*}
			      \mbox{maximize:}   & & \vec r^T \mat \Sigma_{\mbox{inter}}\vec r\\
			      \mbox{subject to:} & & \vec r^T \mat \Sigma_{\mbox{intra}} \vec r =1
		      \end{eqnarray*}
		      \spread
		\item Lagrange multiplier method ($\lambda>0$):
		      \begin{displaymath}
			      \vec{r}^* = \argmax_{\vec r}
			      \left\{
			      \vec r^T \mat \Sigma_{\mbox{inter}} \vec r
			      - \lambda{\vec r^T \mat \Sigma_{\mbox{intra}} \vec r}
			      \right\}
		      \end{displaymath}
	\end{itemize}
\end{frame}


\subsection{Dimensionality Reduction}

\begin{frame}

	\frametitle{Dimensionality Reduction}

	A few comments on dimensionality reduction:

	\begin{itemize}
		\item PCA does not require a classified set of feature vectors \\
		      (in contrast to LDA). \\[.15cm]
		\item PCA transformed features are approximately normally distributed (central limit theorem). \\[.15cm]
		\item Components of PCA transformed features are mutually independent. \\[.15cm]
		\item There exist many other methods for dimensionality reduction, e.\,g.,
		      Sammon transform, independent component analysis. \\[.15cm]
		\item Usually the estimation of transforms is computationally prohibited. \\[.15cm]
		\item \structure{Johnson-Lindenstrauss lemma}: If vectors are projected onto a randomly selected subspace of suitably high dimension, then the distances between the vectors are approximately preserved.
	\end{itemize}
\end{frame}

{
\usebackgroundtemplate{\includegraphics[width=\paperwidth]{png/nextTime.png}}
\begin{frame}[plain]
\end{frame}
}


\subsection{LDA application: adidas\_1 }

\begin{frame}
	\frametitle{The adidas\_1: A Digital Revolution in Sports}

	\vspace{.5cm}
	\begin{columns}[c,onlytextwidth]
		\begin{column}{0.4\textwidth}
			\includegraphics[width=1.25\linewidth]{\jpgdir/adidas_1_components.\jpg}
		\end{column}\begin{column}{0.6\textwidth}
			\begin{itemize}
				\item For the first time ever, sport specific information can be processed with a running shoe
				\item A built-in microprocessor permits an adaptation of the shoe to the prevailing run situation
				      \begin{itemize}
					      \item Running speed
					      \item Runner fatigue
					      \item Running surface
					      \item \ldots
				      \end{itemize}
				\item Pattern Recognition at the LME provides the algorithms used for recognition
			\end{itemize}
		\end{column}
	\end{columns}
\end{frame}


\begin{frame}
	\frametitle{The adidas\_1: System Overview}

	\vspace{.5cm}

	\begin{columns}[c, onlytextwidth]
		\begin{column}{0.5\textwidth}
			\structure{Important parts of the adidas\_1:}
			\begin{itemize}
				\item A cushioning element (01) with a magnetic system for compression measurement \\
				      \begin{itemize}
					      \item $f_\mathsf{sample} = 1\mbox{kHz}$
					      \item resolution $\Delta d = 0.1\,\mbox{mm}$
				      \end{itemize}
				\item A microcontroller and user interface (02)
				      \begin{itemize}
					      \item $f_\mathsf{clock} = 24\,\mbox{MHz}$
					      \item 8 kB program memory
				      \end{itemize}
				\item A motor for cushioning adaptation \\
				      using a cable system (03)
			\end{itemize}
		\end{column}\begin{column}{0.3\textwidth}
			\includegraphics[width=\linewidth]{\jpgdir/adidas_shoe_exploded.\jpg}
		\end{column}
	\end{columns}
\end{frame}


\begin{frame}
	\frametitle{The adidas\_1: Classification Framework Requirements}

	\begin{itemize}
		\item Only a few, simple features can be calculated in real time \\[.3cm]
		\item The classification system has to be efficient, \\
		      but computationally undemanding \\[.3cm]
		\item LDA classifier yields a linear decision boundary and can be implemented using a polynomial of order one with weights $\alpha_i$ and features $x_i$ \\[.3cm]
		\item In the two class case:
		      \begin{displaymath}
			      \mbox{sgn}(\vec\alpha^T\vec x+\alpha_0) =
			      \mbox{sgn} \left ( \alpha_1 x_1 + \alpha_2 x_2 + \ldots + \alpha_d x_d + \alpha_0 \right )
		      \end{displaymath}
		      yields decision for either class.
	\end{itemize}
\end{frame}


\begin{frame}
	\frametitle{Classification System: Computed Features}

	\begin{itemize}
		\item 19 features initially computed for classification experiments
		\item Feature selection: 3 features selected for implementation
	\end{itemize}

	\begin{center}
		\includegraphics[width=.7\linewidth]{\jpgdir/adidas_features.\jpg}
	\end{center}
\end{frame}


\begin{frame}
	\frametitle{Classification System: LDA Classifier Visualization}

	\begin{itemize}
		\item Visualization of the decision region for hard/soft surface classification \\
		      in 3D feature space
	\end{itemize}

	\begin{center}
		\includegraphics[width=.45\linewidth]{\jpgdir/adidas_decision_region.\jpg}
	\end{center}
\end{frame}


\iftrue

	\subsection{PCA application: Shape Modeling}

	\begin{frame}
		\frametitle{Shape Modeling}

		\begin{itemize}
			\item Each shape is represented by $n$ sampled surface points.
			\item Surface points are denoted by $\vec p_k\in \real^3$, $k=1,2, \dots, n$.
			\item The set of surface points is encoded in a single vector (shape vector):
			      \begin{displaymath}
				      \vec x = \left(
				      \begin{array}{c}
						      \vec p_{1} \\
						      \vec p_{2} \\
						      \vdots     \\
						      \vec p_{n}
					      \end{array}
				      \right)
				      = \left(
				      \begin{array}{c}
						      p_{1,1} \\ p_{1,2} \\ p_{1,3} \\ p_{2,1}\\ \vdots \\ p_{n,3}
					      \end{array}
				      \right)
				      \in \real^{3n}
			      \end{displaymath}
			      with $\vec p_k=(p_{k,1}, p_{k,2},  p_{k,3})^T$.
		\end{itemize}
	\end{frame}


	\begin{frame}
		\frametitle{Shape Modeling \cont}

		We have $m$ shapes, thus $m$ shape vectors, and can generate the landmark configuration matrix:

		\begin{displaymath}
			\mat L =  [\vec x_1, \vec x_2, \dots, \vec x_m]
		\end{displaymath}
		\pause

		Now we can compute the PCA of the columns of $\mat L$ and get the spectral decomposition of the associated covariance matrix

		\begin{displaymath}
			\mat \Sigma_L = \sum_i \lambda_i \vec e_i \vec e_i^T
		\end{displaymath}

		where $\lambda_i$ denote the eigenvalues and $\vec e_i$ the eigenvectors.
	\end{frame}


	\begin{frame}
		\frametitle{Shape Modeling \cont}

		Shape vectors $\vec x^*$ within the eigenvector space can be computed using linear combinations of $l$ eigenvectors:

		\begin{displaymath}
			\vec x^* = \bar{\vec x} + \sum_{i=1}^l a_i \vec e_i
		\end{displaymath}

		where $\bar{\vec x}$ denotes just the mean of the column vectors of $\mat L$ and $a_i\in \real$ are the shape parameters.
	\end{frame}


	\begin{frame}
		\frametitle{Application of PCA: Segmentation}

		\begin{itemize}
			\item Lung, liver or kidneys
			\item Generate and train an Active Shape Model (ASM) for such organs; \\
			      requires training data and ``gold standard'' segmentation
			\item Once \structure{point correspondences} are found, the different variations within the training data can be easily approximated by its Eigenvectors
		\end{itemize}

\begin{figure}
  \copyrightbox[b]{
      \makebox[.3\linewidth]{
        \href{https://www.sciencedirect.com/science/article/abs/pii/S0895611108001006}{
          \includegraphics[height=.2\textwidth]{\psdir/ASMFirstEigenMode.\ps}\hspace{2cm}
        }
      }
      \makebox[.3\linewidth]{
        \href{https://www.sciencedirect.com/science/article/abs/pii/S0895611108001006}{
          \includegraphics[height=.2\textwidth]{\psdir/ASMSecondEigenMode.\ps}
        }
      }
  }{M. Spiegel, D. Hahn, V. Daum, J. Wasza, J. Hornegger. \href{https://www.sciencedirect.com/science/article/abs/pii/S0895611108001006} {``Segmentation of kidneys using a new active shape model generation technique based on non-rigid image registration'', Computerized Medical Imaging and Graphics 2009}}
\caption{Variation of the mean kidney shape along the first and second Eigenvector.}
  \end{figure}

	\end{frame}


	\begin{frame}
		\frametitle{Application of PCA: Segmentation \cont}
		
\begin{figure}
  \copyrightbox[b]{
      \makebox[.5\linewidth]{
        \href{https://www.sciencedirect.com/science/article/abs/pii/S0895611108001006}{
          \includegraphics[height=.45\textwidth]{\psdir/ASMSegmentationExample.\ps}
        }
      }
  }{M. Spiegel, D. Hahn, V. Daum, J. Wasza, J. Hornegger. \href{https://www.sciencedirect.com/science/article/abs/pii/S0895611108001006} {``Segmentation of kidneys using a new active shape model generation technique based on non-rigid image registration'', Computerized Medical Imaging and Graphics 2009}}
\caption{Iterative segmentation progress of a right kidney using an ASM.}
  \end{figure}

	\end{frame}


	\begin{frame}
		\frametitle{Application of PCA: Segmentation \cont}


\begin{figure}
  \copyrightbox[b]{
      \makebox[.5\linewidth]{
        \href{https://www.sciencedirect.com/science/article/abs/pii/S0895611108001006}{
          \includegraphics[height=.45\textwidth]{\jpgdir/ASM3DResults.\jpg}
        }
      }
  }{M. Spiegel, D. Hahn, V. Daum, J. Wasza, J. Hornegger. \href{https://www.sciencedirect.com/science/article/abs/pii/S0895611108001006} {``Segmentation of kidneys using a new active shape model generation technique based on non-rigid image registration'', Computerized Medical Imaging and Graphics 2009}}
\caption{3-D view of the segmentation result.}
  \end{figure}
	\end{frame}

\fi

{
\usebackgroundtemplate{\includegraphics[width=\paperwidth]{png/nextTime.png}}
\begin{frame}[plain]
\end{frame}
}


\subsection{Notes on Regression}

\subsubsection{Linear Regression}

\begin{frame}
	\frametitle{Notes on Regression}

	In the two class situation, we set $y\in \{-1,+1\}$ and use the decision rule:

	\begin{displaymath}
		y^* = \mbox{sgn}(\vec \alpha^T\vec x + \alpha_0).
	\end{displaymath}

	We can compute the linear decision boundary simply by least-square estimation.
\end{frame}


\begin{frame}
	\frametitle{Linear Regression \cont}

	For a given set of learning data we use matrix notation:

	\begin{displaymath}
		\mat{X} = \left(
		\begin{array}{cc}
				\vec x_1^T & 1      \\
				\vec x_2^T & 1      \\
				\vdots     & \vdots \\
				\vec x_m^T & 1
			\end{array}
		\right) \in \real^{m\times (d+1)} \qquad \vec y \quad = \quad
		\left(
		\begin{array}{c}
				y_1    \\
				y_2    \\
				\vdots \\
				y_m    \\
			\end{array}
		\right) \in \real^m
	\end{displaymath}

	and define

	\begin{displaymath}
		\vec \theta = { \vec \alpha\choose \alpha_0}
	\end{displaymath}
\end{frame}


\begin{frame}
	\frametitle{Linear Regression \cont}

	One option to estimate $\vec \theta$ is to solve the linear regression problem:

	\begin{displaymath}
		\hat{\vec \theta} = \argmin_{\vec \theta}\| \mat X \vec \theta - \vec y\|_{2}^2
	\end{displaymath}
\end{frame}


\begin{frame}
	\frametitle{Linear Regression \cont}

	The \structure{least-square estimator} for the $L_2$-norm:

	\begin{eqnarray*}
		\widehat{\vec \theta} &=& \argmin_{\vec\theta} \ \sum_{i=1}^m (\vec \theta^T\vec x_i-y_i)^2\\
		&=& \argmin_{\vec\theta} \ (\mat{X}\vec \theta-\vec y)^T(\mat{X}\vec \theta-\vec y)
	\end{eqnarray*}
	\pause

	and thus we get

	\begin{displaymath}
		\widehat{\vec \theta} = (\mat{X}^T\mat{X})^{-1}\mat{X}^T\vec y
	\end{displaymath}

	if the column vectors of $\mat X$ are mutually independent.
\end{frame}


\begin{frame}
	\frametitle{Linear Regression \cont}

	\structure{A few obvious questions:} \\[.5cm]

	\begin{itemize}
		\item Why should we prefer the Euclidean norm ($L_2$-norm)? \\[.5cm]
		\item Will different norms lead to different results? \\[.5cm]
		\item Which norm and decision boundary is the best one? \\[.5cm]
		\item Can we incorporate prior knowledge in linear regression?
	\end{itemize}
\end{frame}


\subsubsection{Ridge Regression}

\begin{frame}
	\frametitle{Ridge Regression}

	In \structure{\emph{ridge regression}} (also called \emph{regularized regression}) we extend the \\
	objective function by an additional term constraining the Euclidean length \\
	of the parameter vector $\vec{\theta}$:

	\begin{itemize}
		\item It is linear regression with the log-likelihood penalized by $-\lambda\vec \theta^T \vec \theta$ \\
		      where $\lambda > 0$, or alternatively \\[.5cm]
		\item It is extended by a prior distribution on the parameter vector $\vec\theta$
		      \begin{displaymath}
			      \vec \theta = {\cal N}(\vec 0,\mbox{diag}( \tau^2) )
		      \end{displaymath}
	\end{itemize}
\end{frame}


\begin{frame}
	\frametitle{Ridge Regression \cont}

	\structure{Regularized regression:}

	\begin{eqnarray*}
		\widehat{\vec\theta}
		&=& \argmin_{\vec\theta} \| \mat{X} \vec\theta - \vec y \|^2_2 + \lambda \|\vec\theta\|^2_2 \\ \pause
		&=& \argmin_{\vec\theta} \ (\mat{X} \vec\theta - \vec y)^T (\mat{X} \vec\theta-\vec y) +
		\lambda \cdot \vec\theta^T\vec \theta
	\end{eqnarray*}
	\pause

	and thus we get the estimator:

	\begin{displaymath}
		\widehat{\vec\theta} = (\mat{X}^T\mat{X}+\lambda \mat{I})^{-1}\mat{X}^T\vec y
	\end{displaymath}
\end{frame}


\begin{frame}
	\frametitle{Ridge Regression \cont}

	\structure{Notes:}

	\begin{itemize}
		\item The term $\lambda \mat{I}$ adds a positive constant $\lambda$ to the diagonal elements.
		\item The problem is non-singular even if $\mat{X}^T \mat{X}$ is not of full rank. \\
		\item This was the main motivation of ridge regression when it was first introduced in statistics in 1970. \\[.25cm] \pause
		\item The ridge solutions are not equivariant under scaling of the inputs: \\
		      \structure{standardize} the input before solving the regression problem! \\[.25cm] \pause
		\item The intercept $\alpha_0$ should not be penalized:
		      \begin{itemize}
			      \item \structure{Center} the input $\vec{x}_i$.
			      \item Estimate $\alpha_0$ by $\bar{y} = \frac{1}{m} \sum_{i=1}^m y_i$.
			      \item Estimate the remaining coefficients by a ridge regression without  intercept. \\
			            Matrix $\vec{X}$ has $d$ columns (instead of $d+1$).
		      \end{itemize}
	\end{itemize}
\end{frame}


\begin{frame}
	\frametitle{Ridge Regression \cont}

	\structure{Statistical approach:} parameters $\alpha_j$ are random variables \\[.5cm]

	\begin{itemize}
		\item Suppose
		      \begin{displaymath}
			      \forall 1 \le i \le m: \quad y_i
			      \sim \mathcal{N}(\underbrace{\vec{\alpha}^T \vec{x}_i + \alpha_0}_\text{mean}, \underbrace{\sigma^2}_\text{variance})
			      \onslide<2->{
				      = \frac{1}{\sqrt{2 \pi} \cdot \sigma} e^{-\frac{1}{2} \cdot \frac{(y_i - \vec{\alpha}^T \vec{x}_i - \alpha_0)^2}{\sigma^2}}
			      }
		      \end{displaymath}
		\item Parameters $\alpha_j$ are assumed to be independent of each other.
		\item Prior distribution of $\alpha_j$:
		      \begin{displaymath}
			      \forall 1 \le j \le d: \quad \alpha_j
			      \sim \mathcal{N}(0, \tau^2)
			      \onslide<3->{
				      = \frac{1}{\sqrt{2 \pi} \cdot \tau} e^{-\frac{1}{2} \cdot \frac{(\alpha_j - 0)^2}{\tau^2}}
			      }
		      \end{displaymath}
	\end{itemize}
\end{frame}


\begin{frame}
	\frametitle{Ridge Regression \cont}

	\begin{itemize}
		\item \structure{Maximizing the posterior probability of $\vec{\alpha}$ for given $\sigma^2$ and $\tau^2$:}
		      \footnotesize
		      \begin{eqnarray*}
			      \argmax_{\vec{\alpha}} \prod_{i=1}^m p(\vec{\alpha}|y_i) \pause
			      &=& \argmax_{\vec{\alpha}} \Bigg\{ \prod_{i=1}^m p(\vec{\alpha}) \cdot p(y_i | \vec{\alpha}) \Bigg\} \\ \pause
			      &=& \argmax_{\vec{\alpha}} \Bigg\{ \prod_{j=1}^d p(\alpha_j) \cdot \prod_{i=1}^m p(y_i | \vec{\alpha}) \Bigg\} \\ \pause
			      &=& \argmax_{\vec{\alpha}} \Bigg\{ \sum_{j=1}^d \log p(\alpha_j) + \sum_{i=1}^m \log p(y_i | \vec{\alpha}) \Bigg\} \\ \pause
			      &=& \argmax_{\vec{\alpha}} \Bigg\{ - \frac{1}{2 \tau^2} \sum_{j=1}^d \alpha_j^2 - \frac{1}{2 \sigma^2} \sum_{i=1}^m (y_i - \vec{\alpha}^T \vec{x} - \alpha_0)^2 \Bigg\} \\ \pause
			      &=& \argmin_{\vec{\alpha}} \Bigg\{ \frac{\sigma^2}{\tau^2} \sum_{j=1}^d \alpha_j^2 + \sum_{i=1}^m (y_i - \vec{\alpha}^T \vec{x} - \alpha_0)^2 \Bigg\}
		      \end{eqnarray*}
	\end{itemize}
\end{frame}


\begin{frame}
	\frametitle{Ridge Regression \cont}

	\begin{itemize}
		\item \structure{Maximizing the posterior probability of $\vec{\alpha}$ for given $\sigma^2$ and $\tau^2$:} \\[.25cm]
		      {\footnotesize
		      \begin{displaymath}
			      \argmax_{\vec{\alpha}} \prod_{i=1}^m p(\vec{\alpha}|y_i)
			      = \argmin_{\vec{\alpha}} \Bigg\{ \lambda \vec{\alpha}^T \vec{\alpha} + (\mat{X}\vec{\alpha} - \vec{y})^T (\mat{X}\vec{\alpha} - \vec{y}) \Bigg\}
			      \quad \text{with} \quad
			      \lambda = \frac{\sigma^2}{\tau^2}
		      \end{displaymath}
		      }
		      \vspace{.25cm} \pause
		\item \structure{The ridge estimate is the mode of the posterior pdf!}
	\end{itemize}
\end{frame}


\subsubsection{Lasso}

\begin{frame}
	\frametitle{Lasso}
	Regularized regression using a mixture of $L_2$- and $L_1$-norm, \\
	where the residual is penalized using the $L_2$-norm and \\
	the regularizer uses the $L_1$-norm:

	\begin{displaymath}
		\widehat{\vec \theta} = \argmin_{\vec\theta} \|\mat{X}\vec \theta-\vec y\|^2_2 + \lambda \cdot \|\vec \theta\|_1
	\end{displaymath}
	\pause

	The lasso is used to compute a sparse solution of the system of \\
	linear equations, i.\,e.\ the number of non-zero elements in $\vec \theta$ shall be small.
\end{frame}


\subsection{Lessons Learned}

\begin{frame}
	\frametitle{Lessons Learned}

	\begin{itemize}
		\item Principal component analysis \\[.5cm]
		\item Linear discriminant analysis with and without dimension reduction \\[.5cm]
		\item Both PCA and LDA relate to an eigenvalue eigenvector problem \\[.5cm]
		\item Alternative formulation of LDA using the Fisher transform \\[.5cm]
		\item Linear and ridge regression for classification
	\end{itemize}
\end{frame}

{
\usebackgroundtemplate{\includegraphics[width=\paperwidth]{png/nextTime.png}}
\begin{frame}[plain]
\end{frame}
}


\subsection{Further Readings}

\begin{frame}
	\frametitle{Further Readings}

	You are required to be familiar with \structure{linear algebra} and \structure{matrix calculus}:

	\begin{itemize}
		\item SIAMS best selling book in the last decade:\\[.15cm]
		      Lloyd N. Trefethen, David Bau III: \\
		      \structure{Numerical Linear Algebra}, \\
		      SIAM, Philadelphia, 1997. \\[0.15cm]
		\item All about matrix derivatives and related problems is described in the Matrix Cookbook:
		      \structure{\url{http://www.matrixcookbook.com}} \\[.3cm]
	\end{itemize}

	Basics on \structure{discriminant analysis} can be found in

	\begin{itemize}
		\item T. Hastie, R. Tibshirani, and J. Friedman: \\
		      \structure{The Elements of Statistical Learning --}\\
		      \structure{ Data Mining, Inference, and Prediction},\\
		      2nd edition, Springer, New York, 2009.
	\end{itemize}
\end{frame}


\begin{frame}
	\frametitle{Further Readings \cont}

	\structure{Details on the adidas\_1 shoe and the implemented classifier:}

	\begin{itemize}
		\item B.~Eskofier, F.~H\"onig, P.~K\"uhner: \\
		      \structure{Classification of Perceived Running Fatigue in Digital Sports}, \\
		      Proceedings of the 19th International Conference on Pattern Recognition (ICPR 2008), Tampa, Florida, U.\,S.\,A., 2008
	\end{itemize}
	\spread

	\structure{Details on the shape modeling of kidneys and its application to segmentation:}

	\begin{itemize}
		\item M.~Spiegel, D.~Hahn, V.~Daum, J.~Wasza, J.~Hornegger: \\
		      \structure{Segmentation of kidneys using a new active shape model generation technique based on non-rigid image registration}, \\
		      Computerized Medical Imaging and Graphics 2009 33(1):29-39
	\end{itemize}
\end{frame}


\subsection{Comprehensive Questions}

\begin{frame}
	\frametitle{Comprehensive Questions}

	\begin{itemize}
		\item What is the difference between PCA and LDA? \\[1cm]
		\item How can PCA and LDA be combined to achieve a high rank reduction? \\[1cm]
		\item Write down a straight forward objective function for linear regression! \\[1cm]
		\item What happens if we replace the $L_2$-norm by another norm?
	\end{itemize}
\end{frame}

