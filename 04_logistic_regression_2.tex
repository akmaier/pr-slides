\section{Logistic Regression II}

\subsection{Parameterization}

\begin{frame}
  \frametitle{Parameterization}
  
  \begin{itemize}
    \item Until now, $F(\vec x)$ was some arbitrary function in $\vec x$ \\[.1cm]
      \structure{Example:} $ F(\vec x) = \vec x^T \mat A \vec x + \vec \alpha^T\vec x + \alpha_0 $ with components defined by Gaussian distributions \\[.25cm] \pause
    \item We can express a nonlinear $F(\vec x)$ as a scalar product by lifting $\vec x$ into a higher dimensional space: \\
      Given \\
      \hspace{0.5cm} $\vec x = \left(x_1, x_2\right)^T \in \real^2,~ \mat A = \left( \begin{array}{cc} a_{11} & a_{12} \\ a_{21} & a_{22} \end{array} \right), ~ \vec \alpha = \left(\alpha_1, \alpha_2 \right)^T, ~\alpha_0 $\,, \\
      then \\
      \hspace{0.5cm} $F(\vec x) = a_{11} x_1^2 + (a_{12} + a_{21}) x_1 x_2 + a_{22} x_2^2 + \alpha_1 x_1 + \alpha_2 x_2 + \alpha_0$\,.\\[.25cm] \pause
    \item Rewrite $F(\vec x) = \vec \theta^T \vec x'$ with $\vec \theta, \vec x' \in \real^6$: \\ \pause
      \hspace{0.5cm} $\vec \theta = (a_{11}, a_{12} + a_{21}, a_{22}, \alpha_1, \alpha_2, \alpha_0)^T$ \\
      \hspace{0.5cm} $\vec x' = ( x_1^2, x_1 x_2, x_2^2, x_1, x_2, 1)^T$
  \end{itemize}
\end{frame}


\begin{frame}
  \frametitle{Parameterization \cont}

  \begin{citeblock}{Definition}

    We write the parameterized logistic function in the following:
    \begin{eqnarray*}
      g(\vec \theta^T \vec x) &=& \frac{1}{1+e^{-\vec \theta^T \vec x}}
    \end{eqnarray*}
    where $\vec \theta, \vec x$ are the lifted parameters of the original decision function $F$ \\
    (if it was not already linear).
  \end{citeblock}
\end{frame}


\subsection{Learning in Logistic Regression}

\subsubsection{Log-Likelihood Function}

\begin{frame}
  \frametitle{Log-Likelihood Function}
 
  \begin{itemize}
    \item Let us assume the posteriors are given by
      \begin{eqnarray*}
        p(y=0|\vec x) &=&  1-g(\vec \theta^T\vec x) \\
        p(y=1|\vec x) &=&  g(\vec \theta^T\vec x)
      \end{eqnarray*}
      where $g(\vec \theta^T\vec x)$ is the sigmoid function parameterized in $\vec \theta$. \\[.3cm]
    \item The parameter vector $\vec{\theta}$ has to be estimated from a set $S$ of $m$ training samples:
      \begin{eqnarray*}
        S &=& \{ (\vec x_1, y_1), (\vec x_2, y_2), (\vec x_3, y_3), \dots, (\vec x_m, y_m) \}\quad .
      \end{eqnarray*}
      \pause
    \item Method of choice: Maximum Likelihood Estimation
  \end{itemize}
\end{frame}


\begin{frame}
  \frametitle{Log-Likelihood Function \cont}
 
  Before we work on the formulas of the ML-estimator, we rewrite the posteriors
  using Bernoulli probability:

  \begin{eqnarray*}
    p(y|\vec x) &=& \pause g(\vec \theta^T\vec x)^{y}(1-g(\vec \theta^T\vec x))^{1-y}\\[.3cm] 
  \end{eqnarray*}

  which shows the great benefit of the chosen notation for class numbers.
\end{frame}


\begin{frame}
  \frametitle{Log-Likelihood Function \cont}
 
  Now we can compute the log-likelihood function \\
  (assuming that the training samples are mutually independent):
 
  \begin{eqnarray*}
    \mathcal{L} (\vec \theta) &=& \log \left( \prod_{i=1}^m p(y_i|\vec x_i) \right) \\ \pause
                              &=& \sum_{i=1}^m \log \left( g(\vec \theta^T\vec x_i)^{y_i}\,\big(1-g(\vec \theta^T\vec x_i)\big)^{1-y_i} \right) \\ \pause
                              &=& \sum_{i=1}^m \left(y_i \log g(\vec \theta^T\vec x_i) + (1-y_i)\log\big(1-g(\vec \theta^T\vec x_i)\big) \right)
  \end{eqnarray*}
\end{frame}


\begin{frame}
  \frametitle{Log-Likelihood Function \cont}
 
  Simplification of the log-likelihood function:
 
  \begin{eqnarray*}
    \mathcal{L} (\vec \theta) 
      &=& \sum_{i=1}^m \left( y_i \log g(\vec \theta^T\vec x_i) + (1-y_i)\log\big(1-g(\vec \theta^T\vec x_i)\big) \right) \\ \pause
      &=& \sum_{i=1}^m \left( y_i \log \frac{e^{\vec \theta^T \vec x_i}}{1 + e^{\vec \theta^T \vec x_i}} + (1 - y_i) \log \frac{1}{1 + e^{\vec \theta^T \vec x_i}} \right) \\ \pause
      &=& \sum_{i=1}^m \left( y_i \vec \theta^T \vec x_i + \log \frac{1}{1 + e^{\vec \theta^T \vec x_i}} \right) \\ \pause
      &=& \sum_{i=1}^m \left( y_i \vec \theta^T \vec x_i + \log \big( 1 - g(\vec \theta^T \vec x_i) \big) \right)
  \end{eqnarray*}
\end{frame}


\begin{frame}
  \frametitle{Log-Likelihood Function \cont}

  \structure{Notes for the expert:} 

  \begin{itemize}
    \item The negative of the log-likelihood function is the cross entropy of\\
      $y$ and $g(\vec \theta^T\vec x)$. \\[.5cm]
    \item The negative of the log-likelihood function is a convex function.
  \end{itemize}
\end{frame}


{
\usebackgroundtemplate{\includegraphics[width=\paperwidth]{png/nextTime.png}}
\begin{frame}[plain]
\end{frame}
}


\subsubsection{Newton-Raphson Iteration}

\begin{frame}
  \frametitle{Maximization of the log-likelihood function}

  \begin{itemize}
    \item The log-likelihood function is concave. 
    \item We use the \point\href{http://www.stat.washington.edu/quinn/classes/560/Newton.pdf}{\structure{Newton-Raphson}} algorithm to solve the unconstrained optimization problem:
      \spread

      For the $(k+1)$-st iteration step, we get:
%
      \begin{eqnarray*}
        \vec \theta^{(k+1)} &=& \vec \theta^{(k)} - \left(  \frac{\partial^2}{\partial\vec \theta \partial \vec \theta^T} \mathcal{L} \left(\vec \theta^{(k)} \right)  \right)^{-1}   
       \frac{\partial}{\partial\vec \theta} \mathcal{L}\left(\vec \theta^{(k)}\right)
      \end{eqnarray*}
  \end{itemize}
  \spread
 
  \structure{Note:} If you write the Newton-Raphson iteration in matrix form, you will end up with a weighted least squares iteration scheme.
\end{frame}


\begin{frame}
  \frametitle{Newton-Raphson Iteration}

  \structure{Taylor's Theorem:}\\[.3cm]
   
  Approximation of a $k$-times differentiable function $f(x)$ \\
  around a given point $x_0$:
 
  {\small
  \begin{displaymath}
    f(x_0+h) = f(x_0) + f'(x_0) h + \frac{f''(x_0)}{2!} h^2 + \ldots +
           \frac{f^{(k)}(x_0)}{k!} h^k + r_k(x_0 + h) h^k, 
           \quad \lim_{h \rightarrow 0} r_k(x_0 + h) = 0
  \end{displaymath}
  }
  \pspread

  \structure{Second order Taylor polynomial:}
  \begin{displaymath}
    f(x_0 + h) \approx f(x_0) + f'(x_0) h + \frac{1}{2} f''(x_0) h^2
  \end{displaymath}
\end{frame}
  

\begin{frame}
  \frametitle{Newton-Raphson Iteration \cont}

  \structure{Extremum:}

  \begin{eqnarray*}
    f'(x_0 + h)         & = & f'(x_0) + f''(x_0) h ~\stackrel{!}{=}~ 0 \\[.3cm]
                \hat{h} & = & - \frac{f'(x_0)}{f''(x_0)} \\[.3cm]
    x_1 = x_0 + \hat{h} & = & x_0 - \frac{f'(x_0)}{f''(x_0)} 
  \end{eqnarray*}
\end{frame}


\begin{frame}
  \frametitle{Newton-Raphson Iteration \cont}

  \begin{center}
    \resizebox{.7\linewidth}{!}{
      \alt<8->{
        \input{\texfigdir/newton-raphson8.pstex_t}
      }{\alt<7>{
        \input{\texfigdir/newton-raphson7.pstex_t}
      }{\alt<6>{
        \input{\texfigdir/newton-raphson6.pstex_t}
      }{\alt<5>{
        \input{\texfigdir/newton-raphson5.pstex_t}
      }{\alt<4>{
        \input{\texfigdir/newton-raphson4.pstex_t}
      }{\alt<3>{
        \input{\texfigdir/newton-raphson3.pstex_t}
      }{\alt<2>{
        \input{\texfigdir/newton-raphson2.pstex_t}
      }{
        \input{\texfigdir/newton-raphson1.pstex_t}
      }}}}}}}
    }
  \end{center}
\end{frame}


\begin{frame}
  \frametitle{Gradient of the Log-Likelihood Function}

  \structure{The gradient:}
 %
  {\small
  \begin{eqnarray*}
    \frac{\partial}{\partial \theta_j} \mathcal{L}(\vec \theta)
      &=& \frac{\partial}{\partial \theta_j} \left( \sum_{i=1}^m \left( y_i \vec \theta^T \vec x_i + \log \big( 1 - g(\vec \theta^T \vec x_i) \big) \right) \right)\\ \pause
      &=& \sum_{i=1}^m \left( y_i x_{i,j} - \frac{1}{1-g(\vec \theta^T\vec x_i)} \frac{\partial}{\partial \theta_j}g(\vec \theta^T \vec x_i) \right)
  \end{eqnarray*}
  }
  \pause 
%  
  Now we use the derivative of the sigmoid function and get
%
  {\small
  \begin{eqnarray*}
     \frac{\partial}{\partial \theta_j} \mathcal{L}(\vec \theta) 
       &=& \sum_{i=1}^m \left( y_i x_{i,j} - \frac{1}{1-g(\vec \theta^T\vec x_i)} g(\vec \theta^T \vec x_i) \big(1-g(\vec \theta^T \vec x_i)\big) x_{i,j} \right) \\ \pause
       &=& \sum_{i=1}^m \left( y_i - g(\vec \theta^T\vec x_i) \right) x_{i,j}
  \end{eqnarray*}
  }
% 
   where $x_{i,j}$ is the $j$-th component of the $i$-th training feature vector.
\end{frame}


\begin{frame}
  \frametitle{Gradient of the Log-Likelihood Function \cont}
  
  Finally, we have a quite simple gradient: 
  
  {\small
  \begin{eqnarray*}
    \frac{\partial}{\partial \theta_j} \mathcal{L}(\vec \theta) 
    &=& \sum _{i=1}^m  \left( y_i - g(\vec \theta^T\vec x_i) \right) x_{i,j}
  \end{eqnarray*}
  }

  where $x_{i,j}$ is the $j$--th component of the $i$--th training feature vector. \\[.3cm]
 
  Or in vector notation:
  {\small
  \begin{eqnarray*}
    \frac{\partial}{\partial\vec \theta} \mathcal{L}(\vec \theta) 
    &=& \sum _{i=1}^m  \left( y_i-g(\vec \theta^T\vec x_i) \right)\vec x_{i}
  \end{eqnarray*}
  }
\end{frame}


\begin{frame}
  \frametitle{Hessian of the Log-Likelihood Function}

  \begin{itemize}
    \item The Newton-Raphson algorithm requires the Hessian matrix.
    \item Remember the derivative of the sigmoid function!
  \end{itemize}

  \begin{eqnarray*}
    \frac{\partial^2}{\partial\vec \theta \partial \vec \theta^T} \mathcal{L}(\vec \theta) &=& -
    \sum _{i=1}^m  g(\vec \theta^T\vec x_i) \left(1-g(\vec \theta^T\vec x_i)\right)\vec x_i \vec x_i^T
  \end{eqnarray*}  
\end{frame}


\subsection{Perceptron and Logistic Regression}

\begin{frame}
   \frametitle{Perceptron and Logistic Regression}
   
   \begin{center}
     \resizebox{.7\linewidth}{!}{
       \input{\texfigdir/perceptron.pstex_t}
     }
   \end{center}
\end{frame}


\subsection{Lessons Learned}

\begin{frame}
  \frametitle{Lessons Learned}
   
  \begin{itemize}
    \item Posteriors can be rewritten in terms of a logistic function.\\[.5cm]
    \item Given the decision boundary $F(\vec x)=0$, we can write down the posterior $p(y|\vec x)$ right away.\\[.5cm]
    \item Decision boundary for normally distributed feature vectors for each class is a quadratic function.\\[.5cm]
    \item If Gaussians share the same covariances, the decision boundary is a linear function.
  \end{itemize}
\end{frame}

{
\usebackgroundtemplate{\includegraphics[width=\paperwidth]{png/nextTime.png}}
\begin{frame}[plain]
\end{frame}
}


\subsection{Further Readings}

\begin{frame}
  \frametitle{Further Readings}

  \begin{itemize}
    \item T. Hastie, R. Tibshirani, and J. Friedman: \\
      \structure{The Elements of Statistical Learning --}\\
      \structure{ Data Mining, Inference, and Prediction},\\
      2nd edition, Springer, New York, 2009. \\[.3cm]
    \item David W. Hosmer, Stanley Lemeshow: \\
      \structure{Applied Logistic Regression}, 2nd Edition, \\
      John Wiley \& Sons, Hoboken, 2000.
  \end{itemize}
\end{frame}


\subsection{Comprehensive Questions}

\begin{frame}
  \frametitle{Comprehensive Questions \cont}

  \begin{itemize}
    \item How can a nonlinear function be written as a scalar product? \\[.7cm] \pause
    \item What is the objective function for the ML-estimation of the logistic regression parameters? \\[.7cm] \pause
    \item What is the difference between a gradient descent and Newton-Raphson numerical optimization scheme? \\[.7cm] \pause
    \item What is the parameter update rule for the logistic regression parameters using the Newton-Raphson scheme?
  \end{itemize}
\end{frame}

