\section{Pattern Recognition Basics}

\subsection{Classification of Simple Patterns}

\begin{frame}
	\frametitle{Classification of Simple Patterns}

	The system for the classification of simple patterns has the following generic structure\\
	\vspace{2cm}
	\pause
%\begin{centering}
	$\hspace{1.03cm} \overset{\vec f}{\longrightarrow}\fbox{Preprocessing}\overset{\vec g}{\longrightarrow}\fbox{Feature Extraction} \overset{\vec c}{\longrightarrow}\fbox{Classification} \overset{y}{\longrightarrow}$\\
	$\hspace{1.03cm}\hspace{8.5cm} \uparrow $\\
	$\hspace{1.03cm}\hspace{4.3cm} \fbox{Training Samples}\longrightarrow \fbox{Learning}$
\end{frame}

\begin{frame}
	\frametitle{Classification of Simple Patterns \cont}

	\begin{itemize}
		\item {\em \structure{Supervised learning:}}

		      $m$ training samples include feature and associated class number
		      \begin{displaymath}
			      S = \{ (\vec x_1, y_1), (\vec x_2, y_2), (\vec x_3, y_3), \dots, (\vec x_m, y_m) \}
		      \end{displaymath}
              where $\vec x_i \in \mathcal{X}$ denotes the feature vector and $y_i\in Z$ denotes the class number of sample $i$.
	      If nothing special is mentioned ${\mathcal{X}}\subseteq \mathbb{R}^d$.
		      \pause
		      \vspace{0.5cm}
		\item {\em \structure{Unsupervised learning:}}

		      $m$ training samples just include features, no class assignments and even the number of classes is (not always) known
		      \begin{displaymath}
			      S = \{ \vec x_1, \vec x_2, \vec x_3, \dots, \vec x_m \}
		      \end{displaymath}
	\end{itemize}
\end{frame}


\subsection{Bayesian Classifier}

\begin{frame}
	\frametitle{Bayesian Classifier}

	\structure{Notation:}

	\begin{center}
		\begin{minipage}{0.7\textwidth}
			\begin{itemize}
				\item[ $ \vec x \in \mathbb{R}^d:$]  $d$-dimensional feature vector
				\item[ $y:$] class number \\
				      (usually $y\in\{0,1\}$ or $y\in\{-1,+1\}$)
				\item[$p(y):$] prior probability of pattern class $y$
				\item[$p(\vec x):$] evidence\\
				      (distribution of features in $d$-dimensional feature space)
				\item[$p(\vec x , y):$] joint probability density function (pdf)
				\item[$p(\vec x |y):$] class conditional density
				\item[$p(y| \vec x):$] posterior probability
			\end{itemize}
		\end{minipage}
	\end{center}
\end{frame}


\begin{frame}
	\frametitle{Bayesian Classifier \cont}
\vspace{-.52cm}
  \begin{figure}
    \resizebox{.85\linewidth}{!}{
      \alt<25->{
        \includegraphics[width=.85\linewidth]{\pngdir/condProb/Folie25.\png}
      }{\alt<24>{
        \includegraphics[width=.85\linewidth]{\pngdir/condProb/Folie24.\png}
      }{\alt<23>{
        \includegraphics[width=.85\linewidth]{\pngdir/condProb/Folie23.\png}
      }{\alt<22>{
        \includegraphics[width=.85\linewidth]{\pngdir/condProb/Folie22.\png}
      }{\alt<21>{
        \includegraphics[width=.85\linewidth]{\pngdir/condProb/Folie21.\png}
      }{\alt<20>{
        \includegraphics[width=.85\linewidth]{\pngdir/condProb/Folie20.\png}
      }{\alt<19>{
        \includegraphics[width=.85\linewidth]{\pngdir/condProb/Folie19.\png}
      }{\alt<18>{
        \includegraphics[width=.85\linewidth]{\pngdir/condProb/Folie18.\png}
      }{\alt<17>{
        \includegraphics[width=.85\linewidth]{\pngdir/condProb/Folie17.\png}
      }{\alt<16>{
        \includegraphics[width=.85\linewidth]{\pngdir/condProb/Folie16.\png}
      }{\alt<15>{
        \includegraphics[width=.85\linewidth]{\pngdir/condProb/Folie15.\png}
      }{\alt<14>{
        \includegraphics[width=.85\linewidth]{\pngdir/condProb/Folie14.\png}
      }{\alt<13>{
        \includegraphics[width=.85\linewidth]{\pngdir/condProb/Folie13.\png}
      }{\alt<12>{
        \includegraphics[width=.85\linewidth]{\pngdir/condProb/Folie12.\png}
      }{\alt<11>{
        \includegraphics[width=.85\linewidth]{\pngdir/condProb/Folie11.\png}
      }{\alt<10>{
        \includegraphics[width=.85\linewidth]{\pngdir/condProb/Folie10.\png}
      }{\alt<9>{
        \includegraphics[width=.85\linewidth]{\pngdir/condProb/Folie9.\png}
      }{\alt<8>{
        \includegraphics[width=.85\linewidth]{\pngdir/condProb/Folie8.\png}
      }{\alt<7>{
        \includegraphics[width=.85\linewidth]{\pngdir/condProb/Folie7.\png}
      }{\alt<6>{
        \includegraphics[width=.85\linewidth]{\pngdir/condProb/Folie6.\png}
      }{\alt<5>{
        \includegraphics[width=.85\linewidth]{\pngdir/condProb/Folie5.\png}
      }{\alt<4>{
        \includegraphics[width=.85\linewidth]{\pngdir/condProb/Folie4.\png}
      }{\alt<3>{
        \includegraphics[width=.85\linewidth]{\pngdir/condProb/Folie3.\png}
      }{\alt<2>{
        \includegraphics[width=.85\linewidth]{\pngdir/condProb/Folie2.\png}
      }{
        \includegraphics[width=.85\linewidth]{\pngdir/condProb/Folie1.\png}
      }}}}}}}}}}}}}}}}}}}}}}}}
    }
  \end{figure}

\end{frame}

\begin{frame}
	\frametitle{Bayesian Classifier \cont}

	\structure{Bayes rule:}\\[.5cm]

	\begin{eqnarray*}
		\underbrace{p(\vec x , y)}_{joint\ pdf}
		&=& \pause \underbrace{p(y)}_{prior} \cdot \underbrace{p(\vec x | y)}_{class\ conditional\ pdf} \\[.5cm] \pause
		&=& \underbrace{ p(\vec x)}_{evidence} \cdot \underbrace{ p(y| \vec x)}_{posterior}
	\end{eqnarray*}
\end{frame}


\begin{frame}
	\frametitle{Bayesian Classifier \cont}

	Now we get the posterior as follows:

	\begin{eqnarray*}
		p(y| \vec x)
		&=& \pause \frac{p(y) \cdot p(\vec x | y)}{p(\vec x)} \\ \pause
		&=& \frac{p(y) \cdot p(\vec x | y)}{\sum\limits_{y'}p(\vec x , y')} \\ \pause
		&=& \frac{p(y) \cdot p(\vec x | y)}{\sum\limits_{y'}p(y') \cdot p(\vec x | y')}\\
	\end{eqnarray*}
\end{frame}


\begin{frame}
	\frametitle{Bayesian Classifier \cont}

	\structure{Note:}
	\begin{displaymath}
		p(\vec x) = \sum\limits_{y}p(y) \cdot p(\vec x | y)
	\end{displaymath}
	is a \structure{marginal} of $p(\vec x , y)$.

	\begin{itemize}
		\item We get $p(\vec{x})$ by marginalizing $p(\vec{x}, y)$ over $y$.
		\item Accordingly we get $p(y)$ by marginalizing $p(\vec{x}, y)$ over $\vec{x}$, i.\,e.
		      \begin{eqnarray*}
			      p(y)&=& \int p(\vec{x}, y) \mathsf{d}\vec{x}
		      \end{eqnarray*}
	\end{itemize}

	\alert{Did you notice:} $y$ is a discrete random variable whereas $\vec{x}$ is a continuous random vector (summation vs.\ integration).
\end{frame}


{
\usebackgroundtemplate{\includegraphics[width=\paperwidth]{png/nextTime.png}}
\begin{frame}[plain]
\end{frame}
}


\begin{frame}
	\frametitle{Bayesian Classifier \cont}

	Now let us summarize the Bayesian decision rule:\\[.3cm]
	We decide for the class $y^*$ according to the decision rule
	\begin{eqnarray*}
		y^* &=& \pause \argmax\limits_{y} p(y | \vec x)\ \\[.3cm] \pause
		&=& \argmax\limits_{y} \frac{p(y) \cdot p(\vec x | y)}{p(\vec x)} \\[.3cm] \pause
		&=& \argmax\limits_{y} p(y) \cdot p(\vec x | y) \\[.3cm] \pause
		&=& \argmax\limits_{y} \{\log p(y)\ + \log p(\vec x | y)\}
	\end{eqnarray*}
\end{frame}


\begin{frame}
	\frametitle{Bayesian Classifier \cont}

	\structure{Notes:} \\[.3cm]

	\begin{itemize}
		\item The key aspect in designing a classifier is to find a good model \\
		      for the posterior $p(y|\vec x)$. \\[.3cm]
		\item Feature vectors $\vec x$ usually have fixed dimensions $d$ in simple classification schemes, \\[.3cm]
		\item but ${\mathcal{X}}$ is not necessarily a subset of $\mathbb{R}^d$: \\
		      features of varying dimension, sequences and sets of features
	\end{itemize}
\end{frame}


\begin{frame}
	\frametitle{Bayesian Classifier \cont}

	\begin{itemize}
		\item \structure{Generative modeling:} \\
		      modeling and estimation of $p(y)$ and $p(\vec x | y)$. \\[.5cm]
		\item \structure{Discriminative modeling:} \\
		      straight modeling and estimation of $p(y|\vec x)$.
	\end{itemize}
\end{frame}


\subsection{Optimality of the Bayesian Classifier}

\begin{frame}
	\frametitle{Optimality of the Bayesian Classifier}

	%  \begin{citeblock}{Definition}
	\begin{definition}
		$l(y_{1},y_{2})$ is the \structure{loss} if a feature vector belonging to class $y_{2}$
		is assigned to class $y_{1}$. The $(0,1)$-loss function is defined by
		\begin{eqnarray*}
			l(y_{1},y_{2}) &= &\left\{ \begin{array}{cc}
				0 & ,\ if\ y_{1}=y_{2} \\
				1 & ,\ otherwise
			\end{array} \right.
		\end{eqnarray*}
		%  \end{citeblock}
	\end{definition}
\end{frame}


\begin{frame}
	\frametitle{Optimality of the Bayesian Classifier \cont}

	The \structure{best (or optimal) decision rule} according to classification loss minimizes the average loss L:

	\begin{eqnarray*}
		\mathsf{AL}(\vec x , y) &=& \sum\limits_{y'}l(y,y')p(y'|\vec x)
	\end{eqnarray*}

\end{frame}


\begin{frame}
	\frametitle{Optimality of the Bayesian Classifier \cont}

	Using the $(0,1)$-loss function, the class decision is based on:

	\begin{eqnarray*}
		y^* &=& \argmin\limits_{y} \mathsf{AL}(\vec x, y)\\
		\pause  &=& \argmin\limits_{y} \sum\limits_{y'} l(y,y') \cdot p(y'|\vec x)\\
		%           \pause  &=& \argmin\limits_{y} (1-p(y|\vec x))\\
		\pause  &=& \argmax\limits_{y} p(y|\vec x)\\
	\end{eqnarray*}
\end{frame}


\begin{frame}
	\frametitle{Optimality of the Bayesian Classifier \cont}

	\structure{Conclusion:}

	\begin{itemize}
		\item The optimal classifier w.\,r.\,t.\ the (0,1)-loss function applies the Bayesian decision rule.
		\item This classifier is called \structure{Bayesian classifier}. \\[.75cm]
	\end{itemize}
	\spread

	\vorsicht The loss function is {\bf NOT} convex. \vfill
\end{frame}


\subsection{Lessons Learned}

\begin{frame}
	\frametitle{Lessons Learned}

	\begin{itemize}
		\item General structure of a classification system \\[.5cm] \pause
		\item Supervised and unsupervised learning \\[.5cm] \pause
		\item Basics on probabilities (probability, pdf, Bayes rule, etc.) \\[.5cm] \pause
		\item Optimality of Bayes classifier and the role of the loss function \\[.5cm] \pause
		\item Discriminative and generative approach to model a posteriori probability
	\end{itemize}
\end{frame}

{
\usebackgroundtemplate{\includegraphics[width=\paperwidth]{png/nextTime.png}}
\begin{frame}[plain]
\end{frame}
}


\subsection{Further Readings}

\begin{frame}
	\frametitle{Further Readings}

	\begin{itemize}
		\item Heinrich Niemann: \\
		      \structure{Pattern Analysis}, \\
		      Springer Series in Information Sciences 4, Springer, Berlin, 1982. \\[.15cm]
		\item Heinrich Niemann: \\
		      \structure{Klassifikation von Mustern}, \\
		      Springer Verlag, Berlin, 1983. \\[.15cm]
		\item Richard O. Duda, Peter E. Hart, David G. Stork: \\
		      \structure{Pattern Classification}, 2nd Edition, \\
		      John Wiley \& Sons, New York, 2000.
	\end{itemize}
\end{frame}

