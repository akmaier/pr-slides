\section{Rosenblatt's Perceptron (1957)}

\subsection{Motivation}

\begin{frame}
  \frametitle{Motivation}

  \begin{itemize}
    \item We want to compute a linear decision boundary. \\[.5cm]
    \item We assume that classes are linearly separable. \\[.5cm]
    \item Computation of a linear separating hyperplane that\\
      minimizes the distance of misclassified feature vectors \\
      to the decision boundary.
  \end{itemize}
\end{frame}


\subsection{Objective Function}

\begin{frame}
  \frametitle{Objective Function}

  Assume the following:

  \begin{itemize}
    \item Class numbers are $y=\pm 1$.
    \item The decision boundary is a linear function:
      \begin{displaymath}
        y^* = \mbox{sgn}(\vec \alpha^T\vec x + \alpha_0).
      \end{displaymath}
      \pause
    \item Parameters $\alpha_0$ and $\vec \alpha$ are chosen according to the optimization problem \\[.2cm]

      \begin{center}
        \tikz[baseline]{ 
          \node[fill=bl1!100,anchor=base,rounded corners=3pt] (d1) {
            \color{bl3} 
            $\displaystyle \mbox{minimize} \quad \bigg\{ D(\alpha_0, \vec \alpha)= -\sum_{\vec x_i\in {\cal M}} y_i \cdot (\vec \alpha^T\vec x_i + \alpha_0) \bigg\}$
          };  
        }\\[.2cm]
      \end{center}

      where $\cal M$ includes the misclassified feature vectors.
  \end{itemize}
\end{frame}


\begin{frame}
  \frametitle{Objective Function \cont}

  \begin{itemize}
    \item The elements of the sum in the objective function depend on \\
      the set of misclassified feature vectors $\cal M$. \\[.5cm] \pause
    \item In each iteration step the cardinality of $\cal M$ might change. \\[.5cm] \pause
    \item The cardinality of $\cal M$ is a discrete variable. \\[.5cm] \pause
    \item \structure{Competing variables:} continuous parameters of linear decision boundary and the discrete cardinality of $\cal M$.
  \end{itemize}
\end{frame}


\subsection{Minimization of Objective Function}

\begin{frame}
  \frametitle{Minimization of Objective Function}

  Remember the objective function $D(\alpha_0, \vec \alpha)$:
  \begin{displaymath}
    \mbox{minimize} \quad \quad D(\alpha_0, \vec \alpha)= -\sum_{\vec x_i\in {\cal M}} y_i \cdot (\vec \alpha^T\vec x_i + \alpha_0)
  \end{displaymath}
  \pause
 
  The gradient of the objective function is: \pause
  \begin{eqnarray*}
    \frac{\partial}{\partial \alpha_0} \ D(\alpha_0, \vec \alpha)&=&  -\sum_{\vec x_i\in {\cal M}} y_i \\ \pause
    \frac{\partial}{\partial \vec\alpha} \ D(\alpha_0, \vec \alpha)&=& -\sum_{\vec x_i\in {\cal M}} y_i\cdot  \vec x_i \\
  \end{eqnarray*}
\end{frame}


\begin{frame}
  \frametitle{Minimization of Objective Function \cont}

  We want to take an update step right after having visited each misclassified observation.
  The update rule in the $(k+1)$-st iteration step is:

  \begin{displaymath}
    { \alpha_0^{(k+1)} \choose \vec \alpha^{(k+1)}} = \pause  { \alpha_0^{(k)} \choose \vec \alpha^{(k)}} + \lambda {y_i \choose y_i\cdot \vec x_i}
  \end{displaymath} 

  Here $\lambda$ is the learning rate which can be set to $1$ without loss of generality.
\end{frame}


\begin{frame}
  \frametitle{Minimization of Objective Function \cont}

  \begin{algorithmic}
    \STATE  \structure{Input:} training data:  $S = \{ (\vec x_1, y_1), (\vec x_2, y_2), (\vec x_3, y_3), \dots, (\vec x_m, y_m) \}$ \pause
    \STATE initialize  $k=0$, $\alpha^{(0)}_0=0$ and $\vec \alpha^{(0)}=  \vec 0$
    \REPEAT 
      \STATE select pair $(\vec x_i, y_i)$ from training set. \pause
      \IF {$y_i\cdot (\vec x_i^T \vec\alpha^{(k)} + \alpha_0^{(k)})\leq 0$} 
        \STATE \
        \STATE $\displaystyle{ \alpha_0^{(k+1)} \choose \vec \alpha^{(k+1)}} =  { \alpha_0^{(k)} \choose \vec \alpha^{(k)}} +  {y_i \choose y_i\cdot \vec x_i}$
        \STATE \
        \STATE $k\leftarrow k+1$
      \ENDIF
      \pause
    \UNTIL{$y_i\cdot (\vec x_i^T \vec\alpha^{(k)} + \alpha_0^{(k)})> 0 \quad \mbox {for all} \quad i$}
    \STATE \structure{Output:} $\alpha_0^{(k)}$ and $\vec \alpha^{(k)}$
  \end{algorithmic}
\end{frame}


\subsection{Remarks on Perceptron Learning}

\begin{frame}
  \frametitle{Remarks on Perceptron Learning}

  \begin{itemize}
    \item The update rule is extremely simple. \\[.3cm]
    \item Nothing happens if we classify all $\vec x_i$ correctly using the given linear decision boundary. \\[.3cm]
    \item The parameter $\vec \alpha$ of the decision boundary is a linear combination of feature vectors. \\[.3cm] \pause
    \item The decision boundary thus is:
    \begin{displaymath}
       F(\vec x) = \left (\sum_{i\in \cal E} y_i\cdot \vec x_i \right)^T \vec x + \sum_{i\in \cal E} y_i \quad 
                 = \quad \sum_{i\in \cal E} y_i\cdot \langle \vec x_i, \vec x\rangle + \sum_{i\in \cal E} y_i 
    \end{displaymath}
    where $\cal E$ is the list of indices that required an update \\
	(indices may appear more than once). 
  \end{itemize}
\end{frame}


\begin{frame}
  \frametitle{Remarks on Perceptron Learning \cont}

  \begin{itemize}
    \item The final linear decision boundary depends on the initialization, \\
      i.\,e.\ $\alpha_0^{(0)}$ and $\vec \alpha^{(0)}$. \\[.5cm]
    \item The number of iterations can be rather large. \\[.5cm]
    \item If data are not linearly separable, the proposed learning algorithm will not converge. The algorithm will end up in hard to detect cycles.
  \end{itemize}
\end{frame}


\subsection{Convergence of Learning Algorithm}

\begin{frame}
  \frametitle{Convergence of Learning Algorithm}

  \begin{theorem}[Convergence Theorem of Rosenblatt and Novikoff]
    Assume that for all $i=1,2, \dots, m$
    \begin{displaymath}
      y_i(\vec x_i^T\vec \alpha^* +\alpha^*_0)\geq \rho
    \end{displaymath}
    where $\rho>0$ and $\|\vec \alpha^*\|=1$.
    Let $M= \max_i \|\vec x_i\|_2$. \\[.3cm]
    
    The perceptron learning algorithm converges to a linear decision boundary after $k$ iterations, where $k$ is bounded by
    \begin{displaymath}
      k \leq \frac{(\alpha_0^{*2}+1)(1+M^2)}{\rho^2}.
    \end{displaymath}
  \end{theorem}
\end{frame}


\begin{frame}
  \frametitle{Convergence of Learning Algorithm \cont}

  Let us look at the estimated parameters after $k$ iterations and \
  how the parameters change with iterations:

  \begin{eqnarray*}
    {\alpha_0^{(k)} \choose \vec \alpha^{(k)}}^T{\alpha_0^* \choose \vec \alpha^*} 
      &=& \pause \left( { \alpha_0^{(k-1)} \choose \vec \alpha^{(k-1)}} +  {y_i \choose y_i\cdot \vec x_i}\right)^T {\alpha_0^* \choose \vec \alpha^*}\\ \pause
      &=& { \alpha_0^{(k-1)} \choose \vec \alpha^{(k-1)}} ^T{\alpha_0^* \choose \vec \alpha^*} +  {y_i \choose y_i\cdot \vec x_i}^T {\alpha_0^* \choose \vec \alpha^*} \\ \pause
      &\geq& { \alpha_0^{(k-1)} \choose \vec \alpha^{(k-1)}} ^T{\alpha_0^* \choose \vec \alpha^*} + \rho \\ \pause
      &\geq& k\rho
  \end{eqnarray*}

  \structure{Conclusion:} The more iterations (i.\,e.\ misclassifications) we have, \\
  the more the vectors are aligned.
\end{frame}


\begin{frame}
  \frametitle{Convergence of Learning Algorithm \cont}

  Now we apply Cauchy-Schwartz inequality for inner products:

  \begin{eqnarray*}
    {\alpha_0^{(k)} \choose \vec \alpha^{(k)}}^T{\alpha_0^* \choose \vec \alpha^*} 
      &\leq& \left\|  {\alpha_0^{(k)} \choose \vec \alpha^{(k)}}\right\|_2\cdot  \left\| {\alpha_0^* \choose \vec \alpha^*}\right\|_2 \\[.5cm] \pause
      &=&\left \|  {\alpha_0^{(k)} \choose \vec \alpha^{(k)}}\right\|_2\cdot  \sqrt{\alpha_0^{*2}+1}
  \end{eqnarray*}
\end{frame}


\begin{frame}
  \frametitle{Convergence of Learning Algorithm \cont}

  The norm of the vector estimated in the $k$-th iteration step is:

  \begin{eqnarray*}
    & &  \left \|  {\alpha_0^{(k)} \choose \vec \alpha^{(k)}}\right\|_2^2 =
   \left\| { \alpha_0^{(k-1)} \choose \vec \alpha^{(k-1)}} +  {y_i \choose y_i\cdot \vec x_i}\right\|_2^2 \\[.5cm] \pause
   & & \vspace*{-2cm} = { \alpha_0^{(k-1)} \choose \vec \alpha^{(k-1)}} ^T{ \alpha_0^{(k-1)} \choose \vec \alpha^{(k-1)}} 
           + 2 { \alpha_0^{(k-1)} \choose \vec \alpha^{(k-1)}}^T{y_i \choose y_i\cdot \vec x_i}
          + {y_i \choose y_i\cdot \vec x_i}^T{y_i \choose y_i\cdot \vec x_i}
  \end{eqnarray*}
\end{frame}


\begin{frame}
  \frametitle{Convergence of Learning Algorithm \cont}

  We only go into iteration step $(k+1)$ if we did a mistake in iteration $k$. \\
  A misclassification implies:
  
  \begin{displaymath}
    { \alpha_0^{(k)} \choose \vec \alpha^{(k)}}^T{y_i \choose y_i\cdot \vec x_i} 
    = y_i\cdot (\vec x_i^T \vec\alpha^{(k)} + \alpha_0^{(k)})\quad < \quad 0
  \end{displaymath}
  \pause 

  And thus we get
  
  \begin{displaymath}
    \left \|  {\alpha_0^{(k)} \choose \vec \alpha^{(k)}}\right\|_2^2 
    \leq { \alpha_0^{(k-1)} \choose \vec \alpha^{(k-1)}} ^T{ \alpha_0^{(k-1)} \choose \vec \alpha^{(k-1)}} 
           + {y_i \choose y_i\cdot \vec x_i}^T{y_i \choose y_i\cdot \vec x_i}
    \leq k(1+M^2)
  \end{displaymath}
\end{frame}


\begin{frame}
  \frametitle{Convergence of Learning Algorithm \cont}

  Wrap-up:
  
  \begin{displaymath}
    k\rho 
    \leq {\alpha_0^{(k)} \choose \vec \alpha^{(k)}}^T{\alpha_0^* \choose \vec \alpha^*} 
    \leq \left \|  {\alpha_0^{(k)} \choose \vec \alpha^{(k)}}\right\|_2\cdot  \sqrt{\alpha_0^{*2}+1}
  \end{displaymath}
  \pause

  Using Cauchy-Schwartz:
  
  \begin{displaymath}
    k\rho\quad
    \leq \quad \left \| {\alpha_0^{(k)} \choose \vec \alpha^{(k)}}\right\|_2\cdot  \sqrt{\alpha_0^{*2}+1}\quad \leq \quad  \sqrt{k(1+M^2)(\alpha_0^{*2}+1)}
  \end{displaymath}
  \pause

  shows:
  
  \begin{displaymath}
    k \quad \leq\quad \frac{(\alpha_0^{*2}+1)(1+M^2)}{\rho^2}
  \end{displaymath}
\end{frame}


\subsection{Lessons Learned}

\begin{frame}
  \frametitle{Lessons Learned}

  \begin{itemize}
    \item Objective function changes in each iteration step. \\[.5cm]
    \item Optimization problem is discrete. \\[.5cm]
    \item Very simple learning rule. \\[.5cm]
    \item \vorsicht \structure{Very important:} Number of iterations does \structure{not} depend on the \\
      dimension of the feature vectors.
  \end{itemize}
\end{frame}

{
\usebackgroundtemplate{\includegraphics[width=\paperwidth]{png/nextTime.png}}
\begin{frame}[plain]
\end{frame}
}


\subsection{Further Readings}

\begin{frame}
  \frametitle{Further Readings}

  \begin{itemize}
    \item Brian D. Ripley: \\
      \structure{Pattern Recognition and Neural Networks}, \\
      Cambridge University Press, Cambridge, 1996.\\[.3cm]
    \item T. Hastie, R. Tibshirani, and J. Friedman: \\
      \structure{The Elements of Statistical Learning --}\\
      \structure{ Data Mining, Inference, and Prediction},\\
      2nd edition, Springer, New York, 2009.
  \end{itemize}
\end{frame}


\subsection{Comprehensive Questions}

\begin{frame}
  \frametitle{Comprehensive Questions}

  \begin{itemize}
    \item What is Rosenblatt's perceptron? \\[1cm]
    \item What is the objective function for Rosenblatt's perceptron? \\[1cm]
    \item Why is the optimization of the objective function nonlinear? \\[1cm]
    \item When and how does Rosenblatt's perceptron algorithm converge?
  \end{itemize}
\end{frame}
