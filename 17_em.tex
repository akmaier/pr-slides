\def\p{\vec{\theta}}
\def\hatp{\vec{\hat{\theta}}}
\def\BiasField{\vec{\beta}}

\section{The Expectation Maximization Algorithm}

\subsection{Parameter Estimation}

\begin{frame}
  \frametitle{Parameter Estimation Methods}

  \structure{Goal:} Derivation of a parameter estimation technique that can deal with
  \begin{itemize}
    \item high dimensional parameter spaces and
    \item latent, hidden, incomplete data.
  \end{itemize}
  \pspread
  
  Parameter estimation techniques known from statistics: \\[0.25cm]
  
  \begin{enumerate}
    \item \structure{Maximum likelihood estimation (ML estimation)}
      \begin{itemize}
        \item All observations are assumed to be mutually statistically independent. 
        \item The observations are kept fixed.
        \item The (log-)likelihood function is optimized regarding the parameters. \\[0.25cm] \pause
      \end{itemize}
    \item \structure{Maximum a-posteriori estimation (MAP estimation)}
      \begin{itemize}
        \item The probability density function of the parameters $p(\vec{\theta})$ to be estimated \\
          is known.
      \end{itemize}
  \end{enumerate}
\end{frame}


\begin{frame}
  \frametitle{Parameter Estimation}

  Let $X$ be the observed random variable and $\p$ the parameter set. \\
  The estimates of $\p$ are denoted by $\hatp$. \\
  Let $x$ be an event assigned to the random variable $X$. \\[.25cm] \pause 
%
  \begin{itemize}
    \item \structure{ML estimation:} $\displaystyle \hatp= \argmax_{\p}\ p(x; \p) = \argmax_{\p}\ \log p(x; \p)$  \\[.25cm] \pause 
    \item \structure{MAP estimation:}
      {\small
        \begin{eqnarray*}
          \hatp &=& \argmax_{\p}\ p(\p|{x})\\
                &=& \argmax_{\p}\ \frac{p(\p)\,p(x|\p)}
                                       {\sum_{\p} p(\p)\,p(x|\p)} \\
                &=& \argmax_{\p}\ \log p(\p) + \log p(x|\p)
        \end{eqnarray*}
      }
    \item[] Here $\p$ is considered as a random variable and its probability density function $p(\p)$ is known.
  \end{itemize}
\end{frame}


\begin{frame}
  \frametitle{ML Estimation: Example}

  \begin{ovalblock}{Example}
    Let us assume a Gaussian distributed random vector:
%
    \begin{displaymath}
      p(\vec{x};\vec{\mu}, \mat{\Sigma}) = \frac{1}{\sqrt{\det(2\pi\mat{\Sigma})}}
        e^{-\frac{1}{2}(\vec{x}-\vec{\mu})^T \mat{\Sigma}^{-1} (\vec{x}-\vec{\mu})} \pause
    \end{displaymath}
%    
    \begin{itemize}
      \item We observe the random vectors $\vec{x}_1, \vec{x}_2, \dots, \vec{x}_m$ (training data). \pause
      \item Based on these training data, we have to estimate the mean vector $\vec{\mu}$ and the covariance matrix $\mat{\Sigma}$.
    \end{itemize}
  \end{ovalblock}
\end{frame}


\begin{frame}
  \frametitle{ML Estimation: Example \cont}

  \begin{ovalblock}{Example \cont}
    \small
    The ML estimator assumes \structure{mutually independent observations} and \\
    optimizes the pdf for the given set of training data:
%
    {\footnotesize
      \begin{eqnarray*}
        \{\hat{\vec{\mu}}, \hat{\mat{\Sigma}}\}
          &=& \argmax_{\vec{\mu},\mat{\Sigma}}\ \prod_{i=1}^m p(\vec{x}_i;\vec{\mu}, \mat{\Sigma}) \\ \pause 
          &=& \argmax_{\vec{\mu},\mat{\Sigma}}\ \sum_{i=1}^m \log p(\vec{x}_i;\vec{\mu}, \mat{\Sigma}) \\ \pause
          &=& \argmax_{\vec{\mu},\mat{\Sigma}} L(\vec{x}_1, \vec{x}_2, \dots, \vec{x}_m; \vec{\mu}, \mat{\Sigma}) \pause
      \end{eqnarray*}
    }
%
    where the \structure{log-likelihood function} is defined by
%
    {\footnotesize
      \begin{displaymath}
        L:= L(\vec{x}_1, \vec{x}_2, \dots, \vec{x}_m; \vec{\mu}, \mat{\Sigma}) =
        \sum_{i=1}^m \log p(\vec{x}_i;\vec{\mu}, \mat{\Sigma})
      \end{displaymath}
    }
    \vspace{-.25cm}
  \end{ovalblock}
\end{frame}


\begin{frame}
  \frametitle{ML Estimation: Example \cont}

  \begin{ovalblock}{Example \cont}
     \small
     \structure{Necessary conditions} for the estimation of the parameters are:
%
     {\footnotesize
       \begin{displaymath}
         \frac{\partial L}{\partial \vec{\mu}} \stackrel{!}{=} \vec{0} 
         \quad \quad \mbox{and} \quad \quad
         \frac{\partial L}{\partial \mat{\Sigma}} \stackrel{!}{=} \vec{0} \pause
       \end{displaymath}
     }
%
     Now we get for the mean vector:
%    
     {\footnotesize
       \begin{displaymath}
         \frac{\partial L}{\partial \vec{\mu}} = 
         \sum_{i=1}^m \mat{\Sigma}^{-1}(\vec{x_i}-\vec{\mu}) \stackrel{!}{=} 
         \vec{0} \pause
       \end{displaymath}
     }
%
     and thus the \structure{ML estimate for the mean vector} meets our expectation:
%
     {\footnotesize
       \begin{displaymath}
         \vec{\hat{\mu}} = \frac{1}{m}\sum_{i=1}^m \vec{x}_i
      \end{displaymath}
    }
    \vspace{-.25cm}
  \end{ovalblock}
\end{frame}


\begin{frame}
  \frametitle{ML Estimation: Example \cont}

  \begin{ovalblock}{Example \cont}
    \small
    Along the same lines, we get the \structure{estimator of the covariance matrix} \\
    by computation of the zero crossings of the partial derivatives w.\,r.\,t.\ the components of the covariance matrix:
 
    \begin{displaymath}
      \hat{\mat{\Sigma}} = \frac{1}{m} \sum_{i=1}^m (\vec x_i-\hat{\vec{\mu}})
                                                    (\vec x_i-\hat{\vec{\mu}})^T
    \end{displaymath}
  \end{ovalblock}
\end{frame}


\subsection{Gaussian Mixture Models}

\begin{frame}
  \frametitle{Gaussian Mixture Models}
  
  So far, we have considered parameter estimation for statistical models with: \pause

  \begin{itemize}
    \item one class-dependent distribution component \pause
    \item uni- or multivariate feature vectors \pause
    \item the type was mostly Gaussian (normally distributed features)
  \end{itemize}
  \pspread
  
  Now we extend this model by representing the observations with a set of $K$ multivariate Gaussian distributions:\\

  \begin{center}
    \structure{Gaussian Mixture Model (GMM)}
  \end{center}
\end{frame}


\mode<handout>{
  \begin{frame}
    \frametitle{Gaussian Mixture Models \cont}

    \begin{center}
      \resizebox{.8\linewidth}{!}{
        \input{\texfigdir/koerpergroesse_gesamt.pstex_t}
      }
    \end{center}
  \end{frame}
}


\begin{frame}
  \frametitle{Gaussian Mixture Models \cont}

  \begin{center}
    \resizebox{.8\linewidth}{!}{
     \alt<3->{
       \input{\texfigdir/koerpergroesse_weiblich+maennlich.pstex_t}
     }{\alt<2>{
       \input{\texfigdir/koerpergroesse_weiblich.pstex_t}
     }{
        \input{\texfigdir/koerpergroesse_gesamt.pstex_t}
     }}
    }
  \end{center}
\end{frame}


\begin{frame}
  \frametitle{Gaussian Mixture Models \cont}

  \begin{center}
    \resizebox{.8\linewidth}{!}{
     \alt<2->{
       \input{\texfigdir/gmm2.pstex_t}
     }{
       \input{\texfigdir/gmm1.pstex_t}
     }
    }
  \end{center}
\end{frame}


\begin{frame}
  \frametitle{Gaussian Mixture Models \cont}

  \structure{Problem description:} \\[.5cm]
  
  Given $m$ feature vectors in an $d$ dimensional space, find a set of $K$ multivariate Gaussian distributions that best represent the observations. 
  \pspread
  
  GMMs are an example of classification by \structure{\emph{unsupervised learning}}: \pause

  \begin{itemize}
    \item It is not known which feature vectors are generated by which of the $K$ Gaussians \pause
    \item The desired output is, for each feature vector, an estimate of the probability that it is generated by distribution $k$
  \end{itemize}
\end{frame}


\begin{frame}
  \frametitle{Gaussian Mixture Models \cont}

  \structure{GMM parameter estimation:}
%  
  \begin{eqnarray*}
    \vec\mu_k            &\phantom{=}& \mbox{the $K$ means} \\
    \mat\Sigma_k         &\phantom{=}& \mbox{the $K$ covariance matrices of size $d \times d$} \\
    p_k                  &\phantom{=}& \mbox{fraction of all features in component $k$} \\
    p(k|i) \equiv p_{ik} &\phantom{=}& \mbox{the $K$ probabilities for each of the $m$ feature vectors $\vec{x}_i$}
  \end{eqnarray*}  
  \pspread

  \structure{Additional estimates:}
%  
  \begin{eqnarray*}
    p(\vec x) &\phantom{=}& \mbox{probability distribution of observing a feature vector $\vec x$} \\
    L         &\phantom{=}& \mbox{overall log-likelihood function of the estimated parameter set}
  \end{eqnarray*}
\end{frame}


\begin{frame}
  \frametitle{GMM -- Expectation}
  
  The key to the estimation problem is the \structure{overall log-likelihood objective function $L$}:
  \begin{displaymath}
    L = \sum_{i = 1}^{m} \log p(\vec x_i) \pause
  \end{displaymath}
  
  Split $p(\vec x_i)$ % (mixture weight of $\vec x_i$) 
  into its contributions from the $K$ Gaussians:
  \begin{displaymath}
    p(\vec x_i) = \sum_{k=1}^{K} p_k \, {\mathcal N}(\vec x_i | \vec \mu_k, \mat \Sigma_k) \pause
  \end{displaymath}

  Individual probabilities for the $K$ contributions:
  \begin{displaymath}
    p_{ik} \equiv p(k|i) = \frac{ p_k\,{\mathcal N}(\vec x_i | \vec \mu_k, \mat \Sigma_k)}{p(\vec x_i)}
  \end{displaymath}
\end{frame}


\begin{frame}
  \frametitle{GMM -- Maximization}
  
  \structure{Problem:} How do we get $\vec \mu_k, \mat \Sigma_k$ and $p_k$? \pause

  \begin{itemize}
    \item Similar to the ML estimate for the Gaussian, we maximize the log-likelihood by deriving w.\,r.\,t.\ the unknowns. \\[.25cm] \pause
    \item The \structure{ML estimates} are:
      \begin{eqnarray*}
        \hat{\vec \mu}_k 
          &=& \frac{\sum_i p_{ik} \vec x_i}{\sum_i p_{ik}} \\
        \hat{\mat \Sigma}_k 
          &=& \frac{\sum_i p_{ik} (\vec x_i - \hat{\vec \mu}_k)(\vec x_i - \hat{\vec \mu}_k)^T}{\sum_i p_{ik}} \\
        \hat{p}_k 
          &=& \frac{1}{m} \sum_{i=1}^m p_{ik}
      \end{eqnarray*}
  \end{itemize}
\end{frame}


\begin{frame}
  \frametitle{GMM Parameter Estimation}

  \structure{Observations:}

  \begin{itemize}
    \item If we know the values for the parameters ($\vec \mu_k, \mat \Sigma_k, p_k$), we can compute the expectations (\structure{E-step}).\pause 
    \item Once we have the expectations we can compute improved values for the parameters (\structure{M-step}).
  \end{itemize}
  \pspread

  We have found an \structure{iterative solution scheme} for the nonlinear GMM parameter estimation problem:

  \begin{itemize}
    \item \emph{Right at} the ML solution both E- and M-step relations hold. \pause 
    \item The ML parameters are a stationary point for the E- and M-step. \pause 
    \item Starting from any parameter values, an iteration of the E-step combined with an M-step will increase $L$
  \end{itemize}
\end{frame}


\begin{frame}
  \frametitle{GMM Parameter Estimation \cont}
  
  \structure{EM algorithm for GMM parameter estimation:}
%
  \begin{centernss}
    \begin{struktogramm}(100,50)
      \assign{Initialization: $\vec \mu_k^{(0)}, \mat \Sigma_k^{(0)}, p_k^{(0)}$}
      \assign{$j \gets 0$}
      \until{$L$ is no longer changing}
        \assign{\structure{Expectation step}: \\
          compute new values for $p_{ik}, L$ }
        \assign{\structure{Maximization step}: \\
          update values for $\vec \mu_k^{(j)}, \mat \Sigma_k^{(j)}, p_k^{(j)}$ }
        \assign{$j \gets j+1$}
      \untilend
      \assign{Output: estimates $\hat{\vec \mu}_k, \hat{\mat \Sigma}_k, \hat{p}_k$}
    \end{struktogramm}
  \end{centernss}
%
%   \begin{center}
%     \begin{struktogramm}{10cm}{0.7cm}
%       \BLOCK {Initialization: $\vec \mu_k^{(0)}, \mat \Sigma_k^{(0)}, p_k^{(0)}$}
%       \BLOCK {Set $j:=0$}
%       \REPEAT {
%         \BLOCK {\structure{Expectation Step}: \\
%           compute new values for $p_{ik}, L$ }
%         \BLOCK {\structure{Maximization Step}: \\
%           update values for $\vec \mu_k^{(j)}, \mat \Sigma_k^{(j)}, p_k^{(j)}$ }
%         \BLOCK {Set $j:=j+1$}
%       }
%       UNTIL{$L$ is no longer changing}
%       \BLOCK {Output: estimates $\hat{\vec \mu}_k, \hat{\mat \Sigma}_k, \hat{p}_k$}
%     \end{struktogramm}
%   \end{center}
\end{frame}


{
\usebackgroundtemplate{\includegraphics[width=\paperwidth]{png/nextTime.png}}
\begin{frame}[plain]
\end{frame}
}


\subsection{Missing Information Principle}

\begin{frame}
  \frametitle{Missing Information Principle}

  A colloquial formulation of the \structure{missing information principle (MIP)} is \\
  as simple as:

  \begin{center}
    \tikz[baseline]{
      \node[fill=bl1!100, anchor=base, rounded corners=3pt, inner sep=3mm] (d1) {
        \color{bl3}
        observable information $=$ complete information $-$ hidden information
      };
    }
  \end{center}
\end{frame}


\begin{frame}
  \frametitle{Missing Information Principle \cont}

  \structure{Mathematical formalization of the MIP:}

  \begin{itemize}
    \item observable random variable: $X$
    \item hidden random variable: $Y$
    \item parameter set: $\p$ \\[0.25cm]
  \end{itemize}
  \pspread
  
  The joint probability density of the events $x$ (observation) and $y$ (hidden) is:
  \begin{displaymath}
    p(x,y; \p) = p(x; \p)\  p(y | x; \p) \pause
  \end{displaymath}
%
  and thus:
  \begin{displaymath}
    p(x; \p)= \frac{p(x, y; \p)}{p(y|x; \p)}
  \end{displaymath}
  \pause
  
  \vspace*{0.25cm}
  The mathematical formulation of the MIP is:
  \begin{displaymath}
    -\log p(x; \p) = -\log p(x, y; \p) -(-\log p(y|x; \p))
  \end{displaymath}
\end{frame}


\begin{frame}
  \frametitle{Key Equation}

  We now consider the mathematical formulation of the key equation and derive an iterative parameter estimation scheme: \pause

  \begin{itemize}
   \item Let $i$ denote the iteration parameter. \pause 
   \item Consider the key equation $(i+1)$-st iteration
     {\small
       \begin{displaymath}
         \log p\left(x    ; \hatp^{(i+1)} \right) =
         \log p\left(x, y ; \hatp^{(i+1)} \right) -
         \log p\left(y|x  ; \hatp^{(i+1)} \right)~,
       \end{displaymath}
     }
     where $\hatp^{(i+1)}$ denotes the estimation in iteration step $(i+1)$. \pause
   \item Now we multiply both sides with $p\left( y|x ; \hatp^{(i)} \right)$ and \\
     integrate over the hidden event $y$:
     \footnotesize
     \begin{eqnarray*}
             \int p\left(y|x ; \hatp^{(i)} \right) \log p\left( x ;   \hatp^{(i+1)} \right)\, \mathsf{d}y \pause
         &=& \int p\left(y|x ; \hatp^{(i)} \right) \log p\left(x, y ; \hatp^{(i+1)} \right)\, \mathsf{d}y -{} \\
         & & \int p\left(y|x ; \hatp^{(i)} \right) \log p\left(y|x  ; \hatp^{(i+1)} \right)\, \mathsf{d}y
     \end{eqnarray*}
  \end{itemize}
\end{frame}


\begin{frame}
  \frametitle{Key Equation \cont}
  
  Now consider the left hand side of this equation:
  
  \begin{eqnarray*}
          & & \int p\left(y|x ; \hatp^{(i)}   \right) \, \log p\left( x ; \hatp^{(i+1)} \right)\ \mathsf{d}y ={} \\ \pause
          &=& \log p\left(x   ; \hatp^{(i+1)} \right) \, \int p\left( y|x ; \hatp^{(i)} \right)\ \mathsf{d}y = \\ \pause
          &=& \log p\left(x   ; \hatp^{(i+1)} \right) \pause
  \end{eqnarray*}
  
  \begin{itemize}
    \item \structure{Observation:} The left side of the key equation is the log likelihood function of observations. \\[.3cm] \pause 
    \item \structure{Conclusion:} The maximization of the right hand side of the above key equation corresponds to a ML estimation
  \end{itemize}
\end{frame}


\begin{frame}
  \frametitle{Kullback-Leibler Statistics and Entropy}
  
  For the terms on the right hand side we introduce the following notation \\
  (formally this is incorrect due to the differences in the iteration index):
  
  \begin{itemize}
    \item \structure{Kullback-Leibler Statistics}
      \begin{displaymath}
        Q(\hatp^{(i)};\hatp^{(i+1)}) =
        \int p(y|x; \hatp^{(i)}) \log p(x, y;\hatp^{(i+1)}) \ \mathsf{d}y \pause
      \end{displaymath}
   \item \structure{Entropy:}
     \begin{displaymath}
       H(\hatp^{(i)};\hatp^{(i+1)}) = 
       - \int p(y|x; \hatp^{(i)}) \log p(y|x; \hatp^{(i+1)})\ \mathsf{d}y
      \end{displaymath}
  \end{itemize}
\end{frame}


\begin{frame}
  \frametitle{Kullback-Leibler Statistics}

  Let us first take a closer look at the Kullback-Leibler statistics:
  \begin{displaymath}
    Q(\p,\p') = \int p(y| x; \p)\ \log{p(x,y;\p')}\ \mathsf{d}y
  \end{displaymath}

  The Kullback-Leibler statistics (also called $Q$-function) w.\,r.\,t.\ $\p'$ given $\p$ is the \structure{conditional expectation}:
  \begin{displaymath}
    E[\log p(x,y;\p') \ | \ x, \p] = 
    \int p(y| x; \p)\ \log{p(x,y;\p')}\ \mathsf{d}y
  \end{displaymath}
\end{frame}


\iffalse
\begin{frame}
 \frametitle{Kullback-Leibler Statistics}
 Using the argument that the position of a maximum of a function does
 not change by scaling the function value with a positive
 factor and by translation, we see that for give $X$ and $\p$::
 \begin{eqnarray*}
   \hat{\p}' &=& \argmax_{\p'} \int p(Y| X; \p)\ \log{p(X,Y;\p')} \ dy\\
            &=& \argmax_{\p'} \frac{1}{p(X;\p)}
                                 \int p(X, Y; \p)\ \log
                                 {p(X,Y;\p')} \ dy\\
            &=& \argmax_{\p'} \int p(X, Y; \p)\ \log{p(X,Y;\p')} \ dy\\
            &=& \argmax_{\p'} -\int p(X, Y; \p)\ \log{p(X,Y;\p')}\ dy
                                 +\int p(X, Y; \p)\ \log{p(X,Y;\p)}\ dy\\
            &=& \argmax_{\p'} \int p(X, Y; \p)\
                \log\frac{p(X,Y;\p')}{p(X,Y;\p)} \ dy
 \end{eqnarray*}
\end{frame}
\fi


\begin{frame}
  \frametitle{Key Equation}
  
  The \structure{key equation} of the Expectation Maximization algorithm (EM algorithm) can be rewritten:

  \begin{center}
    \tikz[baseline]{
      \node[fill=bl1!100, anchor=base, rounded corners=3pt, inner sep=3mm] (d1) {
        \color{bl3}
        $\log p\left( x; \hatp^{(i+1)} \right) = 
        Q\left( \hatp^{(i)}; \hatp^{(i+1)} \right) + 
        H\left( \hatp^{(i)}; \hatp^{(i+1)} \right)$
      };
    }
  \end{center}
  \spread

  \begin{itemize}
    \item Below we will motivate that the maximization of the Kullback-Leibler statistics can replace the optimization of the log-likelihood function. \\[.5cm]
    \item A complete proof can be found in the literature (see Further Readings).
  \end{itemize}
\end{frame}


\begin{frame}
  \frametitle{Entropy Changes with Iterations}
 
  For the entropy we get the inequality:
  \begin{displaymath}
    H(\p;\p') \geq H(\p; \p) \pause
  \end{displaymath}

  This is shown rather straightforward:

  \begin{eqnarray*}
    & & \hspace{-1cm}H({\p};{\p'}) - H(\p;{\p}) \\ \pause 
    &=& - \int p(y|x;\p) \, \log p(y|x;{\p}')\ \mathsf{d}y
        + \int p(y|x;\p) \, \log p(y|x;{\p})\ \mathsf{d}y \\ \pause 
    &=& - \int p(y|x;\p) \, \log \frac{p(y|x;{\p}')}{p(y|x;{\p})}\ \mathsf{d}y \\ \pause 
    &=&   \int p(y|x;\p) \, \log \frac{p(y|x;{\p})}{p(y|x;{\p'})}\ \mathsf{d}y 
  \end{eqnarray*}
\end{frame}


\begin{frame}
  \frametitle{Entropy Changes with Iterations \cont}
  
  The difference of the considered entropies

  \begin{eqnarray*}
    & & \hspace{-1cm}H({\p};{\p'})-H(\p;{\p}) = \\
    &=& \int p(y|x;{\p}) \, \log \frac{p(y|x;{\p})}{p(y|x;{\p'})}\ \mathsf{d}y \ge 0
  \end{eqnarray*}

  is thus the Kullback-Leibler divergence of the pdf's $p(y|x;{\p})$ and $p(y|x;{\p'})$, \\
  and the Kullback-Leibler divergence is known to be non-negative.
\end{frame}


\begin{frame}
  \frametitle{Entropy Changes with Iterations \cont}

  The best to see this is to make use of the inequality
  \begin{displaymath}
  \log(x) \leq x-1
  \end{displaymath}

  and conclude:
  \begin{eqnarray*}
    \int p(x) \, \log \frac{p(x)}{q(x)}\ \mathsf{d}x 
      & =  & -\int p(x) \, \log \frac{q(x)}{p(x)}\ \mathsf{d}x \\ \pause
      &\geq& \int p(x) \left( 1 - \frac{q(x)}{p(x)} \right) \, \mathsf{d}x\\ \pause 
      & =  & 1-1 = 0
  \end{eqnarray*}
\end{frame}


\subsection{Expectation Maximization Algorithm}

\begin{frame}
  \frametitle{Expectation Maximization Algorithm}

  \structure{The basic idea of the EM algorithm:} \\[0.5cm]

  Instead of maximizing the log-likelihood function on the left hand side \\
  of the key-equation, we maximize the Kullback-Leibler statistics iteratively \\
  while ignoring the entropy term.
\end{frame}


\begin{frame}
  \frametitle{Expectation Maximization Algorithm \cont}

  \begin{centernss}
    \resizebox{.85\linewidth}{!}{
      \begin{struktogramm}(100,50)
        \assign{Initialization: $\hatp^{(0)}$}
        \assign{$i \gets -1$}
        \until{$\hatp^{(i+1)}$ $=$ $\hatp^{(i)}$}
          \assign{$i \gets i+1$}
          \assign{ \structure{Expectation step:} \\[.25cm]
            \centerline{$Q\left( \hatp^{(i)}\, ; \,\p \right) := \int p\left( y|x; \hatp^{(i)} \right) \log p(x, y;{\p} )\ \mathsf{d}y$} } 
          \assign{ \structure{Maximization step:} \\[.25cm]
            \centerline{$\displaystyle {\hatp}^{(i+1)} \gets \argmax_{{\p}} Q\left( \hatp^{(i)}\,;\, \p \right)$} }
        \untilend
        \assign {Output: estimate $\hatp \gets \hatp^{(i)}$}
      \end{struktogramm}
    }
  \end{centernss}
%
%   \begin{center}
%     \begin{struktogramm}{10cm}{0.7cm}
%       \BLOCK {Initialization: $\hatp^{(0)}$}
%       \BLOCK {Set $i:=-1$}
%       \REPEAT {
%         \BLOCK { Set $i:= i+1$ }
%         \BLOCK { \structure{Expectation Step:} \\
%           $Q\left( \hatp^{(i)}\, ; \,\p \right) := \int p\left( y|x; \hatp^{(i)} \right) \log p(x, y;{\p} )\ \mathsf{d}y$ } 
%         \BLOCK { \structure{Maximization Step:} \\
%           $\displaystyle {\hatp}^{(i+1)} := \argmax_{{\p}} Q\left( \hatp^{(i)}\,;\, \p \right)$ }
%       }
%       UNTIL {$\hatp^{(i+1)}$ $=$ $\hatp^{(i)}$}
%       \BLOCK {Output: estimate $\hatp:= \hatp^{(i)}$}
%     \end{struktogramm}
%   \end{center}
\end{frame}


\subsection{Advantages of the EM Algorithm}

\begin{frame}
  \frametitle{Advantages of the EM Algorithm}

  A few \structure{practical positive aspects} regarding the EM algorithm:

  \begin{itemize}
    \item The maximum of the KL statistics is usually computed using zero crossings of the gradient. \\[.3cm] \pause 
    \item Mostly we find \structure{closed form iteration schemes}. \\[.3cm] \pause 
    \item Easy to implement closed form iteration formulas (if these exist). \\[.3cm] \pause 
    \item Iteration scheme is numerically robust. \\[.3cm] \pause 
    \item Closed form iterations have constant memory requirements. \\[.3cm] \pause 
    \item If the argument in the logarithm can be factorized properly, \\
      we observe a decomposition of the parameter space \\
      (independent lower dimensional sub-spaces)
  \end{itemize}
\end{frame}


\subsection{Drawbacks of EM}

\begin{frame}
  \frametitle{Drawbacks of EM}

  The EM algorithm has a few \structure{major drawbacks}: \pause

  \begin{itemize}
    \item slow, slow, slow convergence \\
      (should not be used in run time critical applications) \pause 
    \item local optimization method, i.\,e.\ the initialization $\hatp^{(0)}$ has to lie in the area of attraction of the global maximum.
  \end{itemize}

  \begin{center}
    \resizebox{.6\linewidth}{!}{
      \input{\texfigdir/em_local.pstex_t}
    }
  \end{center}
\end{frame}


\subsection{Constrained Optimization}

\begin{frame}
  \frametitle{Constrained Optimization}

  Many optimization problems in the context of the EM algorithm are \\
  of the following form: \\[.25cm]

  \begin{ovalblock}{Example}
    \small
    Optimize the multivariate function
    {\footnotesize
      \begin{displaymath}
        f_0(p_1,p_2,\dots,p_K) = \sum_{k=1}^{K} a_k \ \log p_k
      \end{displaymath}
    }
%    
    subject to
    {\footnotesize
      \begin{eqnarray*}
        \sum_{k=1}^K p_k & =    & 1 \\
        p_k              & \geq & 0
      \end{eqnarray*}
    }
    \vspace{-.5cm}
  \end{ovalblock}
\end{frame}


\begin{frame}
  \frametitle{Constrained Optimization \cont}
  
  \begin{ovalblock}{Example}
    \small
    Application of the \structure{Lagrange multiplier} method:
    {\footnotesize
      \begin{displaymath}
        L(p_1,p_2,\dots,p_K) = 
        \sum_{k=1}^K a_k\ \log p_k + \nu \left( \sum_{k=1}^K p_k-1 \right) \pause
      \end{displaymath}
    }
    
    The optimization can be done using the \structure{partial derivative}:
    {\footnotesize
      \begin{displaymath}
        \frac{\partial L(p_1,p_2,\dots,p_K)}{\partial p_k} =
        \frac{a_k}{p_k} + \nu \stackrel{!}{=} 0 \quad .
      \end{displaymath}
    }
    \vspace{-.25cm}
  \end{ovalblock}
\end{frame}


\begin{frame}
  \frametitle{Constrained Optimization \cont}
  
  \begin{ovalblock}{Example \cont}
    \small
    The \structure{Lagrange multiplier} is:
    {\footnotesize
    \begin{displaymath}
      a_k= -\nu p_k~. \pause
    \end{displaymath}
    }
 
    Due to the fact that the $p_k$'s are unknown, we have to apply a trick to get $\nu$. \\
    We just sum both sides of the above equation over all $k$ and get:
    {\footnotesize
      \begin{displaymath}
        \nu = -\sum_{k=1}^K a_k~. \pause
      \end{displaymath}
    }

    The estimator for $p_k$ now is:
    {\footnotesize
      \begin{displaymath}
        \hat{p}_k = \frac{a_k}{\sum_{l=1}^K a_l}
      \end{displaymath}
    }
  \end{ovalblock}
\end{frame}


\begin{frame}
  \frametitle{EM Algorithm: Example}

  \begin{ovalblock}{Example}
    \small
    Estimate the priors $p_k$ of classes $k=1,2, \dots, K$ from the observation $x$ \\
    where the probability density function of observations is given by the marginal over all classes:
    {\footnotesize
    \begin{displaymath}
      p(x;\BiasField) = \sum_{k=1}^K p_k\,p(x|k; \BiasField) \pause
    \end{displaymath}
    }

    \structure{Application of the EM scheme:}
    \begin{itemize}
      \item observable random measurement: $x$
      \item hidden random measurement: $k$
      \item parameter set: $\p = \{p_k; k=1,\dots, K\}$
    \end{itemize}
  \end{ovalblock}
\end{frame}


\begin{frame}
  \frametitle{EM Algorithm: Example \cont}

  \begin{ovalblock}{Example}
    \small
    For illustration purposes let us consider three classes.
    If events, in this case 2-D points, are labeled by colors representing different classes, the priors are easily estimated by relative frequencies.

    \begin{center}
      \resizebox{.45\linewidth}{!}{
        \input{\texfigdir/gauss_em1.pstex_t}
      }
    \end{center}
  \end{ovalblock}
\end{frame}


\begin{frame}
  \frametitle{EM Algorithm: Example \cont}

  \begin{ovalblock}{Example \cont}
    \small
    The problem appears quite difficult, if the class (color) labels are missing.

    \begin{center}
      \resizebox{.45\linewidth}{!}{
        \input{\texfigdir/gauss_em2.pstex_t}
      }
    \end{center}
  \end{ovalblock}
\end{frame}


\begin{frame}
  \frametitle{EM Algorithm: Example \cont}

  \begin{ovalblock}{Example}
    \small
    The \structure{Kullback-Leibler statistics} results in:
    {\footnotesize
      \begin{eqnarray*}
        Q\left( \hatp^{(i)}; \hatp^{(i+1)} \right) 
          &=& \sum_{k=1}^{K} a_k \log \left( \hat{p}_k^{(i+1)} \, p(x|k;\BiasField) \right) \\ \pause 
          &=& \sum_{k=1}^{K} a_k \left( \log \hat{p}_k^{(i+1)} + \log p(x|k; \BiasField) \right) \\ \pause 
          &=& \sum_{k=1}^{K} a_k \log \hat{p}_k^{(i+1)} +
              \sum_{k=1}^{K} a_k \log p(x|k;\BiasField)
      \end{eqnarray*}
    }
    where
    {\footnotesize
      \begin{eqnarray*}
        a_k
          &=& \frac{\hat{p}_k^{(i)} \, p(x|k; \BiasField)}
                   {\sum_{j}\hat{p}_j^{(i)} \, p(x|j; \BiasField)}
      \end{eqnarray*} \vspace{-.2cm}
    }
  \end{ovalblock}
\end{frame}


\begin{frame}
  \frametitle{EM Algorithm: Example \cont}

  \begin{ovalblock}{Example \cont}
    \small
    Now we compute the gradient with respect to $\hat{p}_k^{(i+1)}$ and its zero crossing.\\
    The final estimator for priors now is a closed form iteration scheme:
   
   \begin{displaymath}
     \hat{p}_k^{(i+1)} = 
     \frac{
       \frac{\hat{p}_k^{(i)} \, p(x|k; \BiasField)}
            {\sum_{j}\hat{p}_j^{(i)} \, p(x|j, \BiasField)}
     }{
       \sum_l \frac{\hat{p}_l^{(i)} \, p(x|l; \BiasField)}
                   {\sum_j \hat{p}_j^{(i)} \, p(x|j; \BiasField)}
     } \pause =  
     \frac{\hat{p}_k^{(i)} \, p(x|k; \BiasField)}
          {\sum_j \hat{p}_j^{(i)} \, p(x|j; \BiasField)}
    \end{displaymath}
  \end{ovalblock}
\end{frame}


\begin{frame}
  \frametitle{Initialization of Priors:}

  \begin{itemize}
   \item Use prior medical knowledge about the frequency of tissue classes \\[.5cm]
   \item If no prior information is available, assume uniform distribution
  \end{itemize}
\end{frame}

\iffalse
\fi
\iffalse

\begin{figure}
  \centerline{\psscaleboxto(\linewidth,0){%
   \includegraphics[height=.4\textheight]{mr_wells0.eps}
   \includegraphics[height=.4\textheight]{mr_wells1.eps}
  }}
 \caption[MRI enhancement \cite{IMG-Wells}]
   {Original MRI (left) and the estimated bias field (right)
          using the EM technique
         (Courtesy of W. Wells)}\label{f:mr:enhance}
\end{figure}

\begin{figure}
  \centerline{\psscaleboxto(\linewidth,0){%
   \includegraphics[height=.4\textheight]{mr_wells2.eps}
   \includegraphics[height=.4\textheight]{mr_wells3.eps}
  }}
 \caption[MRI segmentation \cite{IMG-Wells}]
   {Original MRI (left) and the result of segmentation (right)
         (Courtesy of W. Wells)}\label{f:mr:segm}
\end{figure}

\begin{figure}
  \centerline{\psscaleboxto(\linewidth,0){%
   \includegraphics[height=.4\textheight]{mr_wells4.eps}
   \includegraphics[height=.4\textheight]{mr_wells5.eps}
  }}
 \caption{White matter surface computed by conventional and
          adaptive segmentation (Courtesy of W. Wells)}\label{f:ii:struct}
\end{figure}
\fi


\subsection{Lessons Learned}

\begin{frame}
  \frametitle{Lessons Learned}

  \begin{itemize}
    \item Standard parameter estimation method: ML estimation \\[.5cm]
    \item If the prior pdf of the parameters is known: MAP estimation \\[.5cm]
    \item In the presence of latent random variables: EM algorithm \\[.5cm]
    \item EM advantages: decomposition of search space, closed form iteration schemes \\[.5cm]
    \item EM disadvantage: slow convergence, local method
  \end{itemize}
\end{frame}

{
\usebackgroundtemplate{\includegraphics[width=\paperwidth]{png/nextTime.png}}
\begin{frame}[plain]
\end{frame}
}


\subsection{Further Readings}

\begin{frame}
  \frametitle{Further Readings}
   
  \small
  \begin{itemize}
    \item Easy to understand tutorial on ML estimation: \\[.15cm]
      In Jae Myung: \\
      \point\href{http://faculty.psy.ohio-state.edu/myung/personal/mle.pdf}{\structure{Tutorial on maximum likelihood estimation}}, \\
      Journal of Mathematical Psychology, 47(1):90-100, 2003 \\[.25cm]
    \item The classics for an introduction to the EM algorithm is: \\[.15cm]
      A.\,P.\ Dempster, N.\,M.\ Laird, D.\,B.\ Rubin: \\
      \point\href{http://web.mit.edu/6.435/www/Dempster77.pdf}{\structure{Maximum Likelihood Estimation from Incomplete Data via the EM Algorithm}}, \\
      Journal of the Royal Statistical Society, Series B, 39(1):1-38. \\[.25cm]
    \item W.\,H.\ Press, S.\,A.\ Teukolsky, W.\,T.\ Vetterling, B.\,P.\ Flannery: \\
      \point\href{http://www.nr.com/}{\structure{Numerical Recipes}}, \\
      3rd Edition, Cambridge University Press, 2007.
  \end{itemize}
\end{frame}


\subsection{Comprehensive Questions}

\begin{frame}
  \frametitle{Comprehensive Questions}

  \begin{itemize}
    \item What is a Gaussian Mixture Model? \\[0.5cm]
    \item What is the missing information principle? \\[0.5cm]
    \item Write down the key equation for the EM algorithm: \\[0.5cm]
    \item Is the EM algorithm a local or a global parameter estimation method?
  \end{itemize}
\end{frame}
