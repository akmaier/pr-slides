\section{Support Vector Machines I}

\subsection{Motivation}

\begin{frame}
  \frametitle{Motivation}

  \begin{itemize}
  \item Assume two linearly separable classes. \\[.5cm]
  \item Computation of linear decision boundary that allows the separation of \\
    training data and that generalizes well.\\[.5cm]
  \item \structure{Vapnik 1996}: Optimal separating hyperplane separates two classes \\
    and maximizes the distance to the closest point from either class. \\
    This results in
    \begin{itemize}
      \item unique solution for hyperplanes, and
      \item (in most cases) better generalization.
    \end{itemize}
  \end{itemize}
\end{frame}
 
 
\begin{frame}
  \frametitle{Motivation \cont}

  Linearly separable and non-separable classes
  
  \begin{columns}
    \column{.5\linewidth}
      \begin{center}
        \resizebox{.9\linewidth}{!}{
          \input{\texfigdir/linearly_separable.pstex_t}
        }
      \end{center}

    \column{.5\linewidth}
      \begin{center}
        \resizebox{.9\linewidth}{!}{
          \input{\texfigdir/not_linearly_separable.pstex_t}
        }
      \end{center}
  \end{columns}
\end{frame}


\mode<handout>{
\begin{frame}
  \frametitle{Motivation \cont}

  Many, many, many solutions \ldots
  
  \begin{center}
    \resizebox{.45\linewidth}{!}{
      \input{\texfigdir/linear_separation.pstex_t}
    }
  \end{center}

  \vspace{-.3cm}
  \structure{Idea:} Average the perceptron solutions.
\end{frame}
}

\mode<beamer>{
\begin{frame}
  \frametitle{Motivation \cont}

  Many, many, many solutions \ldots
  
  \begin{center}
    \resizebox{.45\linewidth}{!}{
      \alt<3->{
        \input{\texfigdir/linear_separation3.pstex_t}
      }{\alt<2>{
        \input{\texfigdir/linear_separation2.pstex_t}
      }{
        \input{\texfigdir/linear_separation1.pstex_t}
      }}
    }
  \end{center}

  \vspace{-.3cm}
  \structure{Idea:} Average the perceptron solutions.
\end{frame}
}


\begin{frame}
 \frametitle{Motivation \cont}

 We distinguish between: \\[.25cm]

 \begin{columns}
   \column{.5\linewidth}
     \hspace*{0.75cm} \structure{1.\ Hard margin} problem
     \begin{center}
       \resizebox{.9\linewidth}{!}{
         \input{\texfigdir/svm_hard_margin.pstex_t}
       }
     \end{center}
     \pause
   \column{.5\linewidth}
     \hspace*{0.75cm} \structure{2. Soft margin} problem
     \begin{center}
       \resizebox{.9\linewidth}{!}{
         \input{\texfigdir/svm_soft_margin.pstex_t}
       }
     \end{center}
 \end{columns}
\end{frame}


\subsection{Remarks on Linear Algebra}

\begin{frame}
  \frametitle{Remarks on Linear Algebra}

  Assume we have an \structure{affine function} that defines the decision boundary: \pause

  \begin{eqnarray*}
    f(\vec x) &=& \vec \alpha^T \vec x + \alpha_0
  \end{eqnarray*}
  \pause
  
  \begin{itemize}
    \item For any point $\vec x$ on the hyperplane, we have $f(\vec x)= 0$. \\[.5cm] \pause
    \item A necessary condition for two points on the hyperplane is:
      \begin{displaymath}
        f(\vec x_1) = f(\vec x_2) \quad \mbox{and thus} \quad \vec \alpha^T (\vec x_1-\vec x_2) =0.
        \pause
      \end{displaymath}
    \item The \structure{normal vector $\vec n$} of the hyperplane is $\vec n= \vec \alpha/\|\vec \alpha\|_2$.
  \end{itemize}
\end{frame}


\begin{frame}
  \frametitle{Remarks on Linear Algebra \cont}
 
  \begin{itemize}
    \item The signed distance $d$ of a point $\vec x$ to the hyperplane is: \pause
      \begin{displaymath}
        d = 
        \frac{\|\vec \alpha\|_2}{\vec \alpha^T\vec \alpha} \cdot f(\vec x) = 
        \frac{1}{\|\vec \alpha\|_2}\cdot f(\vec x)  =
        \frac{1}{\|\nabla f(\vec x)\|_2}\cdot f(\vec x) \pause
      \end{displaymath}
    \item Assume points $\vec x_1, \vec x_2$ on either side of the margin satisfy $f(\vec x_1)= +1$ and $f(\vec x_2)=-1$. \\[.5cm]
            
      Thus we have:

      \begin{displaymath}
        \vec \alpha^T (\vec x_1 - \vec x_2) = 2 \pause
        \quad \mbox{and} \quad
        \frac{\vec \alpha^T}{\|\vec \alpha\|_2} (\vec x_1-\vec x_2) 
        = \frac{2}{\|\vec \alpha\|_2}
      \end{displaymath}
  \end{itemize}
\end{frame}


\subsection{Hard Margin Problem}

\begin{frame}
  \frametitle{Constrained Optimization Problem}
 
  \structure{Constraints:}
  
  \begin{itemize}
    \item Separation of classes has to be done with margin:
      \begin{eqnarray*}
        \vec \alpha^T \vec x_i+\alpha_0 &\leq& -1, \quad \mbox{if} \quad y_i= -1 \\
        \vec \alpha^T \vec x_i+\alpha_0 &\geq& +1, \quad \mbox{if} \quad y_i= +1 \\ \pause
      \end{eqnarray*}
    \item This is equivalent to:
      \begin{displaymath}
        y_i\cdot (\vec \alpha^T\vec x_i + \alpha_0) \geq 1
      \end{displaymath}
  \end{itemize}
\end{frame}


\begin{frame}
  \frametitle{Constrained Optimization Problem \cont}

  The \structure{maximization} of the margin corresponds to the following optimization problem with linear constraints:

  \begin{center}
    \tikz[baseline]{
      \node[fill=bl1!100,anchor=base,rounded corners=3pt] (d1) {
        \color{bl3}
        $\begin{aligned}
           \displaystyle 
           \mbox{maximize}   &\qquad \frac{1}{\|\vec \alpha\|_2} \\[.2cm]
           \mbox{subject to} &\qquad \forall i:~ y_i\cdot (\vec \alpha^T\vec x_i + \alpha_0) \geq 1
        \end{aligned}$
      };
    }\\[.2cm]
  \end{center}
  \pause

  \structure{Note:}
  
  \begin{itemize}
    \item Linear constraints ensure that all feature vectors have maximum distance to decision boundary. \\[.15cm] \pause
    \item Basically we compute the distance of the convex hulls of feature sets. \\[.15cm] \pause
    \item We need constrained optimization methods to solve the problem.
  \end{itemize}
\end{frame}


\begin{frame}
  \frametitle{Constrained Optimization Problem \cont}

  The optimization problem is equivalent to \\[.75cm]

  \begin{center}
    \tikz[baseline]{
      \node[fill=bl1!100,anchor=base,rounded corners=3pt] (d1) {
        \color{bl3}
        $\begin{aligned}
           \displaystyle 
           \mbox{minimize}   &\qquad \frac{1}{2} {\|\vec \alpha\|^2_2} \\[.2cm]
           \mbox{subject to} &\qquad \forall i:~ y_i\cdot (\vec \alpha^T\vec x_i + \alpha_0) -1 \geq 0
        \end{aligned}$
      };
    }
  \end{center}
\end{frame}


\begin{frame}
  \frametitle{Constrained Optimization Problem \cont}

  \structure{Remarks on the optimization problem:}
  
  \begin{itemize}
    \item Convex optimization problem \\[.3cm]
    \item Efficient algorithms for solving the convex optimization problem \\
      (interior point method) \\[.3cm]
    \item Standard libraries can be used for minimization \\[.3cm]
    \item Solution is unique
  \end{itemize}
\end{frame}


\subsection{Soft Margin Problem}

\begin{frame}
  \frametitle{Non-linearly Separable Classes}

  If classes are not linearly separable, we have to introduce \structure{\emph{slack variables}}. \\[1cm] \pause
  
  \structure{Convex optimization problem:}
  
  \begin{center}
    \tikz[baseline]{
      \node[fill=bl1!100,anchor=base,rounded corners=3pt] (d1) {
        \color{bl3}
        $\begin{aligned}
           \displaystyle 
           \mbox{minimize}    &\qquad \frac{1}{2} {\|\vec \alpha\|^2_2} + \mu \sum_i \xi_i \\[.3cm]
            \mbox{subject to} &\qquad \forall i:~ -(y_i\cdot (\vec \alpha^T\vec x_i + \alpha_0) -1+\xi_i) \leq 0~,\\[.2cm]
                              &\qquad \forall i:~ -\xi_i \leq 0
        \end{aligned}$
      };
    }
  \end{center}
\end{frame}


\subsection{Lessons Learned}

\begin{frame}
  \frametitle{Lessons Learned}

  \begin{itemize}
    \item Support vector machine \\[.5cm]
    \item Hard and soft margin problem \\[.5cm]
    \item Convex optimization 
  \end{itemize}
\end{frame}

{
\usebackgroundtemplate{\includegraphics[width=\paperwidth]{png/nextTime.png}}
\begin{frame}[plain]
\end{frame}
}


\subsection{Further Readings}

\begin{frame}
  \frametitle{Further Readings}

  \begin{itemize}
    \item Bernhard Sch{\"o}lkopf, Alexander J. Smola: \\
      \structure{Learning with Kernels}, \\
      The MIT Press, Cambridge, 2003. \\[.15cm]
    \item Vladimir N. Vapnik: \\
      \structure{The Nature of Statistical Learning Theory}, \\
      Information Science and Statistics, Springer, Heidelberg, 2000. \\[.15cm]
    \item S.~Boyd, L.~Vandenberghe: \\
      \structure{Convex Optimization}, \\
      Cambridge University Press, 2004. \\
      \point{\small \url{http://www.stanford.edu/~boyd/cvxbook/}}
  \end{itemize}
\end{frame}


\subsection{Comprehensive Questions}

\begin{frame}
  \frametitle{Comprehensive Questions}

  \begin{itemize}
    \item What is the concept of a SVM? \\[1cm]
    \item What is the difference between a hard and soft margin SVM? \\[1cm]
    \item What is the convex optimization problem of the hard margin SVM? \\[1cm]
    \item What is the convex optimization problem of the soft margin SVM?
  \end{itemize}
\end{frame}
