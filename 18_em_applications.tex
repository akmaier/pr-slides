\section{Adaptive Segmentation of MRI Data}

\subsection{Introduction}

\begin{frame}
  \frametitle{Introduction}

  \structure{Magnetic Resonance Imaging (MRI)} is an important acquisition technique. \\[.5cm]

  It features:

  \begin{itemize}
    \item high spatial resolution
    \item good soft tissue contrast
    \item does not incorporate ionizing radiation (as computed tomography) \\[.75cm]
  \end{itemize}
  \pause

  Several applications require the \structure{segmentation} (classification) of the acquired images into \structure{tissue types}.
\end{frame}


\begin{frame}
  \frametitle{Introduction \cont}

  Difficulties arise from:

  \begin{itemize}
    \item missing intensity normalization (like Hounsfield units in CT) \pause 
    \item \structure{intensity inhomogeneities}, also known as bias field \\
      (RF coils, acquisition sequences)
  \end{itemize}
  \pspread
  
  \begin{figure}
    \centering
    \subfloat[with bias field]{
      \includegraphics[width=.23\linewidth]{\pngdir/MRI_bias.\png}
    }
    \qquad
    \subfloat[corrected]{
      \includegraphics[width=.23\linewidth]{\pngdir/MRI_nobias.\png}
    }
    \caption{MRI intensity inhomogeneity (Courtesy of F. J{\"a}ger)}
  \end{figure}
\end{frame}


\begin{frame}
  \frametitle{Introduction \cont}
  
  Effect of the bias field on ML segmentation:

  \begin{figure}
    \centering
    \subfloat[]{
      \label{syn-a}
      \includegraphics[width=.22\linewidth]{\pngdir/SyntheticBiased.\png} 
    }
    \onslide<2->
    \subfloat[]{
      \label{syn-b}
      \includegraphics[width=.22\linewidth]{\pngdir/SyntheticBiasField.\png}
    }
    \onslide<3->
    \subfloat[]{
      \label{syn-c}
      \includegraphics[width=.22\linewidth]{\pngdir/SyntheticBadSegmentation.\png}
    }
    \onslide<4>
    \subfloat[]{
      \label{syn-d}
      \includegraphics[width=.22\linewidth]{\pngdir/SyntheticGoodSegmentation.\png}
    }
    \onslide<1->
    \caption{Synthetic image \subref{syn-a} overlaid with artificial bias field \subref{syn-b}, result of ML segmentation \subref{syn-c}, result after modeling bias field within segmentation \subref{syn-d} (Courtesy of W. Wells).}
  \end{figure}
\end{frame}


\begin{frame}
  \frametitle{Introduction \cont}

  W.\,M.\,Wells et al. presented an approach to improve MR brain segmentation (1996):

  \begin{itemize}
    \item statistical approach to intensity-based segmentation of MRI
    \item statistical modeling of bias field (smoothness constraint)
    \item usage of EM algorithm for simultaneous computation of tissue classification and intensity inhomogeneity correction
  \end{itemize}
  \pspread

  \structure{Typical EM problem:}

  \begin{itemize}
    \item The missing data is the tissue class assignment for each pixel. \pause 
    \item If the tissue was classified, the bias field could easily be computed. \pause
    \item If the bias field was known, the tissue classification would be much easier. 
  \end{itemize}
\end{frame}


\subsection{Bias Field Model}

\begin{frame}
  \frametitle{Bias Field Model}

  \begin{itemize}
    \item Let $\tilde{X}_i$ be the (unknown) intensity of the $i$-th voxel of the MRI data and \\
      $B_i$ the corresponding bias field. \pause
    \item The bias field is assumed to be multiplicative:
      \begin{displaymath}
        X_i = \tilde{X}_i \cdot B_i
      \end{displaymath}
      \pause
    \item Using a log-transform on the intensities yields an additive bias field model:
      \begin{displaymath}
        Y_i = \log X_i = \log \tilde{X}_i + \beta_i ~,~\mbox{with}~ \beta_i = \log B_i
      \end{displaymath}
      \pause
    \item The bias field is then:
      \begin{displaymath}
        \vec \beta = \left(\beta_0, \beta_1, \ldots, \beta_{n-1}\right)^T
      \end{displaymath}
      with $n$ being the number of voxels.
  \end{itemize}
\end{frame}


\begin{frame}
  \frametitle{Bias Field Model \cont}

  \begin{itemize}
    \item The bias field is assumed to change smoothly over the entire image domain.
    \item It is modeled by an $n$-dimensional zero mean Gaussian prior:
      \begin{displaymath}
        p(\vec \beta) = {\mathcal N}(\vec{\beta} ; \vec{0}, \mat{\Psi}_{\vec{\beta}})
      \end{displaymath}
  \end{itemize}
  \pspread 
  
  \structure{Notes:}

  \begin{itemize}
    \item $\mat{\Psi}_{\vec{\beta}}$ is a $n \times n$-dimensional covariance matrix \pause
    \item $\mat{\Psi}_{\vec{\beta}}$ is too large to compute directly in practice \pause
    \item Instead of the full covariance matrix, a banded estimate is chosen in practice.
  \end{itemize}
\end{frame}


\subsection{Bayesian Approach}

\begin{frame}
  \frametitle{Bayesian Approach}

  \structure{Variables:}

  \begin{eqnarray*}
    Y_i         &\phantom{=}& \mbox{log-transformed observed intensity at $i$-th voxel} \\
    \Gamma_i    &\phantom{=}& \mbox{tissue class of the $i$-th voxel} \\
    \mu_\Gamma  &\phantom{=}& \mbox{mean intensity for tissue class $\Gamma$} \\
    \psi_\Gamma &\phantom{=}& \mbox{variance of tissue class $\Gamma$}
  \end{eqnarray*}

  The intensities are assumed to be scalar values, therefore: $\mu_\Gamma, \psi_\Gamma \in \real$ 
\end{frame}


\begin{frame}
  \frametitle{Bayesian Approach \cont}

  Assuming \structure{statistical independence} of the intensities, the probability density for the entire image $\vec Y = \left(Y_0, Y_1, \ldots, Y_{n-1} \right)^T$ is: \pause
  
  \begin{displaymath}
    p(\vec Y | \vec \beta) = \prod_i p(Y_i | \beta_i)
  \end{displaymath}
  \pause 
  
  The probability of the observations is modeled as a \structure{Gaussian mixture} over the tissue classes:
  \begin{displaymath}
    p(Y_i | \beta_i) = \pause \sum_{\Gamma} p(Y_i, \Gamma | \beta_i) = \pause 
    \sum_{\Gamma} p(\Gamma) \, p(Y_i | \Gamma, \beta_i) 
  \end{displaymath}
  \pause 
  
  with 
  \begin{displaymath}
    p(Y_i | \Gamma, \beta_i) = {\mathcal N}(Y_i ; \mu_\Gamma + \beta_i, \psi_{\Gamma})
  \end{displaymath}
\end{frame}


\begin{frame}
  \frametitle{Bayesian Approach \cont}
  
  \structure{Observations so far:}

  \begin{itemize}
    \item Each tissue class is modeled with a normal distribution. \\[.25cm] \pause
    \item The modeling of the observed intensity distribution yields a Gaussian mixture model. \\[.25cm] \pause
    \item $p(\Gamma)$ is a stationary prior probability for the tissue class. \\[.25cm] \pause
    \item The estimators for the GMM are non-linear!
  \end{itemize}
\end{frame}


\begin{frame}
  \frametitle{Bayesian Approach \cont}

  Using \structure{Bayes rule} to derive an objective function for the bias field:
  \begin{displaymath}
    p(\vec \beta | \vec Y) = \frac{p(\vec Y | \vec \beta) \, p(\vec \beta)} {p(\vec Y)}
  \end{displaymath}
  \pspread

  Applying the \structure{MAP principle} yields an estimator for the bias field:

  \begin{eqnarray*}
    \hat{\vec \beta} &=& \argmax_{\vec \beta} p(\vec \beta | \vec Y) \\ \pause 
                     &=& \argmax_{\vec \beta} \log p(\vec \beta | \vec Y) \\ \pause 
                     &=& \argmax_{\vec \beta} \big( \log p(\vec Y | \vec \beta) + \log p(\vec \beta) \big)
  \end{eqnarray*}
\end{frame}


\begin{frame}
  \frametitle{Gradient Computation}
  
  At the optimum, the gradient w.\,r.\,t.\ $\vec \beta$ has to be zero:

  \begin{eqnarray*}
    \frac{\partial}{\partial \beta_i} \log p(\vec \beta | \vec Y) 
    &=& \frac{\partial}{\partial \beta_i} \left( \log p(\vec Y | \vec \beta) + \log p(\vec \beta) \right) \\ \pause 
    &=& \frac{\partial}{\partial \beta_i} \left( \sum_j \log p(Y_j | \beta_j) + \log p(\vec \beta) \right) \\ \pause 
    &=& \frac{\frac{\partial}{\partial \beta_i} p(Y_i | \beta_i)}{p(Y_i | \beta_i)} + \frac{\frac{\partial}{\partial \beta_i} p(\vec \beta)}{p(\vec \beta)} \\
    &\stackrel{!}{=}& 0.
  \end{eqnarray*}
\end{frame}


\begin{frame}
  \frametitle{Gradient Computation \cont}

  {\footnotesize
  \begin{eqnarray*}
    \frac{\frac{\partial}{\partial \beta_i} p(Y_i | \beta_i)}{p(Y_i | \beta_i)} 
    &=& \underbrace{
          \frac{\sum_{\Gamma} p(\Gamma) \, \frac{\partial}{\partial \beta_i} {\mathcal N}(Y_i ; \mu_\Gamma + \beta_i, \psi_{\Gamma}) }
               {\sum_{\Gamma} p(\Gamma) \, {\mathcal N}(Y_i ; \mu_\Gamma + \beta_i, \psi_\Gamma)}
        }_{\mbox{\small substitute GMM}} \\[.25cm] \pause
    &=& \frac{\sum_{\Gamma} p(\Gamma) \, {\mathcal N}(Y_i ; \mu_\Gamma + \beta_i, \psi_\Gamma) \, \psi_\Gamma^{-1} \, (Y_i - \mu_\Gamma - \beta_i)}
             {\sum_\Gamma p(\Gamma) \, {\mathcal N}(Y_i ; \mu_\Gamma + \beta_i, \psi_\Gamma) } \\[.25cm] \pause 
    &=& \sum_\Gamma W_{i\Gamma} \left( \psi_\Gamma^{-1} (Y_i - \mu_\Gamma - \beta_i) \right)
  \end{eqnarray*}
  }

  Weight for the $i$-th voxel and tissue class $\Gamma$:
  {\footnotesize
  \begin{eqnarray*}
    W_{i\Gamma} := 
      \frac{p(\Gamma) \cdot \mathcal{N}(Y_i ; \mu_\Gamma + \beta_i, \psi_\Gamma) }
           {\sum_{\Gamma} p(\Gamma) \cdot \mathcal{N}(Y_i ; \mu_\Gamma + \beta_i, \psi_\Gamma)} 
  \end{eqnarray*}
  }
\end{frame}


\begin{frame}
  \frametitle{Gradient Computation \cont}

  Rewriting the equation:

  {\footnotesize
  \begin{eqnarray*}
    \frac{\frac{\partial}{\partial \beta_i} p(Y_i | \beta_i)}
         {p(Y_i | \beta_i)} 
    &=& \sum_\Gamma W_{i\Gamma} \left( \psi_\Gamma^{-1} (Y_i - \mu_\Gamma - \beta_i) \right) \\ \pause 
    &=& \sum_\Gamma W_{i\Gamma} \, \psi_j^{-1} \, (Y_i - \mu_\Gamma) - \sum_\Gamma W_{i\Gamma} \, \psi_\Gamma^{-1} \, \beta_i \\ \pause 
    &=& \overline{R_i} - \overline{\psi^{-1}}_i ~\beta_i
  \end{eqnarray*}
  }
  
  Mean residual:
  {\footnotesize
  \begin{displaymath}
    \overline{R_i} := \sum_\Gamma W_{i\Gamma} \, \psi_\Gamma^{-1} \, (Y_i - \mu_\Gamma)
  \end{displaymath}
  }
  
  Mean inverse variance:
  {\footnotesize
  \begin{displaymath}
    \overline{\psi^{-1}}_i := \sum_\Gamma W_{i\Gamma} \, \psi_\Gamma^{-1}
  \end{displaymath}
  }
\end{frame}


\begin{frame}
  \frametitle{Gradient Computation \cont}

  Finishing gradient computation:

  {\small
  \begin{eqnarray*}
    \nabla_{\vec \beta} \log p(\vec \beta | \vec Y) 
      &=& \overline{\vec R} - \overline{\mat{\Psi}^{-1}} \vec \beta + \frac{\nabla_{\vec \beta} p(\vec \beta)}{p(\vec \beta)} \\
      &=& \overline{\vec R} - \overline{\mat{\Psi}^{-1}} \vec \beta - \mat{\Psi}_{\vec{\beta}}^{-1} \vec \beta \\
      &\stackrel{!}{=}& \vec{0}
  \end{eqnarray*}
  }
  \pause 
  
  It follows that:
  {\small
  \begin{displaymath}
    \hat{\vec \beta} = \mat{H} \overline{\vec R} \quad \mbox{with}~ 
    \mat{H} \equiv \left[\overline{\mat{\Psi}^{-1}} + \mat{\Psi}_{\vec{\beta}}^{-1} \right]^{-1}
  \end{displaymath}
  }
  \pause 
  
  $\mat{H}$ is a linear operator that is applied to the mean residual field. \\
  In fact, $\hat{\vec{\beta}}$ can be obtained by low pass filtering of $\overline{\vec R}$ and $\overline{\mat{\Psi}^{-1}}$.
\end{frame}


\subsection{EM-Algorithm}

\begin{frame}
  \frametitle{EM-Algorithm}

  \structure{EM-Algorithm for the adaptive segmentation problem:}

  \begin{eqnarray}
    W_{i\Gamma} 
    &\leftarrow& \frac{p(\Gamma) \cdot \mathcal{N}(Y_i | \mu_\Gamma + \beta_i, \psi_\Gamma)}
                      {\sum_\Gamma p(\Gamma) \cdot \mathcal{N} (Y_i | \mu_\Gamma + \beta_i, \psi_\Gamma)} \label{eq:E} \\[.3cm]
    \hat{\vec \beta} 
    &\leftarrow& \mat{H} \overline{\vec R} \label{eq:M}
  \end{eqnarray}
  \pause 
  
  \begin{itemize}
    \item \structure{E-step:} equation \eqref{eq:E} yields the posterior tissue class probabilities for a known bias field \pause 
    \item \structure{M-step:} equation \eqref{eq:M} yields the new bias field for the current estimates for the tissue probabilities \pause
    \item \structure{Result:} iterating 5-10 times between the E- and the M-step is usually sufficient
  \end{itemize}
\end{frame}


\subsection{Results}

\begin{frame}
  \frametitle{Results}

  \begin{figure}
    \centering
    \subfloat[]{
      \label{brain-a}
      \includegraphics[width=.22\linewidth]{\pngdir/BrainInput.\png}
    }
    \subfloat[]{
      \label{brain-b}
      \includegraphics[width=.22\linewidth]{\pngdir/BrainBadSegment.\png}
    }
    \subfloat[]{
      \label{brain-c}
      \includegraphics[width=.22\linewidth]{\pngdir/BrainBiasField.\png}
    }
    \subfloat[]{
      \label{brain-d}
      \includegraphics[width=.22\linewidth]{\pngdir/BrainAdaptiveSegment.\png}
    }
    \caption{\footnotesize Results of conventional segmentation \subref{brain-b} compared to adaptive segmentation \subref{brain-d} with computed bias field \subref{brain-c} on brain image \subref{brain-a}
(Courtesy of W. Wells).}
  \end{figure}
\end{frame}


\begin{frame}
  \frametitle{Results \cont}

  \begin{figure}
    \centering
    \subfloat[]{
      \label{surf-a}
      \includegraphics[width=.3\linewidth]{\pngdir/BrainGrayMatter.\png}
    }
    \subfloat[]{
      \label{surf-b}
      \includegraphics[width=.3\linewidth]{\pngdir/BrainBadWhiteMatter.\png}
    }
    \subfloat[]{
      \label{surf-c}
      \includegraphics[width=.3\linewidth]{\pngdir/BrainAdaptiveWhiteMatter.\png}
    }
    \caption{\footnotesize Gray matter surface \subref{surf-a} for the previous image example, white matter surface of the conventional algorithm \subref{surf-b} and for the adaptive segmentation  \subref{surf-c} (Courtesy of W. Wells).}
  \end{figure}
\end{frame}


\begin{frame}
  \frametitle{Results \cont}

  \begin{figure}
    \centering
    \subfloat[]{
      \label{bias-a}
      \includegraphics[width=.3\linewidth]{\pngdir/MRIBrainBias.\png}
    }
    \subfloat[]{
      \label{bias-b}
      \includegraphics[width=.3\linewidth]{\pngdir/MRIBrainBiasField.\png}
    }
    \subfloat[]{
      \label{bias-c}
      \includegraphics[width=.3\linewidth]{\pngdir/MRIBiasCorrected.\png}
    }
    \caption{\footnotesize MRI image with bias field \subref{bias-a}, computed bias field \subref{bias-b} and image corrected at the brain region \subref{bias-c} (Courtesy of W. Wells).}
  \end{figure}
\end{frame}


\subsection{Model Extensions}

\begin{frame}
  \frametitle{Model Extensions}

  \structure{Drawbacks} of initial adaptive segmentation algorithm:

  \begin{itemize}
    \item brains should be extracted from entire data \pause
    \item algorithm does not incorporate neighborhood of pixels \pause
    \item purely intensity-based model
  \end{itemize}
  \pspread

  \structure{Extensions} of the algorithm:

  \begin{itemize}
    \item incorporation of atlases for spatial probability maps of tissue classes \pause
    \item definition of vector space for probabilistic atlases to get shape models \pause
    \item voxel neighborhood relations modeled by Markov random fields \pause
    \item incorporation into Bayesian model that is solved by EM approach
  \end{itemize}
\end{frame}


\begin{frame}
  \frametitle{Model Extensions \cont}

  Result using an extended model:

  \begin{figure}
    \includegraphics[width=.8\linewidth]{\pngdir/MRIBrainAtlasSegmentation.\png}
    \caption{\footnotesize MRI segmentation of the thalamus and caudate using an atlas-based EM segmentation algorithm (Courtesy of K. Pohl).}
  \end{figure}
\end{frame}


\subsection{Lessons Learned}

\begin{frame}
  \frametitle{Lessons Learned}
 
  \begin{itemize}
    \item Bayesian approach for MRI data segmentation \\[.25cm]
    \item incorporation of bias field estimation \\[.25cm]
    \item nonlinear problem is solved iteratively using EM algorithm \\[.25cm]
    \item improvement of results by incorporating atlases
  \end{itemize}
\end{frame}

{
\usebackgroundtemplate{\includegraphics[width=\paperwidth]{png/nextTime.png}}
\begin{frame}[plain]
\end{frame}
}


\subsection{Further Readings}

\begin{frame}
  \frametitle{Further Readings}
 
  \begin{itemize}
    \item Original paper on adaptive MRI segmentation: \\[.15cm]
      W.\,M.\ Wells, R.\,Kikinis, W.\,E.\,L.\ Grimson, F.\ Jolesz: \\
      \structure{Adaptive segmentation of MRI data}, \\
      IEEE Transactions on Medical Imaging, 15:429-442, 1996. \\[.3cm]
    \item F.\ J{\"a}ger, J.\ Hornegger: \\
      \structure{Nonrigid registration of joint histograms for intensity standardization \\ in magnetic resonance imaging}, \\
      IEEE Transactions on Medical Imaging, 28(1):137-150, 2009.
  \end{itemize}
\end{frame}


\begin{frame}
  \frametitle{Further Readings \cont}

  Extensions of the model with shape models, atlas registration and MRFs: \\[.15cm]
  
  \begin{itemize}
    \item K.\,M.\ Pohl, J.\ Fisher, J.\,J.\ Levitt, M.\,E.\ Shenton, R.\ Kikinis, W.\,E.\,L.\ Grimson, W.\,M.\ Wells: \\
      \structure{A Unifying Approach to Registration, Segmentation, and \\ Intensity Correction}, \\
      Proc. MICCAI, pp.\ 310-318, 2005. \\[.3cm]
      % In Proc. MICCAI 2005: Eighth International Conference on Medical Image Computing and Computer Assisted Intervention, Palm Springs, CA, U.\,S.\,A.,
      %Springer-Verlag, Part I, vol. 3749 of Lecture Notes in Computer Science, 
    \item K.\,M.\ Pohl, J.\ Fisher, S.\ Bouix, M.\,E.\ Shenton, R.\,W.\ McCarley, W.\,E.\,L.\ Grimson, R.\ Kikinis, W.\,M.\ Wells: \\
      \structure{Using the logarithm of odds to define a vector space on \\ probabilistic atlases}, \\
      Medical Image Analysis, 11(6), pp.\ 465-477, 2007.
  \end{itemize}
\end{frame}


\subsection{Comprehensive Questions}

\begin{frame}
  \frametitle{Comprehensive Questions}

  \begin{itemize}
    \item What is the idea of combined MR segmentation and \\ 
      bias field correction? \\[.5cm]
    \item What is the E-step in this context? \\[.5cm]
    \item What is the M-step? \\[.5cm]
    \item How can the update formulas be derived?
  \end{itemize}
\end{frame}
