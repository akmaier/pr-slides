\section{Logistic Regression I}

\begin{frame}
  \frametitle{Logistic Regression}

  Logistic Regression is a \structure{discriminative model}, because it models \\
  the posterior probabilities $p(y|\vec{x})$ directly.
\end{frame}


\subsection{Posteriors and the Logistic Function}

\begin{frame}
  \frametitle{Posteriors and the Logistic Function}
 
  For two classes $y\in\{0, 1\}$ we get:
  
  \begin{eqnarray*}
    p(y=0| \vec x) 
      &=& \pause \frac{p(y=0) \cdot p(\vec x | y=0)}{p(\vec x)} \\[.5cm] \pause 
      &=& \frac{p(y=0) \cdot p(\vec x | y=0)}{p(y=0)p(\vec x |y=0)+p(y=1)p(\vec x| y=1)} \\[.5cm] \pause
      &=& \frac{1}{1+\frac{p(y=1)p(\vec x| y=1)}{p(y=0)p(\vec x| y=0)}}
  \end{eqnarray*}
\end{frame}
 
 
\begin{frame}
  \frametitle{Posteriors and the Logistic Function \cont}

  \begin{eqnarray*}
    p(y=0| \vec x) 
      &=& \frac{1}{1+\frac{p(y=1)p(\vec x| y=1)}{p(y=0)p(\vec x| y=0)}} \\[.3cm] \pause
      &\phantom{=}& \mbox{\small (\structure{Trick:} extend with exponential and logarithm)} \\[.3cm] \pause
      &=& \frac{1}{1+e^{\log \frac{p(y=1)p(\vec x| y=1)}{p(y=0)p(\vec x| y=0)} } } \\[.3cm] \pause
      &=& \frac{1}{1+e^{-\log\frac{p(y=0)}{p(y=1)} - \log\frac{p(\vec x| y=0)}{p(\vec x|y=1)}}}\\[0.3cm] \pause
      &=& \frac{1}{1+e^{-\log\frac{p(y=0|\vec x)}{p(y=1|\vec x)}}}
  \end{eqnarray*}
\end{frame}


\begin{frame}
  \frametitle{Posteriors and the Logistic Function \cont}

  We see that the posterior for class $y = 0$ can be written in terms of \\
  a \structure{logistic function}:
%
  \begin{eqnarray*}
    p(y=0| \vec x) &=& \frac{1}{1+e^{-F(\vec{x})}}
  \end{eqnarray*}
  \pause

  And thus the posterior for the other class $y = 1$:
%
  \begin{eqnarray*}
    p(y=1| \vec x) &=& \pause 1- p(y=0| \vec x) \\[.3cm] \pause
                   &=& \frac{e^{-F(\vec x)}}{1+e^{-F(\vec{x})}} \\[.3cm] \pause
                   &=&  \frac{1}{1+e^{F(\vec{x})}}
  \end{eqnarray*}
\end{frame}


\begin{frame}
  \frametitle{Posteriors and the Logistic Function \cont}

  \begin{citeblock}{Definition}

    The {\em logistic function} (also called {\em sigmoid function}) is defined by
    \begin{eqnarray*}
      g(x) &=& \frac{1}{1+e^{-x}}
    \end{eqnarray*}
    where $x\in \real$.
  \end{citeblock}
\end{frame}


\begin{frame}
  \frametitle{Posteriors and the Logistic Function \cont}

  The derivative of the sigmoid function fulfills the nice property:

  \begin{eqnarray*}
    g'(x) &=& \left( \frac{1}{1+e^{-x}} \right)' = \pause \left( (1+e^{-x})^{-1}\right)'= \pause \frac{1}{(1+e^{-x})^2}\cdot e^{-x} \\[.3cm] \pause
          &=& \frac{1}{(1+e^{-x})} \cdot \frac{e^{-x}}{(1+e^{-x})} \\[.3cm] \pause
          &=& \frac{1}{(1+e^{-x})} \cdot \frac{1}{(1+e^{x})} \\[.3cm] \pause
          &=& g(x)g(-x) \\[.3cm] \pause
          &=& g(x)(1-g(x)) \quad.
  \end{eqnarray*}
\end{frame}


\begin{frame}
  \frametitle{Posteriors and the Logistic Function \cont}
 
  \begin{figure}
    \resizebox{.85\linewidth}{!}{
      \alt<4->{
        \input{\texfigdir/sigmoid4.pstex_t}
      }{\alt<3>{
        \input{\texfigdir/sigmoid3.pstex_t}
      }{\alt<2>{
        \input{\texfigdir/sigmoid2.pstex_t}
      }{
        \input{\texfigdir/sigmoid1.pstex_t}
      }}}
    }
    \caption{Sigmoid function: $g(ax)=1/(1+e^{-ax})$ for $a=1,2,3,4$}
    \label{f:sigmoid}
  \end{figure}
\end{frame}


{
\usebackgroundtemplate{\includegraphics[width=\paperwidth]{png/nextTime.png}}
\begin{frame}[plain]
\end{frame}
}


\subsection{Decision Boundary}

\begin{frame}
  \frametitle{Decision Boundary}

  The \structure{decision boundary} $\delta(\vec x)=0$ (zero level set) in feature space separates the two classes. \\[.3cm]
  Points $\vec x$ on the decision boundary satisfy:
 
  \begin{eqnarray*}
    {p(y=0|\vec{x})}&=& {p(y=1|\vec x)}
  \end{eqnarray*}
 
  and thus
 
  \begin{eqnarray*}
    \log \frac{p(y=0|\vec{x})}{p(y=1|\vec x)} &=& \pause \log 1 = 0 \quad.
  \end{eqnarray*}
\end{frame}


\begin{frame}
  \frametitle{Decision Boundary \cont}
 
  \begin{lemma}
    The decision boundary is given by $F(\vec x) = 0$.
  \end{lemma}
  \pspread
 
  \structure{Proof:}
%
  \begin{eqnarray*}
    \log \frac{p(y=0|\vec{x})}{p(y=1|\vec x)} &=& F(\vec x) = 0\\[.3cm] \pause
    \frac{p(y=0|\vec{x})}{p(y=1|\vec x)} &=& e^{F(\vec x)} \\[.3cm] \pause
    {p(y=0|\vec{x})}&=& e^{F(\vec x)} {p(y=1|\vec x)} \\[.3cm]
    % {p(y=0|\vec{x})}&=& e^{F(\vec x)}\left( {1-p(y=0|\vec x)}\right)
  \end{eqnarray*} 
\end{frame}


\begin{frame}
  \frametitle{Decision Boundary \cont}
 
  Now we use that the posteriors sum up to one:
  
  \begin{eqnarray*}
    {p(y=0|\vec{x})}&=& e^{F(\vec x)}\left( {1-p(y=0|\vec x)}\right) \\ \\
    \pause {p(y=0|\vec{x})}&=& \frac{e^{F(\vec x)}}{1+e^{F(\vec x)}} \\ \\
    \pause {p(y=0|\vec{x})}&=& \frac{1}{1+e^{-F(\vec x)}} 
  \end{eqnarray*} 
\end{frame}


\begin{frame}
  \frametitle{Decision Boundary \cont}
 
  \begin{figure}
    \resizebox{1\linewidth}{!}{
      \alt<4->{
        \input{\texfigdir/logit3.pstex_t}
      }{\alt<3>{
        \input{\texfigdir/logit2.pstex_t}
      }{\alt<2>{
        \input{\texfigdir/logit1.pstex_t}      
      }{
        \input{\texfigdir/logit0.pstex_t}
      }}}
    }
    \caption{Two Gaussians and their posteriors: {\color{bl3} $\sigma_0$}={\color{gr3}$ \sigma_1$}= 0.25,  {\color{bl3} $\mu_0=-2$},
              {\color{gr3} $\mu_1=1$}}
    \label{f:logit}
  \end{figure}
\end{frame}


\begin{frame}
  \frametitle{Decision Boundary \cont}
  
  \begin{ovalblock}{Example}
    \footnotesize
    Let us assume both classes have normally distributed $d$-dimensional feature vectors:
 
    \begin{eqnarray*}
      p(\vec x | y) &=& \frac{1}{ \sqrt{ \det{(2 \pi \mat{\Sigma}_y})} } 
                        e^{-\frac{1}{2} (\vec{x} - \vec{\mu}_y)^T \mat{\Sigma}_y^{-1} (\vec{x} - \vec{\mu}_y)  }
    \end{eqnarray*}
    \pause
 
    Then we can write the posterior of $y=0$ in terms of a logistic function:
 
    \begin{eqnarray*}
      p(y=0|\vec x) &=& \frac{1}{1+e^{-F(\vec x)}} = \frac{1}{1+e^{ - \left( \vec x^T \mat A \vec x + \vec \alpha^T\vec x + \alpha_0 \right) } } \\ \\
      \pause F(\vec x) &=& \log \frac{p(y=0|\vec x)}{p(y=1|\vec x)} = \log \frac{p(y=0)p(\vec x | y=0) }{p(y=1)p(\vec x|y =1)}
    \end{eqnarray*}
  \end{ovalblock}
\end{frame}


\begin{frame}
  \frametitle{Decision Boundary \cont}

  \begin{ovalblock}{Example cont.}
    \footnotesize
    \begin{eqnarray*}
      F(\vec x) &=&
        \log\frac{p(y=0)}{p(y=1)} +
        \log
        \frac{
          \frac{1}{ \sqrt{ \det{( 2 \pi \mat{\Sigma}_0)}} }
          e^{ -\frac{1}{2} (\vec{x} - \mat{\mu}_0)^T \mat{\Sigma}_0^{-1} (\vec{x} - \vec{\mu}_0) }         
        }{
          \frac{1}{ \sqrt{ \det{(2 \pi \mat{\Sigma}_1)}} }
          e^{ -\frac{1}{2} (\vec{x} - \vec{\mu}_1)^T \mat{\Sigma}_1^{-1} (\vec{x} - \vec{\mu}_1)}
        }
    \end{eqnarray*}
    \pause
 
    This function has the constant component:
% 
    \begin{eqnarray*}
      c &=& \log \frac{p(y=0)}{p(y=1)}+\frac{1}{2}
            \log \frac{ { \det{(2\pi\mat{\Sigma}_1)}}}
                      { { \det{(2\pi\mat{\Sigma}_0)}}}
    \end{eqnarray*}
    \pause 
    
    We observe:
    \begin{itemize}
      \small
      \item Priors imply a constant offset of the decision boundary. 
      \item If priors and covariance matrices of both classes are identical, this offset is $c=0$.
    \end{itemize}
  \end{ovalblock}
\end{frame}


\begin{frame}
  \frametitle{Decision Boundary \cont}

  \begin{ovalblock}{Example cont.}
    \small
    Furthermore we have:
    \begin{eqnarray*}
      \log{ 
        \frac{e^{ - \frac{1}{2} (\vec{x} - \vec{\mu}_0)^T \mat{\Sigma}_0^{-1} (\vec{x} - \vec{\mu}_0)} }
             {e^{ - \frac{1}{2} (\vec{x} - \vec{\mu}_1)^T \mat{\Sigma}_1^{-1} (\vec{x} - \vec{\mu}_1)} }
      } & = & \\[.3cm]
      & & \pause \hspace{-5cm} =
      \frac{1}{2} 
      \left(
        (\vec{x} - \vec{\mu}_1)^T \mat{\Sigma}_1^{-1} (\vec{x} - \vec{\mu}_1) - (\vec{x} - \vec{\mu}_0)^T \mat{\Sigma}_0^{-1} (\vec{x} - \vec{\mu}_0)
      \right) \\[.3cm]
      & & \hspace{-5cm} \pause =
      \frac{1}{2} 
      \left(
        \vec{x}^T (\mat{\Sigma}_1^{-1} - \mat{\Sigma}_0^{-1}) \vec{x} 
        -2 (\vec{\mu}_1^T \mat{\Sigma}_1^{-1} - \vec{\mu}_0^T \mat{\Sigma}_0^{-1} ) \vec{x} + \right.\\
        & & \left. \hspace{-4cm} 
        + \vec{\mu}_1^T \mat{\Sigma}_1^{-1} \vec{\mu}_1 - \vec{\mu}_0^T \mat{\Sigma}_0^{-1} \vec{\mu}_0
      \right)
    \end{eqnarray*}
  \end{ovalblock}
\end{frame}
 
 
\begin{frame}
  \frametitle{Decision Boundary \cont}
  
  \begin{ovalblock}{Example cont.}
    \small
    Now we have:
    \begin{eqnarray*}
      \mat{A}        &=& \frac{1}{2} (\mat{\Sigma}_1^{-1} - \mat{\Sigma}_0^{-1}) \\[.5cm]
      \vec{\alpha}^T &=& \vec{\mu}_0^T \mat{\Sigma}_0^{-1} - \vec{\mu}_1^T \mat{\Sigma}_1^{-1} \\[.5cm]
      \alpha_0       &=& \log \frac{p(y=0)}{p(y=1)} + \frac{1}{2}
                         \left(
                           \log
                             \frac{ \det{( 2 \pi \mat{\Sigma}_1)}}
                                  { \det{( 2 \pi \mat{\Sigma}_0)}} + 
                             \vec{\mu}_1^T \mat{\Sigma}_1^{-1} \vec{\mu}_1 -
                             \vec{\mu}_0^T \mat{\Sigma}_0^{-1} \vec{\mu}_0
                         \right)
    \end{eqnarray*}
  \end{ovalblock}
\end{frame}
  
  
\begin{frame}
  \frametitle{Decision Boundary \cont}
  
  \begin{figure}
    
    \resizebox{.6\linewidth}{!}{
      \alt<4->{
        \input{\texfigdir/plot_decision_boundary.pstex_t}
      }
{\alt<3>{
        \input{\texfigdir/gaussian2.pstex_t}
      }{\alt<2>{
        \input{\texfigdir/gaussian1.pstex_t}
      }{
        \input{\texfigdir/gaussian0.pstex_t}
      }}}
    }
    \caption{Two Gaussian sample sets and the decision boundary}
  \end{figure}
\end{frame}
 
 
\begin{frame}
  \frametitle{Decision Boundary \cont}

  \structure{Quadratic polynomials in the 2 variables $x_1$ and $x_2$}

  \begin{eqnarray*}
    F(\vec{x}) 
      &=& \vec x^T \mat A \vec x + \vec \alpha^T\vec x + \alpha_0 \\
      &=& a x_1^2 + b x_1 x_2 + c x_2^2 + d x_1 + e x_2 + f \stackrel{!}{=} 0
  \end{eqnarray*}
  \vspace{-0.75cm}

  \begin{figure}
  \copyrightbox[b]{
    \subfloat[circles and ellipses]{
      \makebox[.3\linewidth]{
        \href{http://en.wikipedia.org/wiki/File:Conic_sections_with_plane.svg}{
          \includegraphics[height=3cm]{\pngdir/conic_section_circle_ellipse.\png}
        }
      }
    }
    \subfloat[parabolas]{
      \makebox[.3\linewidth]{
        \href{http://en.wikipedia.org/wiki/File:Conic_sections_with_plane.svg}{
          \includegraphics[height=3cm]{\pngdir/conic_section_parabola.\png}
        }
      }
    }
    \subfloat[hyperbolas]{
      \makebox[.3\linewidth]{
        \href{http://en.wikipedia.org/wiki/File:Conic_sections_with_plane.svg}{
          \includegraphics[height=3cm]{\pngdir/conic_section_hyperbola.\png}
        }
      }
    }
  }{Pbroks13,  \href{https://creativecommons.org/licenses/by/3.0} {CC BY 3.0}, via Wikimedia Commons}
  \end{figure}
\end{frame}


\begin{frame}
  \frametitle{Decision Boundary \cont}

  \structure{Posterior probability}

  \begin{center}
    \resizebox{.85\linewidth}{!}{
      \input{\texfigdir/plot_posterior.pstex_t}
    }
  \end{center}
\end{frame}

{
\usebackgroundtemplate{\includegraphics[width=\paperwidth]{png/nextTime.png}}
\begin{frame}[plain]
\end{frame}
}

 
\begin{frame}
  \frametitle{Decision Boundary in Distributions with Equal Dispersion}
 
  \begin{ovalblock}{Example cont.}
    \small
    If both classes share the same covariances i.\,e.\ $\mat\Sigma= \mat\Sigma_0=\mat\Sigma_1$, then the argument of the sigmoid function is linear in the components of $\vec{x}$.
   
    \begin{eqnarray*}
      \mat{A}        &=& \mat{0} \\[.5cm]
      \vec{\alpha}^T &=& (\vec{\mu}_0 - \vec{\mu}_1)^T \mat{\Sigma}^{-1} \\[.5cm]
      \alpha_0       &=& \log \frac{p(y=0)}{p(y=1)} + 
                         \frac{1}{2}(\vec{\mu}_1 - \vec{\mu}_0)^T \mat{\Sigma}^{-1} (\vec{\mu}_1 - \vec{\mu}_0)  
     \end{eqnarray*}
  \end{ovalblock}
\end{frame}


\begin{frame}
  \frametitle{Decision Boundary in Distributions with Equal Dispersion \cont}
   
  \begin{figure}
    \resizebox{.6\linewidth}{!}{
      \alt<2>{
        \input{\texfigdir/gaussian3.pstex_t}
      }{
        \input{\texfigdir/gaussian5.pstex_t}
      }
    }    
    \caption{Identical covariances lead to linear decision boundary}
  \end{figure}
\end{frame}


% \begin{frame}
%   \frametitle{Decision Boundary \cont}
%   
%   \begin{figure}
%     \includegraphics[width=0.7\textwidth]{\psdir/gauss_both.\ps}
%     \caption{Quadratic and linear decision boundary in comparison}
%   \end{figure}
% \end{frame}


\begin{frame}
  \frametitle{Decision Boundary in Distributions with Equal Dispersion \cont}

  \structure{Note:}
  
  \begin{itemize}
    \item If the class conditionals are Gaussians and share the same covariance, the argument of the exponential function is affine in $\vec x$.
    \item This result is even true for a more general family of pdfs and not limited to Gaussians.
  \end{itemize}
\end{frame}


\begin{frame}
  \frametitle{Decision Boundary in Distributions with Equal Dispersion \cont}

  \begin{citeblock}{Definition}
    
The {\em exponential family} is a class of pdf's that can be written in the following canonical form
    \begin{eqnarray*}
      p(\vec x;\vec \theta, \phi) &=& e^{\frac{\vec\theta^T \cdot \vec  x -b(\vec \theta)}{a(\phi)}+c(\vec x,\phi)}
    \end{eqnarray*}
    where $\vec \theta\in \real^d$ is the {\em location parameter vector}, $\phi$ the {\em dispersion parameter}.
  \end{citeblock}
\end{frame}


%  \begin{ovalblock}{Example}
%     Binomial, Poisson, hypergeometric, exponential probability density functions or Gaussians belong to the exponential family.
%  \end{ovalblock}


\begin{frame}
  \frametitle{Exponential Family}

  \structure{Gaussian Probability Density Function \phantom{\cont}}

  \begin{displaymath}
    \mathcal{N}(\vec{x}; \vec{\mu},\mat{\Sigma}) = \frac{1}{ \sqrt{ \det{(2 \pi \mat{\Sigma}_y})} } 
                    e^{-\frac{1}{2} (\vec{x} - \vec{\mu})^T \mat{\Sigma}^{-1} (\vec{x} - \vec{\mu})  } 
  \end{displaymath}
%
  \begin{figure}
    \centering
    \subfloat{
      \resizebox{.3\linewidth}{!}{
        \input{\texfigdir/gauss1.pstex_t}       
      }
    }
    \quad
    \subfloat{
      \resizebox{.3\linewidth}{!}{
        \input{\texfigdir/gauss2.pstex_t}
      }
    }
    \quad
    \subfloat{
      \resizebox{.3\linewidth}{!}{
        \input{\texfigdir/gauss3.pstex_t}
      }
    }

    \caption{Gaussian probability density functions with $\vec\mu = (0, 0)^T$}
  \end{figure}
\end{frame}


\begin{frame}
  \frametitle{Exponential Family}

  \structure{Gaussian Probability Density Function \cont}
%
  \begin{displaymath}
    \mathcal{N}(\vec{x}; \vec{\mu},\mat{\Sigma}) = \frac{1}{ \sqrt{ \det{(2 \pi \mat{\Sigma}_y})} } 
                    e^{-\frac{1}{2} (\vec{x} - \vec{\mu})^T \mat{\Sigma}^{-1} (\vec{x} - \vec{\mu})  } 
  \end{displaymath}
%
  \begin{figure}
    \centering
    \subfloat{
      \resizebox{.3\linewidth}{!}{
        \input{\texfigdir/gauss4.pstex_t}
      }
    }
    \quad
    \subfloat{
      \resizebox{.3\linewidth}{!}{
        \input{\texfigdir/gauss5.pstex_t}
      }
    }
    \quad
    \subfloat{
      \resizebox{.3\linewidth}{!}{
        \input{\texfigdir/gauss6.pstex_t}
      }
    }

    \caption{Gaussian probability density functions with $\vec\mu = (0, 0)^T$}
  \end{figure}
\end{frame}


\begin{frame}
  \frametitle{Exponential Family \cont}

  \structure{Exponential Probability Density Function}
%
  \begin{displaymath}
    f_\lambda(x) = \left\{ 
                 \begin{array}{ll}
                   \lambda e^{-\lambda x} & x \ge 0 \\
                   0                      & x < 0 \\
                 \end{array}
               \right.
  \end{displaymath}
%
  \begin{figure}
    \resizebox{.65\linewidth}{!}{
      \alt<4->{
        \input{\texfigdir/exponential4.pstex_t}
      }{\alt<3>{
        \input{\texfigdir/exponential3.pstex_t}
      }{\alt<2>{
        \input{\texfigdir/exponential2.pstex_t}
      }{
        \input{\texfigdir/exponential1.pstex_t}
      }}}
    }
    \caption{Exponential probability density functions}
  \end{figure}
\end{frame}


\begin{frame}
  \frametitle{Exponential Family \cont}

  \structure{Binomial Probability Mass Function}
%
  \begin{displaymath}
    B(k; p,n) = {n \choose k} p^k (1-p)^{n-k}
  \end{displaymath}
%
  \begin{figure}
    \resizebox{.65\linewidth}{!}{
      \alt<3->{
        \input{\texfigdir/binomial3.pstex_t}
      }{\alt<2>{
        \input{\texfigdir/binomial2.pstex_t}
      }{
        \input{\texfigdir/binomial1.pstex_t}
      }}
    }
    \caption{Binomial probability mass functions for $n=20$}
  \end{figure}
\end{frame}


\begin{frame}
  \frametitle{Exponential Family \cont}

  \structure{Poisson Probability Mass Function}
%
  \begin{displaymath}
    P_\lambda(X=k) = \frac{\lambda^k}{k!} e^{-\lambda}
  \end{displaymath}
%
  \begin{figure}
    \resizebox{.70\linewidth}{!}{
      \alt<3->{
        \input{\texfigdir/poisson3.pstex_t}
      }{\alt<2>{
        \input{\texfigdir/poisson2.pstex_t}
      }{
        \input{\texfigdir/poisson1.pstex_t}
      }}
    }
    \caption{Poisson probability mass functions}
  \end{figure}
\end{frame}

\begin{frame}
  \frametitle{Exponential Family \cont}

  \structure{Hypergeometric Probability Mass Function}

  \begin{displaymath}
    h(k; N,M,n) = \frac{{M \choose k}{N-M \choose n-k}}{{N \choose n}}
  \end{displaymath}
%
  \begin{figure}
    \resizebox{.65\linewidth}{!}{
      \alt<3->{
        \input{\texfigdir/hypergeometric3.pstex_t}
      }{\alt<2>{
        \input{\texfigdir/hypergeometric2.pstex_t}
      }{
        \input{\texfigdir/hypergeometric1.pstex_t}
      }}
    }
    \caption{Hypergeometric probability mass functions}
  \end{figure}
\end{frame}

\begin{frame}
   \frametitle{Decision Boundary \cont}

   \begin{lemma}
     If all class-conditional densities are members of the same exponential family of probability density functions with equal dispersion $\phi$, the decision boundary $F(\vec x)=0$ is linear in the components of $\vec x$.
   \end{lemma}
\end{frame}


\subsection{Lessons Learned}

\begin{frame}
  \frametitle{Lessons Learned}

  \begin{itemize}
    \item Posteriors can be rewritten in terms of a logistic function. \pause
    \item Given the decision boundary $F(\vec x)=0$, we can write down the posterior $p(y|\vec x)$ right away. \pause
    \item Decision boundary for normally distributed feature vectors for each class is a quadratic function. \pause
    \item If Gaussians share the same covariances, the decision boundary is a linear function.
  \end{itemize}
\end{frame}

{
\usebackgroundtemplate{\includegraphics[width=\paperwidth]{png/nextTime.png}}
\begin{frame}[plain]
\end{frame}
}


\subsection{Further Readings}

\begin{frame}
  \frametitle{Further Readings}
  
  \begin{itemize}
    \item T. Hastie, R. Tibshirani, and J. Friedman: \\
      \structure{The Elements of Statistical Learning --}\\
      \structure{ Data Mining, Inference, and Prediction},\\
      2nd edition, Springer, New York, 2009. \\[.3cm]
    \item David W. Hosmer, Stanley Lemeshow: \\
      \structure{Applied Logistic Regression}, 2nd Edition, \\
      John Wiley \& Sons, Hoboken, 2000.
  \end{itemize}
\end{frame}


\subsection{Comprehensive Questions}

\begin{frame}
  \frametitle{Comprehensive Questions}
  
  \begin{itemize}
    \item How can we model the posterior probabilities? \\[1cm] \pause
    \item Formulate the criterion for the decision boundary! \\[1cm] \pause
    \item Describe the shape of the decision boundary for a Gaussian with different and same class covariances! \\[1cm] \pause
    \item What effect does a change of the priors have on the decision boundary?
  \end{itemize}
\end{frame}



