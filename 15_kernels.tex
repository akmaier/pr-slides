\section{Kernels}

\subsection{Motivation}

\begin{frame}
	\frametitle{Motivation}

	Linear decision boundaries in its current form have serious limitations: \pause

	\begin{itemize}
		\item too simple to provide good decision boundaries \\[.3cm] \pause
		\item non-linearly separable data cannot be classified \\[.3cm] \pause
		\item noisy data cause problems \\[.3cm] \pause
		\item formulation deals with vectorial data only
	\end{itemize}
	\pspread

	\structure{Possible solution:} \\

	\begin{itemize}
		\item Map data into a higher dimensional feature space using a \\
		      \structure{non-linear feature transform},
		      then use a linear classifier.
	\end{itemize}
\end{frame}


\begin{frame}
	\frametitle{Dual Representation}

	\begin{itemize}
		\item The SVM \structure{decision boundary} can be rewritten in dual form:
		      {\small
		      \begin{displaymath}
			      f(\vec x) =
			      \vec \alpha^T\vec x +\alpha_0 =
			      \sum_i \lambda_i y_i \vec x_i^T \vec x + \alpha_0
		      \end{displaymath}
		      }
		      where we have used the identity:
		      {\small
		      \begin{displaymath}
			      \vec \alpha = \sum_i \lambda_i y_i \vec x_i~. \pause
		      \end{displaymath}
		      }
		\item The Lagrange dual problem is given by the \structure{optimization problem}:
		      {\small
		      \begin{eqnarray*}
			      \mbox{maximize}   & & -\frac{1}{2} \sum_i \sum_j \lambda_i \lambda_j y_i y_j \cdot \vec x_i^T\vec x_j + \sum_i \lambda_i \\
			      \mbox{subject to} & & \vec{\lambda}\succeq 0, \quad \sum_{i} \lambda_i \, y_i = 0 \pause
		      \end{eqnarray*}
		      }
	\end{itemize}

	\structure{Conclusion:}
	feature vectors $\vec x_i, \vec x_j$, and $\vec x$ only appear in \structure{inner products}, \\
	both in the learning and the classification phase.
\end{frame}


\begin{frame}
	\frametitle{Inner Product and the Perceptron}

	The decision boundary that we get for the perceptron can also be written in terms of inner products:

	\begin{eqnarray*}
		F(\vec x) \pause
		&=& \left (\sum_{i\in \cal E} y_i\cdot \vec x_i \right)^T \vec x + \sum_{i\in \cal E} y_i \quad \\[.5cm] \pause
		&=& \sum_{i\in \cal E} y_i\cdot \langle \vec x_i, \vec x\rangle + \sum_{i\in \cal E} y_i
	\end{eqnarray*}
\end{frame}


\subsection{Feature Transforms}

\begin{frame}{Feature Transforms}

	We select a \structure{feature transform $\phi: \real^d \rightarrow \real^D$}, $D \ge d$, such that \\
	the resulting features $\phi(\vec x_i)$, $i=1,2,\dots, m$, are linearly separable. \pause

	\begin{figure}
		\resizebox{.95\linewidth}{!}{
			\alt<4->{
				\input{\texfigdir/quadratic_features1b.pstex_t}
			}{
				\input{\texfigdir/quadratic_features1.pstex_t}
			}
			\qquad \qquad \qquad \qquad \pause
			\input{\texfigdir/quadratic_features2.pstex_t}
		}
		\caption{Application of the feature transform $\phi(\vec x) = \left({x_1}^2, {x_2}^2\right)^T$.}
	\end{figure}
\end{frame}


\begin{frame}{Feature Transforms \cont}

	\structure{Second Example:} data is not centered\\ \phantom{q}

	\begin{figure}
		\resizebox{1\linewidth}{!}{
			\input{\texfigdir/quadratic_features3.pstex_t}
			\qquad \qquad \qquad \qquad \pause
			\alt<3->{
				\input{\texfigdir/quadratic_features5.pstex_t}
			}{
				\input{\texfigdir/quadratic_features4.pstex_t}
			}
		}
		\caption{Application of the feature transform
			\alt<3->{
				$\phi(\vec x) = \left({x_1}^2, {x_2}^2, x_2\right)^T$.
			}{
				$\phi(\vec x) = \left({x_1}^2, {x_2}^2\right)^T$.
			}
		}
	\end{figure}
\end{frame}


\begin{frame}
	\frametitle{Feature Transforms \cont}

	\begin{ovalblock}{Example}
		Assume the decision boundary is given by the quadratic function
		\begin{displaymath}
			f(\vec x) = a_0 + a_1 {x_1}^2 + a_2 {x_2}^2 + a_3 x_1 x_2 + a_4 x_1+a_5 x_2.
		\end{displaymath}
		Obviously this is not a linear decision boundary.\\[.25cm] \pause

		By the following mapping, we get features that have a linear decision boundary:
		\footnotesize
		\begin{displaymath}
			\phi(\vec x) = \left(
			\begin{matrix}
				%1 \\
				{x_1}^2 \\[1mm] {x_2}^2 \\ x_1 \cdot x_2 \\ x_1 \\ x_2
			\end{matrix}
			\right)
		\end{displaymath}
	\end{ovalblock}
\end{frame}


\begin{frame}
	\frametitle{Feature Transforms \cont}

	Consider distances in the transformed feature space: \\
	%
	\begin{eqnarray*}
		\| \phi(\vec x)-\phi (\vec x')\|_2^2 \pause
		&=& \left\langle \big(\phi(\vec x)-\phi (\vec x')\big), \big(\phi(\vec x)-\phi (\vec x')\big)\right\rangle \\[.5cm] \pause
		&=& \langle \phi(\vec x), \phi(\vec x)\rangle -
		2 \langle \phi(\vec x), \phi(\vec x') \rangle +
		\langle \phi(\vec x'), \phi(\vec x') \rangle
	\end{eqnarray*}
	\pause

	\structure{Conclusion:} Distances can be computed by just evaluating inner products.
\end{frame}


\begin{frame}
	\frametitle{Feature Transforms \cont}

	These feature transforms can be easily incorporated into SVMs: \\[.25cm]

	\begin{itemize}
		\item \structure{Decision boundary:}
		      \begin{center}
			      \small
			      \tikz[baseline]{
				      \node[fill=bl1!100,anchor=base,rounded corners=3pt] (d1) {
					      \color{bl3}
					      $\displaystyle
						      f(\vec x) = \sum_i \lambda_i \cdot y_i \cdot  \langle\phi(\vec x_i), \phi( \vec x)\rangle + \alpha_0
					      $
				      };
			      }
		      \end{center}
		      \pause \vspace{.25cm}
		\item The Lagrange dual problem is given by the \structure{optimization problem}:
		      {\small
		      \begin{eqnarray*}
			      \mbox{maximize}   & & -\frac{1}{2} \sum_i \sum_j \lambda_i \lambda_j y_i y_j \cdot \langle\phi(\vec x_i),\phi(\vec x_j)\rangle + \sum_i \lambda_i \\[.5cm]
			      \mbox{subject to} & & \vec{\lambda}\succeq 0, \quad \sum_{i} \lambda_i \, y_i = 0
		      \end{eqnarray*}
		      }
	\end{itemize}
\end{frame}


\subsection{Kernel Functions}

\begin{frame}
	\frametitle{Kernel Functions}

	\begin{citeblock}{Definition}

		A \structure{\emph{kernel function}} $k: {\cal X} \times {\cal X}\rightarrow \real$ is a symmetric function that maps a pair of features to a real number. For a kernel function the following property holds:
		\begin{displaymath}
			k(\vec x, \vec x') = \langle\phi(\vec x), \phi(\vec x')\rangle
		\end{displaymath}
		for any feature mapping $\phi$.
	\end{citeblock}
	\pspread

	\structure{Note:} \\[.2cm]

	Usually the evaluation of the kernel function is much easier than the computation of transformed features followed by the inner product.
\end{frame}


\begin{frame}
	\frametitle{Kernel Functions \cont}

	\begin{citeblock}{Definition}

		For a given set of feature vectors $\vec x_1, \vec x_2, \dots, \vec x_m$, we define the \\
		\structure{\emph{kernel matrix}}
		\begin{eqnarray*}
			\mat K &=& [K_{i,j}]_{i,j=1,2,\dots,m}~, \quad \mbox{where} \quad K_{i,j} = \langle\phi(\vec x_i), \phi(\vec x_j)\rangle.
		\end{eqnarray*}
	\end{citeblock}
	\pspread

	\structure{Note:} \\[.2cm]
	The entries of the matrix are similarity measures for transformed feature pairs.
\end{frame}


\begin{frame}
	\frametitle{Kernel Functions \cont}

	\begin{lemma}
		The kernel matrix is positive semidefinite.
	\end{lemma}
	\pause

	\vspace{.3cm}
	\structure{Proof:} We need to show $\forall \vec{x}:~\vec x^T \mat K \vec x \geq 0$~: \pause

	\small
	\begin{eqnarray*}
		\vec x^T \mat K \vec x  \pause
		&=& \sum_{i,j=1}^m x_ix_j K_{i,j} \quad \pause
		=  \quad \sum_{i,j=1}^m x_ix_j\ \langle\phi(\vec x_i), \phi(\vec x_j)\rangle \\ \pause
		&=& \sum_{i,j=1}^m  \langle x_i\phi(\vec x_i), x_j\phi(\vec x_j)\rangle \\ \pause
		&=& \left \langle \sum_{i=1}^m x_i\phi(\vec x_i), \sum_{j=1}^mx_j\phi(\vec x_j)\right\rangle \quad \pause
		=  \quad \left\|\sum_{i=1}^m x_i\phi(\vec x_i)\right\|_2^2 \quad
		\geq \quad 0
	\end{eqnarray*}
\end{frame}


\begin{frame}
	\frametitle{Kernel Functions \cont}

	\structure{Typical kernel functions:} \pause

	\begin{itemize}
		\item Linear: $k(\vec x, \vec x') = \langle\vec x, \vec x'\rangle$ \\[.3cm] \pause
		\item Polynomial: $k(\vec x, \vec x') = (\langle\vec x, \vec x'\rangle+1)^d$ \\[.3cm] \pause
		\item Laplacian radial basis function: $k(\vec x, \vec x') = e^{-\frac{\|\vec x- \vec x'\|_1}{\sigma^2}}$ \\[.3cm] \pause
		\item Gaussian radial basis function: $k(\vec x, \vec x') = e^{-\frac{\|\vec x- \vec x'\|_2^2}{\sigma^2}}$ \\[.3cm] \pause
		\item Sigmoid kernel: $k(\vec x, \vec x') =\mbox{tanh}({ \alpha\langle\vec x, \vec x'\rangle+\beta})$
	\end{itemize}
	\pspread

	\structure{Question:} \\[.1cm]
	Can we compute for any kernel function $k(\vec x, \vec x')$ a feature mapping $\phi$ \\
	such that the kernel function can be written as an inner product?
\end{frame}


\begin{frame}
	\frametitle{Kernel Functions \cont}

	\begin{theorem}[Mercer's Theorem]
		For any symmetric function $k: {\cal X}\times {\cal X}\rightarrow \real$ that is square integrable on its domain and which satisfies

		\begin{displaymath}
			\int_{{\cal X}\times {\cal X}} f(\vec x)\,f(\vec x')\,k(\vec x, \vec x')\ \mathsf{d}\vec x\, \mathsf{d} \vec x' \geq 0
		\end{displaymath}

		for all square integrable functions $f$, there exist transforms $\phi_i: {\cal X}\rightarrow \real$ and $\lambda_i\geq 0$ such that:

		\begin{displaymath}
			k(\vec x, \vec x') = \sum_i \lambda_i \,\phi_i(\vec x)\,\phi_i(\vec x')
		\end{displaymath}

		for all $\vec x$ and $\vec x'$.
	\end{theorem}
\end{frame}


\begin{frame}
	\frametitle{Kernel Functions \cont}

	\structure{The Kernel Trick} \\[.15cm]

	In \emph{any} algorithm that is formulated in terms of a positive semidefinite kernel $k$, we can derive an alternative algorithm by replacing the kernel function $k$ by another positive semidefinite kernel $k'$.

	\begin{center}%
		\resizebox{1\linewidth}{!}{%
			\alt<2->{%
				\input{\texfigdir/kernel+linear-svm2.pstex_t}%
			}{%
				\input{\texfigdir/kernel+linear-svm1.pstex_t}%
			}
		}
	\end{center}
\end{frame}


\mode<beamer>{
	\begin{frame}
		\frametitle{Kernel SVMs with Soft Margins}

		\structure{Linear kernel $\langle\vec x, \vec x'\rangle$:}
		\begin{itemize}
			\item the complexity parameter $C$ controls the number of support vectors and
			\item hence the width of the margin and
			\item the orientation of the decision boundary \\[.25cm]
		\end{itemize}

		\begin{figure}
			\centering
			\resizebox{.45\linewidth}{!}{
				\alt<3->{
					\input{\texfigdir/svm-linear3.pstex_t}
				}{\alt<2>{
						\input{\texfigdir/svm-linear2.pstex_t}
					}{
						\input{\texfigdir/svm-linear1.pstex_t}
					}}
			}
			\alt<3->{
				\caption{$C=0.1$: 17 support vectors, 3 misclassifications}
			}{\alt<2>{
					\caption{$C=1$: 11 support vectors, 4 misclassifications}
				}{
					\caption{$C=10$: 8 support vectors, 3 misclassifications}
				}}
		\end{figure}
	\end{frame}
}

\mode<handout>{
	\begin{frame}
		\frametitle{Kernel SVMs with Soft Margins}

		\structure{Linear kernel $\langle\vec x, \vec x'\rangle$:}
		\begin{itemize}
			\item the complexity parameter $C$ controls the number of support vectors and
			\item hence the width of the margin and
			\item the orientation of the decision boundary
		\end{itemize}

		\begin{figure}
			\subfloat[$C=10$: 8 support vectors, \newline 3 misclassifications]{
				\resizebox{1\linewidth}{!}{
					\input{\texfigdir/svm-linear1.pstex_t}
				}
			}
			\subfloat[$C=1$: 11 support vectors, \newline 4 misclassifications]{
				\resizebox{1\linewidth}{!}{
					\input{\texfigdir/svm-linear2.pstex_t}
				}
			}
			\subfloat[$C=0.1$: 17 support vectors, \newline 3 misclassifications]{
				\resizebox{1\linewidth}{!}{
					\input{\texfigdir/svm-linear3.pstex_t}
				}
			}
		\end{figure}
	\end{frame}
}


\mode<beamer>{
	\begin{frame}
		\frametitle{Kernel SVMs with Soft Margins \cont}

		\structure{Polynomial kernel $\langle\vec x, \vec x'\rangle^2$:}{}

		\begin{figure}
			\centering
			\resizebox{.55\linewidth}{!}{%
				\alt<2->{%
					\input{\texfigdir/svm-polynomial2.pstex_t}%
				}{%
					\input{\texfigdir/svm-polynomial1.pstex_t}%
				}
			}
			\alt<2>{
				\caption{$C=1$: 5 support vectors, 0 misclassifications}
			}{
				\caption{$C=10$: 4 support vectors, 0 misclassifications}
			}
		\end{figure}
	\end{frame}
}


\mode<handout>{
	\begin{frame}
		\frametitle{Kernel SVMs with Soft Margins \cont}

		\structure{Polynomial kernel $\langle\vec x, \vec x'\rangle^2$:}

		\vspace{-.15cm}
		\begin{figure}
			\subfloat[$C=10$: 4 support vectors, 0 misclassifications]{
				\resizebox{.55\linewidth}{!}{
					\input{\texfigdir/svm-polynomial1.pstex_t}
				}
			}
			\qquad
			\subfloat[$C=1$: 5 support vectors, 0 misclassifications]{
				\resizebox{.55\linewidth}{!}{
				\input{\texfigdir/svm-polynomial2.pstex_t}
				}
			}
		\end{figure}
	\end{frame}
}


\begin{frame}
	\frametitle{Kernel SVMs with Soft Margins \cont}

	\structure{Gaussian RBF kernel $e^{-0.1 \cdot \langle\vec x, \vec x'\rangle^2}$:}{}

	\begin{figure}
		\centering
		\resizebox{.7\linewidth}{!}{
			\alt<2->{
				\input{\texfigdir/svm-rbf2.pstex_t}
			}{
				\input{\texfigdir/svm-rbf1.pstex_t}
			}
		}
		\caption{$C=10$: 18 support vectors, 3 misclassifications}
	\end{figure}
\end{frame}


{
\usebackgroundtemplate{\includegraphics[width=\paperwidth]{png/nextTime.png}}
\begin{frame}[plain]
\end{frame}
}

%\subsection{Kernel PCA}

\begin{frame}
	\frametitle{Kernel PCA}

	\structure{PCA revisited} \pause

	\begin{itemize}
		\item Let $\vec x_1, \vec x_2, \dots, \vec x_m\in \real^d$ be the feature vectors with zero mean. \pause
		\item Compute the \structure{scatter matrix} (covariance matrix):
		      \begin{displaymath}
			      \mat \Sigma =
			      \frac{1}{m}\sum_{i=1}^m \vec x_i \vec x_i^T \quad
			      \in \quad \real^{d\times d} \pause
		      \end{displaymath}
		\item Compute the \structure{eigenvectors} and \structure{eigenvalues}:
		      \begin{displaymath}
			      \mat \Sigma \vec e_i = \lambda_i \vec e _i \pause
		      \end{displaymath}
		\item \structure{Sort} eigenvectors with decreasing eigenvalues. \pause
		\item Subsequent \structure{projection} of features to the eigenvectors.
	\end{itemize}
\end{frame}


\begin{frame}
	\frametitle{Kernel PCA \cont}

	\structure{Facts from linear algebra:} \\[.5cm] \pause

	\begin{itemize}
		\item The eigenvectors $\vec e_i$ span the same space as the feature vectors. \\[.5cm] \pause
		\item Each eigenvector $\vec e_i$ can be written as a linear combination of \\
		      feature vectors:
		      \begin{displaymath}
			      \vec e_i = \sum_{k} \alpha_{i,k} \vec x_k
		      \end{displaymath}
	\end{itemize}
\end{frame}


\begin{frame}
	\frametitle{Kernel PCA \cont}

	The eigenvector/-value problem for the PCA computation can now be rewritten:

	\begin{eqnarray*}
		\mat \Sigma \vec e_i
		&=& \lambda_i \vec e _i \\[.5cm] \pause
		\left( \frac{1}{m}\sum_{j=1}^m \vec x_j \vec x_j^T\right) \cdot \sum_k \alpha_{i,k} \vec x_k
		&=& \lambda_i \sum_{k} \alpha_{i,k} \vec x_k \\[.5cm] \pause
		\sum_{j,k} \alpha_{i,k} \vec x_j \vec x_j^T\vec x_k
		&=& m \cdot \lambda_i \sum_{k} \alpha_{i,k} \vec x_k
	\end{eqnarray*}
\end{frame}


\begin{frame}
	\frametitle{Kernel PCA \cont}

	\begin{itemize}
		\item The following equations have to be fulfilled for all projections on $\vec x_l$ for all indices $l$:

		      \begin{displaymath}
			      \sum_{j,k} \alpha_{i,k}\vec x_l^T \vec x_j \vec x_j^T\vec x_k =
			      m \cdot \lambda_i \sum_{k} \alpha_{i,k} \vec x_l^T\vec x_k \pause
		      \end{displaymath}
		      %
		\item \structure{Wow} -- now all feature vectors show up in terms of inner products and the kernel trick can be applied, \structure{if \emph{transformed} features $\phi(\vec x)$ have zero mean}. \\[.5cm] \pause

		\item For \emph{any} kernel $k(\vec x, \vec x')$, we get the \structure{key equation for Kernel PCA}: \\[.25cm]

		      \begin{center}
			      \tikz[baseline]{
				      \node[fill=bl1!100,anchor=base,rounded corners=3pt] (d1) {
					      \color{bl3}
					      $\displaystyle \sum_{j,k} \alpha_{i,k} \cdot k(\vec x_l, \vec x_j) \cdot k( \vec x_j,\vec x_k) =
						      m \cdot \lambda_i \cdot \sum_{k} \alpha_{i,k} \cdot k(\vec x_l, \vec x_k)$
				      };
			      }
		      \end{center}

	\end{itemize}
\end{frame}


\begin{frame}
	\frametitle{Kernel PCA \cont}

	This can be written in \structure{matrix notation} using the symmetric, \\
	positive semi-definite kernel matrix $\mat K\in \real^{m\times m}$ and \\
	the vector $\vec \alpha_i = (\alpha_{i,1}, \alpha_{i,2}, \dots, \alpha_{i,m})^T$:

	\begin{eqnarray*}
		\mat K^2 \vec \alpha_i        &=& m\cdot \lambda_i \mat K \vec \alpha_i\\
		\mat K (\mat K \vec \alpha_i) &=& m\cdot \lambda_i ( \mat K \vec \alpha_i)
	\end{eqnarray*}
	\pause

	This is equivalent to
	\begin{displaymath}
		\mat K (\mat K {\vec \alpha}_i - m\cdot \lambda_i \vec \alpha_i) = \vec 0
	\end{displaymath}

	\begin{itemize}
		\item $\mat K \vec \alpha_i$ is an eigenvector of $\mat K$
		\item $\vec \alpha_i$ is an eigenvector of $\mat K$
	\end{itemize}
\end{frame}


\begin{frame}
	\frametitle{Kernel PCA \cont}

	The coefficient vector $\vec \alpha_i$ for the principal components can be computed by solving the \structure{eigenvalue/-vector problem} for $i$: \\[.25cm]

	\begin{center}
		\tikz[baseline]{
			\node[fill=bl1!100,anchor=base,rounded corners=3pt] (d1) {
				\color{bl3}
				$\mat K {\vec \alpha}_i = m \lambda_i  \ \vec \alpha_i$
			};
		}
	\end{center}
	\pause

	\structure{Note:}

	\small
	\begin{itemize}
		\item Kernel PCA (and thus the classical PCA as well) can be computed by \\
		      solving an eigenvector/-value problem for an $(m\times m)$-matrix, \\
		      where $m$ is the \structure{cardinality of the training feature set}. \pause
		\item The principal components cannot be computed easily, because \\
		      only the kernel is known, but \structure{not} $\phi(\vec x)$.  \pause
		\item However, the projection $c$ of the transformed feature vector $\phi(\vec x)$ on \\
		      principal component $\vec e_i= \sum_{k} \alpha_{i,k}  \phi(\vec x_k)$ is easily computed by:
		      \begin{displaymath}
			      c \quad = \quad
			      \phi( \vec x)^T \vec e_i \pause \quad = \quad
			      \sum_{k} \alpha_{i,k} \phi(\vec x)^T\phi(\vec x_k) \quad \pause = \quad
			      \sum_{k} \alpha_{i,k} k(\vec x, \vec x_k)
		      \end{displaymath}
	\end{itemize}
\end{frame}


\begin{frame}
	\frametitle{Kernel PCA \cont}

	It is assumed that the transformed features have \structure{zero mean}:

	\begin{displaymath}
		\frac{1}{m}\sum_{k=1}^m \phi(\vec x_k) = \vec 0. \pause
	\end{displaymath}

	This can be enforced by the \structure{normalization step}:
	%%
	\begin{eqnarray*}
		\tilde\phi(\vec x_i) &=& \phi(\vec x_i) - \frac{1}{m}\sum_{k=1}^m \phi(\vec x_k)
	\end{eqnarray*}
\end{frame}


\begin{frame}
	\frametitle{Kernel PCA}

	Implication for the components of the \structure{centered kernel matrix $\tilde {\mat K}$}:

	\small
	\begin{eqnarray*}
		\tilde{K} _{i,j}
		&=& \tilde\phi(\vec x_i) ^T  \tilde\phi(\vec x_j) \\ \pause
		&=& \left( \phi(\vec x_i) - \frac{1}{m}\sum_{k=1}^m \phi(\vec x_k) \right)^T
		\left( \phi(\vec x_j) - \frac{1}{m}\sum_{k=1}^m \phi(\vec x_k) \right) \\ \pause
		&=& \phi(\vec x_i)^T \phi(\vec x_j) -
		\frac{1}{m}\sum_{k=1}^m \phi(\vec x_i)^T \phi(\vec x_k) -
		\frac{1}{m}\sum_{k=1}^m \phi(\vec x_k)^T  \phi(\vec x_j) + \\
		& & \quad + \frac{1}{m^2}\sum_{k,l=1}^m \phi(\vec x_k)^T \phi(\vec x_l) \\ \pause
		&=& K_{i,j} -
		\frac{1}{m}\sum_{k=1}^m K_{i,k} -
		\frac{1}{m}\sum_{k=1}^m K_{k,j} +
		\frac{1}{m^2}\sum_{k,l=1}^m K_{k,l}
	\end{eqnarray*}
\end{frame}


\begin{frame}
	\frametitle{Kernel PCA \cont}

	\begin{ovalblock}{Example: classical vs. kernel PCA}
		\small
		Consider $m=50$ images with $1024^2$ pixels.
		The pixels define $1024^2$-dimensional feature vectors:
		\begin{displaymath}
			\vec x_1, \vec x_2, \dots, \vec x_{50} \in \real^{2^{20}} \pause
		\end{displaymath}

		The kernel PCA using the linear kernel
		\begin{displaymath}
			k(\vec x_i, \vec x_j) = \vec x_i^T \vec x_j
		\end{displaymath}
		requires the eigenvalue/-vector decomposition of the $(50\times 50)$ kernel matrix. \\[.5cm] \pause

		The classical PCA requires the eigenvalue/-vector decomposition of a $(2^{20} \times 2^{20})$ matrix.
	\end{ovalblock}
\end{frame}


\subsection{String Kernels}

\begin{frame}
	\frametitle{Kernels for Feature Sequences}

	\begin{ovalblock}{Example: string kernels}
		\begin{itemize}
			\item In speech recognition we do not have feature vectors but sequences of feature vectors.
			\item In order to use kernel methods we need a kernel for time series.
		\end{itemize}

		\begin{center}
			\movie[]{
				\includegraphics[height=2.0cm]{\pngdir/aibo.\png}
			}{audio/aibo.aif}
			\quad
			\includegraphics[height=2.0cm]{\pngdir/spectrum.\png}
		\end{center}
		~
	\end{ovalblock}
\end{frame}


\begin{frame}
	\frametitle{Kernels for Feature Sequences \cont}

	\begin{ovalblock}{Example: string kernels (cont.)}
		\begin{itemize}
			\item Feature vectors are considered in $\real^d =\cal X$.
			\item Sequences of feature vectors are elements of $\cal X^*$.
			\item \structure{Problem:} How to define a kernel over the sequence space $\cal X^*$? \\[.5cm]
		\end{itemize}

		\structure{Implications:}
		\begin{itemize}
			\item PCA on feature sequences -- COOL!
			\item SVM for feature sequences -- EVEN COOLER!
		\end{itemize}
	\end{ovalblock}
\end{frame}


\begin{frame}
	\frametitle{Kernels for Feature Sequences \cont}

	\begin{ovalblock}{Example: string kernels (cont.)}
		Comparison of sequences via \emph{dynamic time warping} (DTW): \\[.5cm]

		Given the feature sequences $(p, q\in \{1,2,\dots\})$:
		\begin{eqnarray*}
			\langle \vec x_1, \vec x_2, \dots, \vec x_p\rangle & \in & {\cal X^*} \\
			\langle \vec y_1, \vec y_2, \dots, \vec y_q\rangle & \in & {\cal X^*}
		\end{eqnarray*}
	\end{ovalblock}
\end{frame}


\begin{frame}
	\frametitle{Kernels for Feature Sequences \cont}

	\begin{ovalblock}{Example: string kernels (cont.)}
		\begin{itemize}
			\item Distance is computed by DTW:
			      \begin{displaymath}
				      D( \langle \vec x_1, \vec x_2, \dots, \vec x_p\rangle,
				      \langle \vec y_1, \vec y_2, \dots, \vec y_q\rangle )
				      = \frac{1}{p}\sum_{k=1}^{p} \| \vec x_{v(k)}- \vec y_{w(k)} \|_2
			      \end{displaymath}
			      where $v, w$ define the mapping of indices to indices. \\[.5cm] \pause
			\item The DTW kernel can be defined as:
			      \begin{displaymath}
				      k(\vec x, \vec y) = e^{-D(
						      \langle \vec x_1, \vec x_2, \dots, \vec x_p\rangle,
						      \langle \vec y_1, \vec y_2, \dots, \vec y_q\rangle) }
			      \end{displaymath}
		\end{itemize}
	\end{ovalblock}
\end{frame}


%\subsection{Fisher Kernels}

\begin{frame}
	\frametitle{Fisher Kernels}

	Now we design kernels building on probability density functions $p(\vec x;\vec  \theta)$.

	\begin{itemize}
		\item \structure{Fisher score:}
		      \begin{displaymath}
			      \mat J_{\vec \theta} (\vec x) =
			      -\frac{\partial}{\partial \vec \theta} \log p(\vec x; \vec \theta) \pause
		      \end{displaymath}
		\item \structure{Fisher information matrix:}
		      \begin{displaymath}
			      \mat I (\vec x) =
			      E_{\vec x} [\mat J_{\vec \theta}(\vec x) \mat J_{\vec \theta}^T(\vec x)]
		      \end{displaymath}
	\end{itemize}
	\pspread

	\structure{Note:} \\[.15cm]

	The Fisher information matrix is the curvature of the Kullback-Leibler divergence.
\end{frame}


\begin{frame}
	\frametitle{Fisher Kernels \cont}

	The Fisher kernel can be defined in \structure{two different ways}:

	\begin{displaymath}
		k(\vec x, \vec x') = \mat J_{\vec \theta}^T(\vec x) \mat J_{\vec \theta}(\vec x')
	\end{displaymath}

	or

	\begin{displaymath}
		k(\vec x, \vec x') =  \mat J_{\vec \theta}^T(\vec x)    \mat I ^{-1}(\vec x) \mat J_{\vec \theta}(\vec x')
	\end{displaymath}
\end{frame}


\begin{frame}
	\frametitle{Fisher Kernels \cont}

	\structure{Application:} learning from partially labeled data \pause

	\small
	\begin{itemize}
		\item Some classification approaches require \structure{huge collections of data} \\
		      (e.\,g.\ for text or speech recognition). \pause
		\item Labeling of the data can be \structure{time-consuming} and \structure{costly}. \pause
		\item If the data can be modeled with a small number of well separated components \\
		      (with each component corresponding to a distinct category), \\
		      little labeled data would suffice to assign a proper label to each of them. \pause
		\item A machine learning approach that makes use of only partially labeled data \\
		      usually achieves much better classification performance than \\
		      using only the labeled data alone. \pause
		\item Fisher kernels describe a generative model that can be used in a discriminative approach (e.\,g.\ SVM).
	\end{itemize}
\end{frame}


\subsection{Lessons Learned}

\begin{frame}
	\frametitle{Lessons Learned}

	\begin{itemize}
		\item Limitations of linear decision boundaries \\[.5cm]
		\item Non-linear feature transforms \\[.5cm]
		\item Kernel function and kernel matrix \\[.5cm]
		\item Kernel trick \\[.5cm]
		\item Probabilities and kernels
	\end{itemize}
\end{frame}

{
\usebackgroundtemplate{\includegraphics[width=\paperwidth]{png/nextTime.png}}
\begin{frame}[plain]
\end{frame}
}


%\subsection{Further Readings}

\begin{frame}
	\frametitle{Further Readings}

	\begin{itemize}
		\item Bernhard Sch{\"o}lkopf, Alexander J. Smola: \\
		      \structure{Learning with Kernels}, \\
		      The MIT Press, Cambridge, 2003. \\[.25cm]
		\item Vladimir N. Vapnik: \\
		      \structure{The Nature of Statistical Learning Theory}, \\
		      Information Science and Statistics, Springer, Heidelberg, 2000. \\[.25cm]
		\item John Shawe-Taylor, Nello Cristianini: \\
		      \point \href{https://www.google.de/search?hl=de&q=Kernel\%20Methods\%20for\%20Pattern\%20Analysis+pdf}
		      {\structure{Kernel Methods for Pattern Analysis}}, \\
		      Cambridge University Press, Cambridge, 2004
	\end{itemize}
\end{frame}


\subsection{Comprehensive Questions}

\begin{frame}
	\frametitle{Comprehensive Questions}

	\begin{itemize}
		\item What are the properties of kernel functions? \\[1cm]
		\item What is the kernel matrix? \\[1cm]
		\item What is the kernel trick? \\[1cm]
		\item How can we use kernels for string comparison?
	\end{itemize}
\end{frame}
%
%
% \mode<beamer>{
%   \begin{frame}[plain]
%     \begin{figure}
%       \resizebox{!}{8.5cm}{
%         \alt<3->{
%           \input{\texfigdir/svm-rbf.pstex_t}
%         }{\alt<2>{
%           \input{\texfigdir/svm-polynomial1.pstex_t}
%         }{
%           \input{\texfigdir/svm-linear1.pstex_t}
%         }}
%       }
%     \end{figure}
%   \end{frame}
% }
