\definecolor{brown}{rgb}{0.9059,0.8667,0.7725}
\definecolor{bl1}{rgb}{0.9020,0.9216,0.9412}
\definecolor{bl2}{rgb}{0.8000,0.8431,0.8824}
\definecolor{bl3}{rgb}{0.0000,0.2000,0.4000}
\definecolor{gr1}{rgb}{0.9020,0.9529,0.9020}
\definecolor{gr2}{rgb}{0.8000,0.9020,0.8000}
\definecolor{gr3}{rgb}{0.2471,0.6235,0.2471}
\definecolor{gray4}{rgb}{0.2431,0.2431,0.2431}
\definecolor{gray3}{rgb}{0.3098,0.3098,0.3098}
\definecolor{gray2}{rgb}{0.4863,0.4863,0.4863}
\definecolor{gray1}{rgb}{0.7725,0.7725,0.7725}
%
\def\vec#1{\ensuremath{{\bm #1}}}
\def\mat#1{\vec{#1}}
\newcommand{\symb}{\Pisymbol{psy}}
\newcommand{\rb}[1]{\raisebox{1.5ex}[-1.5ex]{#1}}
\def\spread{\vspace*{\fill}}
\def\pspread{\pause\spread}
\def\forts{{\small (Forts.)}}
\def\cont{{\small (cont.)}}
\def\check{\structure{\ding{51}}}
\def\Check{{\color{gray}\ding{51}}}
\def\point{\structure{\ding{43}}\hspace{0.5em}}
\def\vorsicht{\includegraphics[width=1em]{\pngdir/gefahrenstelle.\png}\hspace{0.5em}}
\def\stopp#1{\pause\vspace*{#1}}
\def\real{\mathbb{R}}
\def\argmin{\mathop{\mathsf{argmin}}}
\def\argmax{\mathop{\mathsf{argmax}}}
\def\sign{{\mathord{\mathsf{sign}}}}
\def\figurename{Fig.}
\def\tablename{Tab.}

%set colors
\setbeamercolor{block title}{bg=faublue!20,fg=faublue}
\setbeamercolor{block body}{bg=faublue!10,fg=black}
\setbeamertemplate{blocks}[shadow=false]

\newcolumntype{x}[1]{>{\centering\arraybackslash\hspace{0pt}}p{#1}}

% neue Umgebungen

\newenvironment<>{ovalblock}[1]%
  {\setbeamertemplate{blocks}[rounded][shadow=\shadow]%
   \begin{block}{#1}}
  {\end{block}%
   \setbeamertemplate{blocks}[default]}


\newenvironment<>{codeblock}[1]%
  {\setbeamertemplate{blocks}[rounded][shadow=\shadow]%
   \begin{exampleblock}{#1}}
  {\end{exampleblock}%
   \setbeamertemplate{blocks}[default]}


\newenvironment<>{bearblock}[1]%
  {\setbeamertemplate{blocks}[rounded][shadow=\shadow]%
   \begin{alertblock}{#1}}
  {\end{alertblock}%
   \setbeamertemplate{blocks}[default]}


\newenvironment<>{citeblock}[2]%
  {\setbeamertemplate{blocks}[rounded][shadow=\shadow]%
	\begin{block}{#1\hfill{\tiny#2}}}
  {\end{block}%
	\setbeamertemplate{blocks}[default]}


% Leere Folien einfügen
\newcommand{\oneemptyslide}[0]{
  \mode<handout>{
    \setbeamercolor{background canvas}{bg=black!0}
    \begin{frame}[plain]
    \end{frame}
    \setbeamercolor{background canvas}{bg=black!5}

    \addtocounter{framenumber}{-1}
  }
}


\newcommand{\twoemptyslides}[0]{
  \ifdefined\fourpages
    \mode<handout>{
      \setbeamercolor{background canvas}{bg=black!0}
      \begin{frame}[plain]
      \end{frame}

      \begin{frame}[plain]
      \end{frame}
      \setbeamercolor{background canvas}{bg=black!5}

      \addtocounter{framenumber}{-2}
    }
  \fi
}


\newcommand{\threeemptyslides}[0]{
  \ifdefined\twopages
    \mode<handout>{
      \setbeamercolor{background canvas}{bg=black!0}
      \begin{frame}[plain]
      \end{frame}
      \setbeamercolor{background canvas}{bg=black!5}

      \addtocounter{framenumber}{-1}
    }
  \fi

  \ifdefined\fourpages
    \mode<handout>{
      \setbeamercolor{background canvas}{bg=black!0}
      \begin{frame}[plain]
      \end{frame}

      \begin{frame}[plain]
      \end{frame}

      \begin{frame}[plain]
      \end{frame}
      \setbeamercolor{background canvas}{bg=black!5}

      \addtocounter{framenumber}{-3}
    }
  \fi
}
